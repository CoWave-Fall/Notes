% mathnote-content.tex
% 数学辅助命令、代码高亮、彩色盒子、流程环境、TikZ 样式、
% 列表与表格微调,以及若干实用命令。

% --------------------------------------------------
% Math helpers
% --------------------------------------------------
\DeclarePairedDelimiter\abs{\lvert}{\rvert}
\DeclarePairedDelimiter\norm{\lVert}{\rVert}
\DeclarePairedDelimiter\ceil{\lceil}{\rceil}
\DeclarePairedDelimiter\floor{\lfloor}{\rfloor}

\newcommand{\R}{\mathbb{R}}
\newcommand{\C}{\mathbb{C}}
\newcommand{\Q}{\mathbb{Q}}
\newcommand{\Z}{\mathbb{Z}}
\newcommand{\N}{\mathbb{N}}
\newcommand{\dd}{\mathop{}\!\mathrm{d}}
\newcommand{\ee}{\mathrm{e}}
\newcommand{\dv}[2]{\frac{\dd #1}{\dd #2}}
\newcommand{\pdv}[2]{\frac{\partial #1}{\partial #2}}

% 代码高亮默认主题:浅色背景 + 柔和配色
\lstset{
  backgroundcolor=\color{surfacegrid},
  basicstyle=\ttfamily\small,
  keywordstyle=\color{screenTone04}\bfseries,
  commentstyle=\color{screenTone06},
  stringstyle=\color{screenTone12},
  frame=none,
  columns=fullflexible,
  showstringspaces=false
}

% 行内关键字与小提示
\NewDocumentCommand{\keyword}{m}{%
  \textcolor{accent}{\textbf{#1}}%
}

\NewDocumentCommand{\inlinehint}{m}{%
  \textcolor{secondary}{\sffamily\footnotesize #1}%
}

% 调色板色块展示:用于说明当前主题色
\NewDocumentCommand{\MathNotePaletteSwatch}{mm}{%
  \tikz[baseline=(label.base)]{
    \node[rounded corners=2pt, draw=#1!65!black, fill=#1, minimum width=0.85cm, minimum height=0.4cm] (chip) {};
    \node[right=0.28cm of chip, anchor=west, font=\sffamily\scriptsize\color{inkgray}] (label) {#2};
  }%
}

% 模式徽章:常用于“例题 / 练习 / 证明”等标记
\NewDocumentCommand{\ModeBadge}{O{accent}m}{%
  \tikz[baseline=(label.base)]\node[label/.style={}] (label) [inner xsep=6pt, inner ysep=1.6pt, rounded corners=2pt, fill=#1!12!white, draw=#1!80!black, font=\sffamily\scriptsize\bfseries\color{#1!30!black}] {#2};%
}

% 工具宏:若参数为空字符串则使用默认文本
\newcommand{\mathnote@ifblank}[3]{%
  \if\relax\detokenize{#1}\relax
    #2%
  \else
    #3%
  \fi
}

% 关键点列表:自动使用圆点 + 紧凑行距
\newenvironment{focuspoints}{%
  \begin{itemize}[label=\tikz{\filldraw[accent] (0,0) circle (2pt);}, leftmargin=1.8em, itemsep=0.2em, topsep=0.1em]
}{\end{itemize}}

% roadmap:步骤流程图形式的列表
\newcounter{roadmapstep}
\newlength{\mathnote@roadmaplabelsep}
\setlength{\mathnote@roadmaplabelsep}{0.9em}
\newlength{\mathnote@roadmaplastlabelwidth}
\newlength{\mathnote@roadmaplastht}
\newlength{\mathnote@roadmaparrowlen}
\setlength{\mathnote@roadmaplastlabelwidth}{0pt}
\setlength{\mathnote@roadmaplastht}{1.2em}
\setlength{\mathnote@roadmaparrowlen}{1.2em}
\newsavebox{\mathnote@roadmaplabelbox}
\newsavebox{\mathnote@roadmapcontentbox}
\newcommand{\mathnote@roadmaparrow}{%
  \setlength{\mathnote@roadmaparrowlen}{\mathnote@roadmaplastht}%
  \ifdim\mathnote@roadmaparrowlen<0.9em
    \setlength{\mathnote@roadmaparrowlen}{0.9em}%
  \fi
  \par\smallskip
  \noindent\hspace{\dimexpr0.5\mathnote@roadmaplastlabelwidth\relax}%
  \begin{tikzpicture}[x=1pt,y=1pt]
    \pgfmathsetlengthmacro{\MathNoteRoadmapArrowLen}{\mathnote@roadmaparrowlen}%
    \draw[accent, line width=0.85pt, -{Latex[length=3mm]}] (0,0) -- (0,-\MathNoteRoadmapArrowLen);
  \end{tikzpicture}%
  \par\smallskip
}
\NewDocumentEnvironment{roadmap}{O{}}{%
  \par\smallskip
  \setcounter{roadmapstep}{0}%
  \setlength{\mathnote@roadmaplastlabelwidth}{0pt}%
  \setlength{\mathnote@roadmaplastht}{1.2em}%
  \setlength{\mathnote@roadmaparrowlen}{1.2em}%
}{%
  \par\smallskip
}
\newcommand{\RoadmapStep}[1]{%
  \stepcounter{roadmapstep}%
  \sbox{\mathnote@roadmaplabelbox}{\ModeBadge[accent]{第\zhnumber{\arabic{roadmapstep}}步}}%
  \sbox{\mathnote@roadmapcontentbox}{%
    \begin{minipage}[t]{\dimexpr\linewidth-\wd\mathnote@roadmaplabelbox-\mathnote@roadmaplabelsep\relax}
      \raggedright\sloppy #1
    \end{minipage}%
  }%
  \ifnum\value{roadmapstep}>1
    \mathnote@roadmaparrow
  \fi
  \noindent
  \usebox{\mathnote@roadmaplabelbox}%
  \hspace{\mathnote@roadmaplabelsep}%
  \usebox{\mathnote@roadmapcontentbox}\par
  \setlength{\mathnote@roadmaplastlabelwidth}{\wd\mathnote@roadmaplabelbox}%
  \setlength{\mathnote@roadmaplastht}{\dimexpr\ht\mathnote@roadmapcontentbox+\dp\mathnote@roadmapcontentbox\relax}%
}

% 复用模块:总结盒中的步骤 Roadmap
\NewDocumentEnvironment{NoteRoadmap}{m}{%
  \begin{summarybox}{#1}%
    \begin{roadmap}%
}{%
    \end{roadmap}%
  \end{summarybox}%
}

% 复用模块:总结盒中的考前清单(多栏紧凑列表)
\NewDocumentEnvironment{NoteChecklist}{m}{%
  \begin{summarybox}{#1}%
    \begin{multicols}{2}%
      \begin{itemize}%
}{%
      \end{itemize}%
    \end{multicols}%
  \end{summarybox}%
}

% --------------------------------------------------
% Box styles
% --------------------------------------------------
\tcbset{
  mathnote box/.style={
    enhanced,
    sharp corners,
    boxrule=0.5pt,
    colback=surface,
    coltitle=inkgray,
    fonttitle=\sffamily\bfseries,
    left=1em,
    right=1em,
    top=0.7em,
    bottom=0.7em,
    before skip=10pt,
    after skip=10pt,
    breakable,
    width=\dimexpr\linewidth-\mathnote@boxindent\relax,
    left skip=\mathnote@boxindent,
    borderline west={1pt}{0pt}{accentline}
  }
}

\newtcolorbox{definitionbox}[2][]{%
  mathnote box,
  title=\mathnote@ifblank{#2}{定义}{#2},
  colback=surface,
  colframe=secondary!70!black,
  coltitle=secondary!15!white,
  fonttitle=\sffamily\bfseries\color{secondary!35!white},
  borderline west={2pt}{0pt}{secondary},
  #1
}

\newtcolorbox{theorembox}[2][]{%
  mathnote box,
  title=\mathnote@ifblank{#2}{定理}{#2},
  colback=surface,
  colframe=accent!70!black,
  coltitle=accent!10!white,
  fonttitle=\sffamily\bfseries\color{accent!35!white},
  borderline west={2pt}{0pt}{accent},
  #1
}

\newtcolorbox{examplebox}[2][]{%
  mathnote box,
  title=\mathnote@ifblank{#2}{例题}{#2},
  colback=surface,
  colframe=highlight!80!black,
  coltitle=highlight!15!white,
  fonttitle=\sffamily\bfseries\color{highlight!40!white},
  borderline west={2pt}{0pt}{highlight},
  #1
}

\newtcolorbox{lemmabox}[2][]{%
  mathnote box,
  title=\mathnote@ifblank{#2}{引理}{#2},
  colback=surface,
  colframe=inkline,
  coltitle=inkgray!30!white,
  fonttitle=\sffamily\bfseries\color{inkgray!45!white},
  borderline west={2pt}{0pt}{inkgray},
  #1
}

\newtcolorbox{notebox}[2][]{%
  mathnote box,
  title=\mathnote@ifblank{#2}{提示}{#2},
  colback=surface,
  colframe=highlight!60!black,
  coltitle=highlight!20!white,
  fonttitle=\sffamily\bfseries\color{highlight!45!white},
  borderline west={2pt}{0pt}{highlight},
  #1
}

\newtcolorbox{summarybox}[2][]{%
  mathnote box,
  title=\mathnote@ifblank{#2}{总结}{#2},
  colback=surface,
  colframe=accent!20!black,
  borderline west={2pt}{0pt}{accent},
  coltitle=accent!10!white,
  fonttitle=\sffamily\bfseries\color{accent!45!white},
  #1
}

\newtcolorbox{conceptbox}[2][]{%
  mathnote box,
  title=\mathnote@ifblank{#2}{概念骨架}{#2},
  colback=surface,
  colframe=secondary!40!black,
  coltitle=secondary!15!white,
  fonttitle=\sffamily\bfseries\color{secondary!40!white},
  borderline west={2pt}{0pt}{secondary},
  #1
}

\newtcolorbox{proofbox}[2][]{%
  mathnote box,
  title=\mathnote@ifblank{#2}{证明}{#2},
  colback=surface,
  colframe=inkline,
  coltitle=inkgray!35!white,
  fonttitle=\sffamily\bfseries\color{inkgray!60!white},
  borderline west={2pt}{0pt}{inkline},
  #1
}

\newtcolorbox{warningbox}[2][]{%
  mathnote box,
  title=\mathnote@ifblank{#2}{排版警示}{#2},
  colback=surface,
  colframe=highlight!80!black,
  borderline west={2pt}{0pt}{highlight},
  coltitle=highlight!20!white,
  fonttitle=\sffamily\bfseries\color{highlight!45!white},
  #1
}

\newtcolorbox{sideinfobox}[1][]{%
  mathnote box,
  colframe=surface,
  boxrule=0pt,
  left=1.6em, % 比普通盒子略多一点内缩
  borderline west={0pt}{0pt}{surface},% 灰竖线改由 overlay 绘制
  overlay={%
    % 灰色细竖线位置:靠近正文一侧
    \coordinate (SIBlineTop) at ([xshift=-0.2em,yshift=-2.0ex]interior.north west);
    \coordinate (SIBlineBottom) at ([xshift=-0.2em,yshift=-2.0ex]interior.south west);
    \draw[inkline, line width=0.6pt] (SIBlineTop) -- (SIBlineBottom);
    % 蓝色感叹号:相对 interior.north west 固定,基本贴着首行文字
    \node[anchor=base east, text=accent, font=\sffamily\bfseries\large]
      at ([xshift=-0.6em,yshift=-4.4ex]interior.north west) {!};
  },
  #1
}

% --------------------------------------------------
% TikZ styles
% --------------------------------------------------
\tikzset{
  mathnote lines/.style={
    line width=0.8pt,
    >=Stealth,
    draw=accentline,
    text=inkgray
  },
  mathnote grid/.style={
    color=inkgray!30,
    line width=0.3pt
  }
}

% --------------------------------------------------
% Tables and lists
% --------------------------------------------------
\newcommand{\mathnote@listbarbegin}[2][0.8em]{%
  \par\noindent
  \begin{tcolorbox}[
    blanker,
    enhanced jigsaw,
    breakable,
    sharp corners,
    boxrule=0pt,
    colback=surface,
    left=#1,
    right=0pt,
    top=0.25em,
    bottom=0.25em,
    borderline west={1.3pt}{0pt}{#2}
  ]%
  \ignorespaces
}
\newcommand{\mathnote@listbarend}{\end{tcolorbox}\ignorespacesafterend}

\setlist[itemize]{leftmargin=1.8em, itemsep=0.25em, before=\mathnote@listbarbegin{accent}, after=\mathnote@listbarend}
\setlist[enumerate]{leftmargin=2.1em, itemsep=0.3em, label=\textbf{\arabic*.}, before=\mathnote@listbarbegin[1em]{secondary}, after=\mathnote@listbarend}
\setlist[description]{font=\sffamily\bfseries, labelsep=0.5em}

\renewcommand{\arraystretch}{1.2}
\captionsetup{font=small, labelfont=bf}

% --------------------------------------------------
% Utility commands
% --------------------------------------------------
\newcommand{\ScreenOnly}[1]{\ifmathnoteprintmode\else #1\fi}
\newcommand{\PrintOnly}[1]{\ifmathnoteprintmode #1\fi}
\newcommand{\DualMode}[2]{\ifmathnoteprintmode #2\else #1\fi}

\NewDocumentCommand{\PageTag}{O{accent}m}{%
  \begin{tikzpicture}[remember picture, overlay]
    \node[anchor=north east, xshift=-8mm, yshift=-35mm, fill=#1, text=white, rounded corners=2pt, inner xsep=6pt, inner ysep=2pt, font=\sffamily\footnotesize] at (current page.north east) {#2};
  \end{tikzpicture}%
}

\newcommand{\SectionTag}[1]{%
  \textcolor{accent}{\Large\bfseries\sffamily #1}%
}
