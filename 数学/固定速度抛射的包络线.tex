% !TEX program = xelatex
% !TEX encoding = UTF-8
% ============================================
% 固定速度抛射的包络线
% 数学笔记 XeLaTeX 模板
% ============================================

\documentclass[a4paper, 11pt]{ctexart}

% ============================================
% 包的引入
% ============================================

% 数学公式与符号
\usepackage{amsmath, amssymb, amsthm, amsfonts}
\usepackage{mathtools} % 增强的数学工具

% 页面设置
\usepackage{geometry}
\geometry{left=2cm, right=2cm, top=2.5cm, bottom=2.5cm}

% 颜色支持
\usepackage{xcolor}

% 绘图支持
\usepackage{tikz}
\usetikzlibrary{arrows.meta, positioning, intersections, quotes, calc}

% 彩色文本框
\usepackage[most]{tcolorbox}

% 超链接
\usepackage[colorlinks=true, linkcolor=myblue, citecolor=mygreen, urlcolor=myred]{hyperref}

% 图片支持
\usepackage{graphicx}

% 其他工具
\usepackage{caption} % 支持 \captionof 命令
\usepackage{enumitem} % 增强的列表环境

% ============================================
% 颜色定义(兼容黑白打印)
% ============================================
\definecolor{myblue}{RGB}{0, 112, 192}
\definecolor{mygreen}{RGB}{0, 176, 80}
\definecolor{myorange}{RGB}{247, 150, 70}
\definecolor{myred}{RGB}{200, 0, 0}
\definecolor{mypurple}{RGB}{152, 78, 163}
\definecolor{mygray}{gray}{0.95}

% ============================================
% 定理、定义、例题环境设置
% ============================================

% 定理环境
\newtcolorbox{theorembox}[1]{
    colback=myblue!5!white,
    colframe=myblue!75!black,
    fonttitle=\bfseries,
    title=定理:#1,
    enhanced,
    attach boxed title to top left={xshift=1cm, yshift=-2mm},
    top=8mm,
    boxed title style={
        colback=myblue!75!black,
        sharp corners
    }
}

% 定义环境
\newtcolorbox{definitionbox}[1]{
    colback=mygreen!5!white,
    colframe=mygreen!75!black,
    fonttitle=\bfseries,
    title=定义:#1,
    enhanced,
    attach boxed title to top left={xshift=1cm, yshift=-2mm},
    top=8mm,
    boxed title style={
        colback=mygreen!75!black,
        sharp corners
    }
}

% 例题/技巧与应用环境
\newtcolorbox{examplebox}[1]{
    colback=myorange!5!white,
    colframe=myorange!75!black,
    fonttitle=\bfseries,
    title=技巧与应用:#1,
    enhanced,
    attach boxed title to top left={xshift=1cm, yshift=-2mm},
    top=8mm,
    boxed title style={
        colback=myorange!75!black,
        sharp corners
    }
}

% 引理环境
\newtcolorbox{lemmabox}[1]{
    colback=mypurple!5!white,
    colframe=mypurple!75!black,
    fonttitle=\bfseries,
    title=引理:#1,
    enhanced,
    attach boxed title to top left={xshift=1cm, yshift=-2mm},
    top=8mm,
    boxed title style={
        colback=mypurple!75!black,
        sharp corners
    }
}

% 注意/提醒环境
\newtcolorbox{notebox}[1]{
    colback=mygray,
    colframe=myred!75!black,
    fonttitle=\bfseries,
    title=注意:#1,
    enhanced,
    attach boxed title to top left={xshift=1cm, yshift=-2mm},
    top=8mm,
    boxed title style={
        colback=myred!75!black,
        sharp corners
    }
}

% ============================================
% 标题格式设置
% ============================================
\usepackage{titlesec}
\titleformat{\section}
    {\Large\bfseries\color{myblue}}
    {\thesection}{1em}{}
\titleformat{\subsection}
    {\large\bfseries\color{myorange}}
    {\thesubsection}{1em}{}

% ============================================
% 数学定理环境(使用 amsthm)
% ============================================
\theoremstyle{definition}
\newtheorem{theorem}{定理}[section]
\newtheorem{definition}{定义}[section]
\newtheorem{lemma}{引理}[section]
\newtheorem{corollary}{推论}[theorem]
\newtheorem{example}{例题}[section]

% ============================================
% 文档开始
% ============================================
\begin{document}

% 标题页
\title{\Huge \bfseries \color{myblue} 固定速度抛射的包络线}
\author{数学笔记}
\date{\today}
\maketitle

% 目录
\tableofcontents
\newpage

% ============================================
% 1. 引言与问题背景
% ============================================
\section{引言与问题背景}

\begin{definitionbox}{包络线}
\textbf{包络线(Envelope)}是指与一族曲线都相切的曲线。在几何上,包络线是这族曲线的“边界”,它恰好与族中的每一条曲线在某一点相切。

对于抛体运动,如果我们从同一点以相同的初速度 $v_0$ 但不同的抛射角 $\theta$ 抛出多个质点,这些质点的轨迹形成一族抛物线。这族抛物线的包络线就是一条特殊的曲线,它表示在给定初速度下,质点能够到达的所有位置点的边界。
\end{definitionbox}

\begin{center}
\begin{tikzpicture}[scale=1.2]
    % 坐标轴
    \draw[->, thick] (-0.5,0) -- (5,0) node[right] {$x$};
    \draw[->, thick] (0,-0.5) -- (0,3.5) node[above] {$y$};
    \node at (0,0) [below left] {$O$};
    
    % 抛射点
    \fill[myred] (0,0) circle (3pt) node[below left] {抛射点};
    
    % 多条不同角度的轨迹(示意)
    \foreach \angle in {15, 30, 45, 60, 75}
    {
        \pgfmathsetmacro{\vzero}{3}
        \pgfmathsetmacro{\g}{9.8}
        \pgfmathsetmacro{\tantheta}{tan(\angle)}
        \pgfmathsetmacro{\costheta}{cos(\angle)}
        \pgfmathsetmacro{\sintheta}{sin(\angle)}
        \pgfmathsetmacro{\xmax}{2*\vzero*\vzero*\sintheta*\costheta/\g}
        \pgfmathsetmacro{\ymax}{\vzero*\vzero*\sintheta*\sintheta/(2*\g)}
        
        \draw[myblue!40, thin, domain=0:\xmax, samples=50] 
            plot (\x, {\x*\tantheta - \g*\x*\x/(2*\vzero*\vzero*\costheta*\costheta)});
    }
    
    % 包络线(抛物线)
    \pgfmathsetmacro{\vzero}{3}
    \pgfmathsetmacro{\g}{9.8}
    \pgfmathsetmacro{\hmax}{\vzero*\vzero/(2*\g)}
    \draw[myred, very thick, domain=0:4.5, samples=100] 
        plot (\x, {\hmax - \g*\x*\x/(2*\vzero*\vzero)});
    \node[myred, above] at (2.5, 1.2) {包络线};
    
    % 标注
    \node[myblue, right] at (4.5, 0.3) {轨迹族};
\end{tikzpicture}
\captionof{figure}{包络线的直观理解:多条轨迹的边界}
\end{center}

\subsection{问题的物理意义}

在抛体运动中,如果我们固定初速度 $v_0$,只改变抛射角 $\theta$,那么:
\begin{itemize}
    \item 不同的抛射角对应不同的轨迹抛物线
    \item 这些轨迹形成一个“轨迹族”
    \item 包络线表示在给定初速度下,质点能够到达的所有位置点的\textbf{边界}
    \item 包络线内部的区域是“可达区域”,外部的区域是“不可达区域”
\end{itemize}

这个问题在军事、体育、工程等领域都有重要应用,例如:
\begin{itemize}
    \item 确定炮弹的射程范围
    \item 分析投掷物体的可达区域
    \item 设计安全防护区域
\end{itemize}

% ============================================
% 2. 抛体运动基础回顾
% ============================================
\section{抛体运动基础回顾}

\begin{definitionbox}{抛体运动的参数方程}
设质点从原点 $O(0,0)$ 以初速度 $v_0$、抛射角 $\theta$(与水平方向夹角)抛出,重力加速度为 $g$(方向竖直向下)。

\textbf{运动分解:}
\begin{itemize}
    \item 水平方向:匀速直线运动,初速度 $v_{0x} = v_0\cos\theta$
    \item 竖直方向:匀变速直线运动,初速度 $v_{0y} = v_0\sin\theta$,加速度 $-g$
\end{itemize}

\textbf{参数方程(以时间 $t$ 为参数):}
\begin{align}
    x(t) &= v_0\cos\theta \cdot t \label{eq:x}\\
    y(t) &= v_0\sin\theta \cdot t - \frac{1}{2}gt^2 \label{eq:y}
\end{align}

其中 $t \geq 0$,且 $y(t) \geq 0$(落地前)。
\end{definitionbox}

\begin{center}
\begin{tikzpicture}[scale=1.5]
    % 坐标轴
    \draw[->, thick] (-0.3,0) -- (4,0) node[right] {$x$};
    \draw[->, thick] (0,-0.3) -- (0,2.5) node[above] {$y$};
    \node at (0,0) [below left] {$O$};
    
    % 抛射点和速度矢量
    \fill[myred] (0,0) circle (2pt);
    \draw[->, myred, very thick] (0,0) -- (1.5, 1.3) node[midway, above left] {$v_0$};
    \draw[->, myblue, thick] (0,0) -- (1.5, 0) node[midway, below] {$v_{0x}$};
    \draw[->, mygreen, thick] (0,0) -- (0, 1.3) node[midway, left] {$v_{0y}$};
    
    % 角度标注
    \draw (0.4, 0) arc (0:40:0.4);
    \node at (0.6, 0.15) {$\theta$};
    
    % 轨迹
    \pgfmathsetmacro{\vzero}{2}
    \pgfmathsetmacro{\g}{9.8}
    \pgfmathsetmacro{\angle}{40}
    \pgfmathsetmacro{\tantheta}{tan(\angle)}
    \pgfmathsetmacro{\costheta}{cos(\angle)}
    \pgfmathsetmacro{\sintheta}{sin(\angle)}
    \pgfmathsetmacro{\xmax}{2*\vzero*\vzero*\sintheta*\costheta/\g}
    
    \draw[myorange, very thick, domain=0:\xmax, samples=100] 
        plot (\x, {\x*\tantheta - \g*\x*\x/(2*\vzero*\vzero*\costheta*\costheta)});
    
    % 关键点标注
    \pgfmathsetmacro{\ymax}{\vzero*\vzero*\sintheta*\sintheta/(2*\g)}
    \pgfmathsetmacro{\xmax}{2*\vzero*\vzero*\sintheta*\costheta/\g}
    \fill[myred] (0.5*\xmax, \ymax) circle (2pt) node[above] {最高点};
    \fill[myred] (\xmax, 0) circle (2pt) node[below right] {落地点};
    
    % 标注
    \node[myorange, right] at (3, 1) {轨迹};
\end{tikzpicture}
\captionof{figure}{抛体运动的速度分解与轨迹}
\end{center}

\subsection{轨迹方程}

从参数方程 \eqref{eq:x} 和 \eqref{eq:y} 中消去参数 $t$,得到轨迹方程:

由 $x = v_0\cos\theta \cdot t$ 得 $t = \frac{x}{v_0\cos\theta}$,代入 $y$ 的表达式:

\begin{align}
    y &= v_0\sin\theta \cdot \frac{x}{v_0\cos\theta} - \frac{1}{2}g\left(\frac{x}{v_0\cos\theta}\right)^2 \nonumber\\
    &= x\tan\theta - \frac{gx^2}{2v_0^2\cos^2\theta} \nonumber\\
    &= x\tan\theta - \frac{gx^2}{2v_0^2}(1 + \tan^2\theta) \label{eq:trajectory}
\end{align}

这是一个关于 $x$ 的二次函数,轨迹为抛物线。

\textbf{重要结论:}
\begin{itemize}
    \item 射程(水平距离):$R = \frac{v_0^2\sin(2\theta)}{g}$,当 $\theta = 45°$ 时取得最大值 $R_{\max} = \frac{v_0^2}{g}$
    \item 最大高度:$H = \frac{v_0^2\sin^2\theta}{2g}$,当 $\theta = 90°$ 时取得最大值 $H_{\max} = \frac{v_0^2}{2g}$
\end{itemize}

% ============================================
% 3. 固定速度抛射的轨迹族
% ============================================
\section{固定速度抛射的轨迹族}

当我们固定初速度 $v_0$,让抛射角 $\theta$ 变化时,得到一族轨迹曲线。

\begin{definitionbox}{轨迹族}
对于固定的初速度 $v_0$,抛射角 $\theta$ 作为参数,轨迹方程 \eqref{eq:trajectory} 可以写成:
\begin{equation}
    y = x\tan\theta - \frac{gx^2}{2v_0^2}(1 + \tan^2\theta), \quad \theta \in \left(0, \frac{\pi}{2}\right) \label{eq:family}
\end{equation}

这表示一个以 $\theta$(或 $\tan\theta$)为参数的曲线族。每条曲线对应一个特定的抛射角。
\end{definitionbox}

\begin{center}
\begin{tikzpicture}[scale=1.0]
    % 坐标轴
    \draw[->, thick] (-0.5,0) -- (5.5,0) node[right] {$x$};
    \draw[->, thick] (0,-0.5) -- (0,3.5) node[above] {$y$};
    \node at (0,0) [below left] {$O$};
    
    % 抛射点
    \fill[myred] (0,0) circle (3pt);
    
    % 多条不同角度的轨迹
    \pgfmathsetmacro{\vzero}{3}
    \pgfmathsetmacro{\g}{9.8}
    
    \foreach \angle in {15, 25, 35, 45, 55, 65, 75}
    {
        \pgfmathsetmacro{\tantheta}{tan(\angle)}
        \pgfmathsetmacro{\costheta}{cos(\angle)}
        \pgfmathsetmacro{\sintheta}{sin(\angle)}
        \pgfmathsetmacro{\xmax}{2*\vzero*\vzero*\sintheta*\costheta/\g}
        
        \draw[myblue!50, thin, domain=0:\xmax, samples=50] 
            plot (\x, {\x*\tantheta - \g*\x*\x*(1+\tantheta*\tantheta)/(2*\vzero*\vzero)});
    }
    
    % 特殊角度标注
    \pgfmathsetmacro{\angle}{45}
    \pgfmathsetmacro{\tantheta}{tan(\angle)}
    \pgfmathsetmacro{\sintheta}{sin(\angle)}
    \pgfmathsetmacro{\costheta}{cos(\angle)}
    \pgfmathsetmacro{\xmax}{2*\vzero*\vzero*\sintheta*\costheta/\g}
    \draw[mygreen, thick, domain=0:\xmax, samples=50] 
        plot (\x, {\x*\tantheta - \g*\x*\x*(1+\tantheta*\tantheta)/(2*\vzero*\vzero)});
    \node[mygreen, above] at (0.5*\xmax, 0.5) {$\theta=45°$};
    
    % 标注
    \node[myblue, right] at (5, 0.5) {轨迹族};
\end{tikzpicture}
\captionof{figure}{固定初速度 $v_0$ 下不同抛射角的轨迹族}
\end{center}

\subsection{轨迹族的特点}

观察轨迹族,我们可以发现:
\begin{itemize}
    \item 所有轨迹都从同一点 $(0,0)$ 出发
    \item 不同角度的轨迹有不同的射程和最大高度
    \item 这些轨迹之间存在一条\textbf{包络线},它是所有轨迹的边界
    \item 包络线内部的区域是可达区域,外部是不可达区域
\end{itemize}

% ============================================
% 4. 包络线的数学推导
% ============================================
\section{包络线的数学推导}

\begin{definitionbox}{包络线的数学定义}
设曲线族由方程 $F(x, y, \alpha) = 0$ 给出,其中 $\alpha$ 是参数。如果存在一条曲线 $C$,使得:
\begin{enumerate}
    \item $C$ 上的每一点都在族中某条曲线上
    \item $C$ 在每一点都与族中经过该点的曲线相切
\end{enumerate}
则称 $C$ 为该曲线族的\textbf{包络线}。

包络线的求法:联立方程组
\begin{equation}
    \begin{cases}
        F(x, y, \alpha) = 0 \\
        \frac{\partial F}{\partial \alpha}(x, y, \alpha) = 0
    \end{cases}
\end{equation}
消去参数 $\alpha$,即可得到包络线的方程。
\end{definitionbox}

\subsection{包络线的推导过程}

对于抛体运动的轨迹族,我们使用 $\tan\theta$ 作为参数会更方便。设 $k = \tan\theta$,则轨迹方程 \eqref{eq:family} 变为:

\begin{equation}
    F(x, y, k) = y - xk + \frac{gx^2}{2v_0^2}(1 + k^2) = 0 \label{eq:F}
\end{equation}

对参数 $k$ 求偏导数:

\begin{equation}
    \frac{\partial F}{\partial k} = -x + \frac{gx^2}{2v_0^2} \cdot 2k = -x + \frac{gx^2k}{v_0^2} = 0 \label{eq:partial}
\end{equation}

由 \eqref{eq:partial} 得:
\begin{equation}
    -x + \frac{gx^2k}{v_0^2} = 0 \quad \Rightarrow \quad x\left(-1 + \frac{gxk}{v_0^2}\right) = 0
\end{equation}

当 $x = 0$ 时,对应抛射点,不是包络线上的点(除起点外)。因此:
\begin{equation}
    -1 + \frac{gxk}{v_0^2} = 0 \quad \Rightarrow \quad k = \frac{v_0^2}{gx} \label{eq:k}
\end{equation}

将 \eqref{eq:k} 代入 \eqref{eq:F}:

\begin{align}
    y &= x \cdot \frac{v_0^2}{gx} - \frac{gx^2}{2v_0^2}\left(1 + \left(\frac{v_0^2}{gx}\right)^2\right) \nonumber\\
    &= \frac{v_0^2}{g} - \frac{gx^2}{2v_0^2}\left(1 + \frac{v_0^4}{g^2x^2}\right) \nonumber\\
    &= \frac{v_0^2}{g} - \frac{gx^2}{2v_0^2} - \frac{gx^2}{2v_0^2} \cdot \frac{v_0^4}{g^2x^2} \nonumber\\
    &= \frac{v_0^2}{g} - \frac{gx^2}{2v_0^2} - \frac{v_0^2}{2g} \nonumber\\
    &= \frac{v_0^2}{2g} - \frac{gx^2}{2v_0^2}
\end{align}

\begin{theorembox}{抛体运动轨迹族的包络线方程}
从同一点以固定初速度 $v_0$、不同抛射角 $\theta$ 抛出的质点,其轨迹族的包络线方程为:
\begin{equation}
    y = \frac{v_0^2}{2g} - \frac{gx^2}{2v_0^2} \label{eq:envelope}
\end{equation}

这是一个开口向下的抛物线,其性质为:
\begin{itemize}
    \item 顶点:$\left(0, \frac{v_0^2}{2g}\right)$,这是竖直上抛能达到的最大高度
    \item 对称轴:$x = 0$($y$ 轴)
    \item 与 $x$ 轴的交点:$x = \pm \frac{v_0^2}{g}$,这是水平抛射的最大射程
\end{itemize}
\end{theorembox}

\subsection{推导方法的说明}

我们也可以直接从轨迹方程出发,使用另一种方法:

将轨迹方程 \eqref{eq:trajectory} 整理为关于 $\tan\theta$ 的二次方程:
\begin{equation}
    \frac{gx^2}{2v_0^2}\tan^2\theta - x\tan\theta + \left(y + \frac{gx^2}{2v_0^2}\right) = 0
\end{equation}

对于给定的点 $(x, y)$,如果它位于某条轨迹上,则上述关于 $\tan\theta$ 的方程有实数解,判别式 $\Delta \geq 0$:

\begin{align}
    \Delta &= x^2 - 4 \cdot \frac{gx^2}{2v_0^2} \cdot \left(y + \frac{gx^2}{2v_0^2}\right) \nonumber\\
    &= x^2 - \frac{2gx^2}{v_0^2}\left(y + \frac{gx^2}{2v_0^2}\right) \nonumber\\
    &= x^2 - \frac{2gx^2y}{v_0^2} - \frac{g^2x^4}{v_0^4} \geq 0
\end{align}

整理得:
\begin{equation}
    y \leq \frac{v_0^2}{2g} - \frac{gx^2}{2v_0^2}
\end{equation}

等号成立时,点 $(x, y)$ 恰好在包络线上,这与 \eqref{eq:envelope} 一致。

% ============================================
% 5. 包络线的性质与应用
% ============================================
\section{包络线的性质与应用}

\subsection{包络线的几何性质}

\begin{center}
\begin{tikzpicture}[scale=1.0]
    % 坐标轴
    \draw[->, thick] (-4,0) -- (4,0) node[right] {$x$};
    \draw[->, thick] (0,-0.5) -- (0,3.5) node[above] {$y$};
    \node at (0,0) [below left] {$O$};
    
    % 抛射点
    \fill[myred] (0,0) circle (3pt) node[below left] {抛射点};
    
    % 多条轨迹(较淡)
    \pgfmathsetmacro{\vzero}{3}
    \pgfmathsetmacro{\g}{9.8}
    
    \foreach \angle in {10, 20, 30, 40, 50, 60, 70, 80}
    {
        \pgfmathsetmacro{\tantheta}{tan(\angle)}
        \pgfmathsetmacro{\costheta}{cos(\angle)}
        \pgfmathsetmacro{\sintheta}{sin(\angle)}
        \pgfmathsetmacro{\xmax}{2*\vzero*\vzero*\sintheta*\costheta/\g}
        
        \draw[myblue!30, very thin, domain=0:\xmax, samples=50] 
            plot (\x, {\x*\tantheta - \g*\x*\x*(1+\tantheta*\tantheta)/(2*\vzero*\vzero)});
    }
    
    % 包络线(粗线)
    \pgfmathsetmacro{\hmax}{\vzero*\vzero/(2*\g)}
    \pgfmathsetmacro{\xmax}{\vzero*\vzero/\g}
    \draw[myred, very thick, domain=-\xmax:\xmax, samples=100] 
        plot (\x, {\hmax - \g*\x*\x/(2*\vzero*\vzero)});
    
    % 标注顶点
    \fill[myred] (0, \hmax) circle (3pt);
    \draw[dashed, mygray] (0, 0) -- (0, \hmax) node[midway, left] {$\frac{v_0^2}{2g}$};
    \node[myred, above] at (0, \hmax) {顶点};
    
    % 标注与x轴交点
    \fill[myred] (\xmax, 0) circle (3pt);
    \fill[myred] (-\xmax, 0) circle (3pt);
    \draw[dashed, mygray] (\xmax, 0) -- (\xmax, 0.3);
    \draw[dashed, mygray] (-\xmax, 0) -- (-\xmax, 0.3);
    \node[below] at (\xmax, 0) {$\frac{v_0^2}{g}$};
    \node[below] at (-\xmax, 0) {$-\frac{v_0^2}{g}$};
    
    % 可达区域(阴影)
    \fill[mygreen!20, domain=-\xmax:\xmax, samples=100] 
        plot (\x, {\hmax - \g*\x*\x/(2*\vzero*\vzero)}) -- (-\xmax, 0) -- (\xmax, 0) -- cycle;
    
    \node[mygreen, right] at (2, 1.5) {可达区域};
    \node[myred, above] at (0, 2.5) {包络线};
\end{tikzpicture}
\captionof{figure}{包络线的几何性质与可达区域}
\end{center}

\begin{theorembox}{包络线的几何性质}
包络线 $y = \frac{v_0^2}{2g} - \frac{gx^2}{2v_0^2}$ 具有以下性质:

\begin{enumerate}
    \item \textbf{顶点}:$\left(0, \frac{v_0^2}{2g}\right)$
    \begin{itemize}
        \item 这是竖直上抛($\theta = 90°$)能达到的最大高度
        \item 也是所有轨迹中能达到的最高点
    \end{itemize}
    
    \item \textbf{对称轴}:$x = 0$($y$ 轴)
    \begin{itemize}
        \item 包络线关于 $y$ 轴对称
        \item 这反映了抛体运动的对称性
    \end{itemize}
    
    \item \textbf{与坐标轴的交点}:
    \begin{itemize}
        \item 与 $y$ 轴交点:$(0, \frac{v_0^2}{2g})$
        \item 与 $x$ 轴交点:$(\pm \frac{v_0^2}{g}, 0)$
        \item 水平距离 $\frac{v_0^2}{g}$ 是水平抛射($\theta = 0°$)的最大射程
    \end{itemize}
    
    \item \textbf{开口方向}:向下
    \begin{itemize}
        \item 包络线是开口向下的抛物线
        \item 包络线内部的区域是可达区域
    \end{itemize}
\end{enumerate}
\end{theorembox}

\subsection{物理意义}

包络线的物理意义非常重要:

\begin{itemize}
    \item \textbf{可达区域的边界}:包络线内部的区域表示在给定初速度 $v_0$ 下,质点能够到达的所有位置点
    \item \textbf{不可达区域}:包络线外部的区域表示无论以什么角度抛射,都无法到达
    \item \textbf{最优抛射角}:包络线上的每个点都对应一个特定的抛射角,这个角度是到达该点的唯一角度(或两个对称角度)
\end{itemize}

\begin{center}
\begin{tikzpicture}[scale=1.2]
    % 坐标轴
    \draw[->, thick] (-0.3,0) -- (3.5,0) node[right] {$x$};
    \draw[->, thick] (0,-0.3) -- (0,2.5) node[above] {$y$};
    \node at (0,0) [below left] {$O$};
    
    % 速度矢量分解示意图
    \fill[myred] (0,0) circle (2pt) node[below left] {抛射点};
    
    % 多个不同角度的速度矢量
    \foreach \angle in {15, 30, 45, 60, 75}
    {
        \pgfmathsetmacro{\vlen}{1.5}
        \pgfmathsetmacro{\vx}{\vlen*cos(\angle)}
        \pgfmathsetmacro{\vy}{\vlen*sin(\angle)}
        
        \draw[->, myblue!50, thick] (0,0) -- (\vx, \vy);
        \draw[->, mygreen!50, thin, dashed] (0,0) -- (\vx, 0);
        \draw[->, myorange!50, thin, dashed] (\vx, 0) -- (\vx, \vy);
    }
    
    % 标注
    \draw[->, myred, very thick] (0,0) -- (1.2, 1.2) node[midway, above left] {$v_0$};
    \draw[->, mygreen, thick] (0,0) -- (1.2, 0) node[midway, below] {$v_0\cos\theta$};
    \draw[->, myorange, thick] (1.2, 0) -- (1.2, 1.2) node[midway, right] {$v_0\sin\theta$};
    
    \node[myblue, right] at (2.5, 1.5) {不同角度的初速度};
    \node[mygreen, right] at (2.5, 1.2) {水平分量};
    \node[myorange, right] at (2.5, 0.9) {竖直分量};
\end{tikzpicture}
\captionof{figure}{不同抛射角的速度分解}
\end{center}

% ============================================
% 6. 例题与应用
% ============================================
\section{例题与应用}

\begin{examplebox}{例题1:求包络线方程}
\textbf{题目:}从原点以初速度 $v_0 = 20\text{ m/s}$ 抛出质点,重力加速度 $g = 10\text{ m/s}^2$。求所有可能轨迹的包络线方程,并确定可达区域。

\textbf{解:}

\textbf{步骤1:}写出轨迹族方程

设抛射角为 $\theta$,轨迹方程为:
\[ y = x\tan\theta - \frac{gx^2}{2v_0^2}(1 + \tan^2\theta) \]

代入 $v_0 = 20$,$g = 10$:
\[ y = x\tan\theta - \frac{10x^2}{2 \times 20^2}(1 + \tan^2\theta) = x\tan\theta - \frac{x^2}{80}(1 + \tan^2\theta) \]

\textbf{步骤2:}使用包络线求法

设 $k = \tan\theta$,则:
\[ F(x, y, k) = y - xk + \frac{x^2}{80}(1 + k^2) = 0 \]

对 $k$ 求偏导:
\[ \frac{\partial F}{\partial k} = -x + \frac{x^2k}{40} = 0 \]

当 $x \neq 0$ 时:
\[ k = \frac{40}{x} \]

\textbf{步骤3:}代入求包络线方程

\[ y = x \cdot \frac{40}{x} - \frac{x^2}{80}\left(1 + \left(\frac{40}{x}\right)^2\right) = 40 - \frac{x^2}{80} - \frac{1600}{80} = 40 - \frac{x^2}{80} - 20 = 20 - \frac{x^2}{80} \]

\textbf{答案:}包络线方程为 $y = 20 - \frac{x^2}{80}$。

\textbf{步骤4:}分析可达区域

\begin{itemize}
    \item 顶点:$(0, 20)$,最大高度为 $20$ 米
    \item 与 $x$ 轴交点:$x = \pm 40$,最大水平距离为 $40$ 米
    \item 可达区域:包络线下方(包括边界)的区域
\end{itemize}
\end{examplebox}

\begin{examplebox}{例题2:判断点是否可达}
\textbf{题目:}在例题1的条件下,判断点 $P(30, 10)$ 是否可达?如果可达,求到达该点的抛射角。

\textbf{解:}

\textbf{方法1:}判断点是否在包络线下方

包络线方程为 $y = 20 - \frac{x^2}{80}$。

当 $x = 30$ 时,包络线上的 $y$ 值为:
\[ y = 20 - \frac{30^2}{80} = 20 - \frac{900}{80} = 20 - 11.25 = 8.75 \]

点 $P(30, 10)$ 的 $y$ 坐标为 $10 > 8.75$,说明该点在包络线上方,\textbf{不可达}。

\textbf{方法2:}使用判别式法

将点 $P(30, 10)$ 代入轨迹方程:
\[ 10 = 30\tan\theta - \frac{30^2}{80}(1 + \tan^2\theta) \]

整理得:
\[ 10 = 30k - \frac{900}{80}(1 + k^2) = 30k - \frac{45}{4}(1 + k^2) \]

\[ 40 = 120k - 45(1 + k^2) = 120k - 45 - 45k^2 \]

\[ 45k^2 - 120k + 85 = 0 \]

判别式:
\[ \Delta = 120^2 - 4 \times 45 \times 85 = 14400 - 15300 = -900 < 0 \]

判别式小于零,说明不存在实数解,因此点 $P(30, 10)$ \textbf{不可达}。
\end{examplebox}

\begin{examplebox}{例题3:求到达包络线上某点的抛射角}
\textbf{题目:}在例题1的条件下,求到达包络线上点 $Q(20, 15)$ 的抛射角。

\textbf{解:}

\textbf{步骤1:}验证点在包络线上

当 $x = 20$ 时,包络线上的 $y$ 值为:
\[ y = 20 - \frac{20^2}{80} = 20 - 5 = 15 \]

点 $Q(20, 15)$ 确实在包络线上。

\textbf{步骤2:}求对应的 $k = \tan\theta$

由包络线的推导过程,$k = \frac{v_0^2}{gx} = \frac{40}{20} = 2$

\textbf{步骤3:}求抛射角

\[ \theta = \arctan(2) \approx 63.4° \]

\textbf{答案:}到达点 $Q(20, 15)$ 的抛射角约为 $63.4°$。

\textbf{验证:}将 $\theta = \arctan(2)$ 代入轨迹方程验证即可。
\end{examplebox}

\begin{notebox}{重要注意事项}
\begin{enumerate}
    \item \textbf{包络线的适用范围}:包络线方程是在理想条件下推导的(忽略空气阻力、假设重力加速度恒定等)。实际应用中需要考虑这些因素。
    
    \item \textbf{可达区域的判断}:判断点 $(x, y)$ 是否可达,只需检查是否满足 $y \leq \frac{v_0^2}{2g} - \frac{gx^2}{2v_0^2}$。
    
    \item \textbf{包络线上的点}:包络线上的每个点(除顶点外)通常对应两个对称的抛射角,一个在 $0°$ 到 $45°$ 之间,一个在 $45°$ 到 $90°$ 之间。
    
    \item \textbf{最大射程}:虽然单个轨迹的最大射程在 $\theta = 45°$ 时取得,但包络线与 $x$ 轴的交点对应的是水平抛射($\theta = 0°$)的射程。
    
    \item \textbf{单位一致性}:计算时注意物理量的单位要一致,通常使用国际单位制(SI)。
\end{enumerate}
\end{notebox}

% ============================================
% 总结
% ============================================
\section{总结}

本笔记介绍了固定速度抛射的包络线问题,主要内容包括:

\begin{itemize}
    \item \textbf{包络线的概念}:与一族曲线都相切的曲线,表示可达区域的边界
    \item \textbf{抛体运动基础}:参数方程、轨迹方程、射程和最大高度
    \item \textbf{轨迹族}:固定初速度、不同抛射角形成的抛物线族
    \item \textbf{包络线方程}:$y = \frac{v_0^2}{2g} - \frac{gx^2}{2v_0^2}$
    \item \textbf{几何性质}:顶点、对称轴、与坐标轴的交点
    \item \textbf{应用}:判断点是否可达、求最优抛射角等
\end{itemize}

包络线问题将数学中的曲线族理论与物理中的抛体运动相结合,是数学物理交叉应用的典型例子。掌握包络线的求法和性质,有助于深入理解抛体运动的本质特征。

% ============================================
% 文档结束
% ============================================
\end{document}


