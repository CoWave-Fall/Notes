% !TEX program = xelatex
% !TEX encoding = UTF-8
% ============================================
% 固定速度抛射的包络线
% 数学笔记 XeLaTeX 模板
% ============================================

\documentclass[a4paper,11pt]{ctexart}

\ProvidesFile{mathnote-preamble.tex}[2024/11/13 Math note modern preamble]
\RequirePackage{iftex}
\ifPDFTeX
  \PackageError{mathnote-preamble}{请使用 \XeLaTeX\ 或 LuaLaTeX 编译该模板}{在 MikTeX / TeX Live 中切换到 \XeLaTeX\ (推荐) 或 LuaLaTeX 后重新编译。}
\fi
\makeatletter
\newif\ifmathnote@fontdirfound
\mathnote@fontdirfoundfalse
\def\mathnote@fontdircandidates{fonts/,./fonts/,../fonts/,../../fonts/,../../../fonts/}
\def\mathnote@fontdir{}
\@for\mathnote@cand:=\mathnote@fontdircandidates\do{%
  \ifmathnote@fontdirfound\else
    \IfFileExists{\mathnote@cand NotoSerif-VF.ttf}{%
      \edef\mathnote@fontdir{\mathnote@cand}%
      \mathnote@fontdirfoundtrue
    }{%
      \IfFileExists{\mathnote@cand SourceHanSerifSC-Regular.otf}{%
        \edef\mathnote@fontdir{\mathnote@cand}%
        \mathnote@fontdirfoundtrue
      }{}%
    }%
  \fi
}
\ifmathnote@fontdirfound\else
  \edef\mathnote@fontdir{fonts/}%
\fi
\@ifundefined{MathNoteFontDir}{%
  \edef\MathNoteFontDir{\mathnote@fontdir}%
}{}

% --------------------------------------------------
% User switches
% --------------------------------------------------
\newif\ifmathnoteprintmode
\mathnoteprintmodefalse
\newcommand{\MathNoteEnablePrint}{%
  \mathnoteprintmodetrue
  \mathnote@applypalette
}
\newif\ifmathnotereviewstamp
\mathnotereviewstampfalse
\newcommand{\mathnote@reviewstamppathbase}{assest/lzlxV-reviewed}
\newcommand{\mathnote@reviewstampsvg}{\mathnote@reviewstamppathbase.svg}
\newcommand{\mathnote@reviewstamppdf}{\mathnote@reviewstamppathbase.pdf}
\newcommand{\mathnote@reviewstampinclude}{}
\newif\ifmathnote@reviewstampplaced
\mathnote@reviewstampplacedfalse
\IfFileExists{\mathnote@reviewstamppdf}{%
  \renewcommand{\mathnote@reviewstampinclude}{%
    \includegraphics[width=20mm,height=20mm,keepaspectratio]{\mathnote@reviewstamppdf}%
  }%
}{%
  \IfFileExists{\mathnote@reviewstampsvg}{%
    \renewcommand{\mathnote@reviewstampinclude}{%
      \includesvg[width=20mm,height=20mm,keepaspectratio]{\mathnote@reviewstamppathbase}%
    }%
  }{}%
}
\newcommand{\mathnote@enablereviewstamp}{%
  \ifx\mathnote@reviewstampinclude\@empty
    \PackageWarning{mathnote-preamble}{Review stamp graphic \mathnote@reviewstampsvg\space (or PDF fallback) not found}%
  \else
    \mathnote@reviewstampplacedfalse
    \AddToHook{shipout/foreground}{%
      \ifmathnote@reviewstampplaced\else
        \begin{tikzpicture}[remember picture, overlay]
          \node[anchor=north east, xshift=-8mm, yshift=-8mm] at (current page.north east){%
            \mathnote@reviewstampinclude
          };
        \end{tikzpicture}%
        \global\mathnote@reviewstampplacedtrue
      \fi
    }%
  \fi
}
\newcommand{\MathNoteEnableReviewStamp}{%
  \mathnotereviewstamptrue
  \mathnote@enablereviewstamp
}

% --------------------------------------------------
% Metadata defaults (can be overwritten in each file)
% --------------------------------------------------
\providecommand{\notetitle}{数学学习笔记}
\providecommand{\noteauthor}{作者}
\providecommand{\notedate}{\today}
\providecommand{\notesubtitle}{现代数学排版示例}
\providecommand{\noteversion}{v1.0}

% --------------------------------------------------
% Core packages
% --------------------------------------------------
\usepackage{geometry}
\geometry{
  paper=a4paper,
  top=2.35cm,
  bottom=2.4cm,
  left=2.1cm,
  right=2.1cm,
  headheight=16pt,
  headsep=14pt
}

\usepackage{fontspec}
\usepackage{metalogo}
\defaultfontfeatures{Ligatures=TeX, Scale=MatchLowercase}

\newcommand{\mathnote@fontfile}[1]{\MathNoteFontDir#1}
\newif\ifmathnote@haslocalserif
\newif\ifmathnote@haslocalsans
\newif\ifmathnote@haslocalmono
\newif\ifmathnote@haslocalcjkserif
\newif\ifmathnote@haslocalcjksans
\newif\ifmathnote@haslocalcjkmono
\newif\ifmathnote@haslocalkai
\newif\ifmathnote@haslocalshserif
\newif\ifmathnote@haslocalshsans
\IfFileExists{\mathnote@fontfile{NotoSerif-VF.ttf}}{\mathnote@haslocalseriftrue}{\mathnote@haslocalseriffalse}
\IfFileExists{\mathnote@fontfile{NotoSans-VF.ttf}}{\mathnote@haslocalsanstrue}{\mathnote@haslocalsansfalse}
\IfFileExists{\mathnote@fontfile{NotoSansMono-VF.ttf}}{\mathnote@haslocalmonotrue}{\mathnote@haslocalmonofalse}
\IfFileExists{\mathnote@fontfile{NotoSerifCJK-VF.ttc}}{\mathnote@haslocalcjkseriftrue}{\mathnote@haslocalcjkseriffalse}
\IfFileExists{\mathnote@fontfile{NotoSansCJK-VF.ttc}}{\mathnote@haslocalcjksanstrue}{\mathnote@haslocalcjksansfalse}
\IfFileExists{\mathnote@fontfile{NotoSansMonoCJK-VF.ttc}}{\mathnote@haslocalcjkmonotrue}{\mathnote@haslocalcjkmonofalse}
\IfFileExists{\mathnote@fontfile{LXGWWenKaiSC-Regular.ttf}}{\mathnote@haslocalkaitrue}{\mathnote@haslocalkaifalse}
\IfFileExists{\mathnote@fontfile{SourceHanSerifSC-Regular.otf}}{\mathnote@haslocalshseriftrue}{\mathnote@haslocalshseriffalse}
\IfFileExists{\mathnote@fontfile{SourceHanSansSC-Regular.otf}}{\mathnote@haslocalshsanstrue}{\mathnote@haslocalshsansfalse}

\newcommand{\mathnote@cjkitalicfont}{}
\newcommand{\mathnote@cjkitalicfeatures}{Language = Chinese Simplified}
\ifmathnote@haslocalkai
  \def\mathnote@cjkitalicfont{LXGWWenKaiSC-Regular}
  \def\mathnote@cjkitalicfeatures{Path = {\MathNoteFontDir}, Extension = .ttf, Language = Chinese Simplified}
\else
  \IfFontExistsTF{LXGW WenKai SC}{%
    \def\mathnote@cjkitalicfont{LXGW WenKai SC}%
    \def\mathnote@cjkitalicfeatures{Language = Chinese Simplified}
  }{%
    \def\mathnote@cjkitalicfont{}%
    \def\mathnote@cjkitalicfeatures{Language = Chinese Simplified}
  }%
\fi
\ifx\mathnote@cjkitalicfont\@empty
  \def\mathnote@cjkitalicfont{FandolKai}
  \def\mathnote@cjkitalicfeatures{Language = Chinese Simplified}
\fi

\newcommand{\mathnote@setlatinfonts}{%
  \ifmathnote@haslocalserif
    \setmainfont{Noto Serif}[
      Path = {\MathNoteFontDir},
      Extension = .ttf,
      UprightFont = NotoSerif-VF,
      ItalicFont = NotoSerif-Italic-VF,
      BoldFont = NotoSerif-VF,
      BoldFeatures = {RawFeature={+wght=760}},
      BoldItalicFont = NotoSerif-Italic-VF,
      BoldItalicFeatures = {RawFeature={+wght=760}}
    ]%
  \else
    \IfFontExistsTF{Noto Serif}{%
      \setmainfont{Noto Serif}[
        ItalicFont = {Noto Serif Italic},
        BoldFont = {Noto Serif Bold},
        BoldItalicFont = {Noto Serif Bold Italic}
      ]%
    }{%
      \setmainfont{TeX Gyre Pagella}%
    }%
  \fi
  \ifmathnote@haslocalsans
    \setsansfont{Noto Sans}[
      Path = {\MathNoteFontDir},
      Extension = .ttf,
      UprightFont = NotoSans-VF,
      ItalicFont = NotoSans-Italic-VF,
      BoldFont = NotoSans-VF,
      BoldFeatures = {RawFeature={+wght=760}},
      BoldItalicFont = NotoSans-Italic-VF,
      BoldItalicFeatures = {RawFeature={+wght=760}}
    ]%
  \else
    \IfFontExistsTF{Noto Sans}{%
      \setsansfont{Noto Sans}[
        ItalicFont = {Noto Sans Italic},
        BoldFont = {Noto Sans Bold},
        BoldItalicFont = {Noto Sans Bold Italic}
      ]%
    }{%
      \setsansfont{TeX Gyre Heros}%
    }%
  \fi
  \ifmathnote@haslocalmono
    \setmonofont{Noto Sans Mono}[
      Path = {\MathNoteFontDir},
      Extension = .ttf,
      UprightFont = NotoSansMono-VF,
      BoldFont = NotoSansMono-VF,
      BoldFeatures = {RawFeature={+wght=740}},
      ItalicFont = NotoSansMono-VF,
      ItalicFeatures = {FakeSlant=0.2}
    ]%
  \else
    \IfFontExistsTF{Noto Sans Mono}{%
      \setmonofont{Noto Sans Mono}[
        BoldFont = {Noto Sans Mono Bold}
      ]%
    }{%
      \setmonofont{TeX Gyre Cursor}%
    }%
  \fi
}

\newcommand{\mathnote@setcjkfonts}{%
  \ifmathnote@haslocalshserif
    \setCJKmainfont{SourceHanSerifSC-Regular}[
      Path = {\MathNoteFontDir},
      Extension = .otf,
      Language = Chinese Simplified,
      BoldFont = SourceHanSerifSC-Bold,
      ItalicFont = {\mathnote@cjkitalicfont},
      ItalicFeatures = {\mathnote@cjkitalicfeatures}
    ]%
  \else
    \IfFontExistsTF{Source Han Serif SC}{%
      \setCJKmainfont{Source Han Serif SC}[Language=Chinese Simplified, ItalicFont={\mathnote@cjkitalicfont}, ItalicFeatures={\mathnote@cjkitalicfeatures}]%
    }{%
      \ifmathnote@haslocalcjkserif
        \setCJKmainfont{NotoSerifCJK-VF}[
          Path = {\MathNoteFontDir},
          Extension = .ttc,
          Language = Chinese Simplified,
          UprightFont = NotoSerifCJK-VF,
          UprightFeatures = {FontIndex=2},
          BoldFont = NotoSerifCJK-VF,
          BoldFeatures = {FontIndex=2,RawFeature={+wght=780}},
          AutoFakeSlant = 0.18,
          ItalicFont = {\mathnote@cjkitalicfont},
          ItalicFeatures = {\mathnote@cjkitalicfeatures}
        ]%
      \else
        \IfFontExistsTF{Noto Serif CJK SC}{%
          \setCJKmainfont{Noto Serif CJK SC}[Language=Chinese Simplified, ItalicFont={\mathnote@cjkitalicfont}, ItalicFeatures={\mathnote@cjkitalicfeatures}]
        }{%
          \setCJKmainfont{FandolSong}[BoldFont={FandolSong-Bold}, ItalicFont={\mathnote@cjkitalicfont}, ItalicFeatures={\mathnote@cjkitalicfeatures}]
        }%
      \fi
    }%
  \fi
  \ifmathnote@haslocalshsans
    \setCJKsansfont{SourceHanSansSC-Regular}[
      Path = {\MathNoteFontDir},
      Extension = .otf,
      Language = Chinese Simplified,
      BoldFont = SourceHanSansSC-Bold,
      ItalicFont = {\mathnote@cjkitalicfont},
      ItalicFeatures = {\mathnote@cjkitalicfeatures}
    ]%
    \setCJKfamilyfont{hei}{SourceHanSansSC-Regular}[
      Path = {\MathNoteFontDir},
      Extension = .otf,
      BoldFont = SourceHanSansSC-Bold
    ]%
  \else
    \IfFontExistsTF{Source Han Sans SC}{%
      \setCJKsansfont{Source Han Sans SC}[Language=Chinese Simplified, ItalicFont={\mathnote@cjkitalicfont}, ItalicFeatures={\mathnote@cjkitalicfeatures}]
      \setCJKfamilyfont{hei}{Source Han Sans SC}[Language=Chinese Simplified]
    }{%
      \ifmathnote@haslocalcjksans
        \setCJKsansfont{NotoSansCJK-VF}[
          Path = {\MathNoteFontDir},
          Extension = .ttc,
          Language = Chinese Simplified,
          UprightFont = NotoSansCJK-VF,
          UprightFeatures = {FontIndex=2},
          BoldFont = NotoSansCJK-VF,
          BoldFeatures = {FontIndex=2,RawFeature={+wght=780}},
          ItalicFont = {\mathnote@cjkitalicfont},
          ItalicFeatures = {\mathnote@cjkitalicfeatures}
        ]%
        \setCJKfamilyfont{hei}{NotoSansCJK-VF}[
          Path = {\MathNoteFontDir},
          Extension = .ttc,
          UprightFont = NotoSansCJK-VF,
          UprightFeatures = {FontIndex=2},
          BoldFont = NotoSansCJK-VF,
          BoldFeatures = {FontIndex=2,RawFeature={+wght=820}}
        ]%
      \else
        \IfFontExistsTF{Noto Sans CJK SC}{%
          \setCJKsansfont{Noto Sans CJK SC}[Language=Chinese Simplified, ItalicFont={\mathnote@cjkitalicfont}, ItalicFeatures={\mathnote@cjkitalicfeatures}]
          \setCJKfamilyfont{hei}{Noto Sans CJK SC}[Language=Chinese Simplified]
        }{%
          \setCJKsansfont{FandolHei}[ItalicFont={\mathnote@cjkitalicfont}, ItalicFeatures={\mathnote@cjkitalicfeatures}]
          \setCJKfamilyfont{hei}{FandolHei}
        }%
      \fi
    }%
  \fi
  \ifmathnote@haslocalshsans
    \setCJKmonofont{SourceHanSansSC-Regular}[
      Path = {\MathNoteFontDir},
      Extension = .otf,
      Language = Chinese Simplified,
      BoldFont = SourceHanSansSC-Bold,
      ItalicFont = {\mathnote@cjkitalicfont},
      ItalicFeatures = {\mathnote@cjkitalicfeatures}
    ]%
  \else
    \IfFontExistsTF{Source Han Sans SC}{%
      \setCJKmonofont{Source Han Sans SC}[Language=Chinese Simplified, ItalicFont={\mathnote@cjkitalicfont}, ItalicFeatures={\mathnote@cjkitalicfeatures}]
    }{%
      \ifmathnote@haslocalcjkmono
        \setCJKmonofont{NotoSansMonoCJK-VF}[
          Path = {\MathNoteFontDir},
          Extension = .ttc,
          Language = Chinese Simplified,
          UprightFont = NotoSansMonoCJK-VF,
          UprightFeatures = {FontIndex=2},
          BoldFont = NotoSansMonoCJK-VF,
          BoldFeatures = {FontIndex=2,RawFeature={+wght=760}},
          ItalicFont = {\mathnote@cjkitalicfont},
          ItalicFeatures = {\mathnote@cjkitalicfeatures}
        ]%
      \else
        \IfFontExistsTF{Noto Sans Mono CJK SC}{%
          \setCJKmonofont{Noto Sans Mono CJK SC}[Language=Chinese Simplified, ItalicFont={\mathnote@cjkitalicfont}, ItalicFeatures={\mathnote@cjkitalicfeatures}]
        }{%
          \setCJKmonofont{FandolFang}[ItalicFont={\mathnote@cjkitalicfont}, ItalicFeatures={\mathnote@cjkitalicfeatures}]
        }%
      \fi
    }%
  \fi
  \ifmathnote@haslocalkai
    \setCJKfamilyfont{kai}{LXGW WenKai SC}[
      Path = {\MathNoteFontDir},
      Extension = .ttf,
      UprightFont = LXGWWenKaiSC-Regular,
      BoldFont = LXGWWenKaiSC-Medium,
      AutoFakeBold = 1.25,
      Language = Chinese Simplified
    ]%
    \setCJKfamilyfont{zhkai}{LXGW WenKai SC}[
      Path = {\MathNoteFontDir},
      Extension = .ttf,
      UprightFont = LXGWWenKaiSC-Regular,
      BoldFont = LXGWWenKaiSC-Medium,
      AutoFakeBold = 1.25,
      Language = Chinese Simplified
    ]%
  \else
    \IfFontExistsTF{LXGW WenKai SC}{%
      \setCJKfamilyfont{kai}{LXGW WenKai SC}[Language=Chinese Simplified]
      \setCJKfamilyfont{zhkai}{LXGW WenKai SC}[Language=Chinese Simplified]
    }{%
      \setCJKfamilyfont{kai}{FandolKai}[Language=Chinese Simplified]
      \setCJKfamilyfont{zhkai}{FandolKai}[Language=Chinese Simplified]
    }%
  \fi
}

\mathnote@setlatinfonts
\mathnote@setcjkfonts
\renewcommand{\kaishu}{\CJKfamily{kai}}
\xeCJKsetup{
  CheckSingle = true,
  RubberPunctSkip = true,
  PunctStyle = plain
}
\xeCJKsetwidth{,}{0.5em}
\xeCJKsetwidth{。}{1em}
\newlength{\mathnote@commaspace}
\setlength{\mathnote@commaspace}{0.5em}
\catcode`,=\active
\protected\def,{,\kern\mathnote@commaspace}
\clubpenalty=10000
\widowpenalty=10000
\displaywidowpenalty=10000

\usepackage{microtype}
\usepackage{setspace}
\setstretch{1.15}

\usepackage{amsmath, amssymb, amsthm, mathtools}
\usepackage{bm}
\usepackage{siunitx}
\usepackage{enumitem}
\usepackage{tikz}
\usetikzlibrary{calc, arrows.meta, decorations.pathmorphing, positioning}
\usepackage{xparse}
\usepackage{etoolbox}
\usepackage{graphicx}
\usepackage{svg}
\usepackage{caption}
\usepackage{booktabs}
\usepackage{tabularx}
\usepackage{multicol}
\usepackage{listings}
\usepackage{tcolorbox}
\tcbuselibrary{skins, breakable, hooks, listingsutf8}
\usepackage{zhnumber}
\usepackage{fancyhdr}
\usepackage{lastpage}
\usepackage{hyperref}
\usepackage{bookmark}

% Block-style paragraph headings to avoid run-in overfull boxes
\renewcommand{\paragraph}{%
  \@startsection{paragraph}{4}{\z@}%
    {1.5ex \@plus 0.5ex \@minus 0.2ex}%
    {0.65em}%
    {\normalfont\normalsize\bfseries}%
}

% --------------------------------------------------
% Colors
% --------------------------------------------------
\definecolor{screenAccent}{HTML}{1565C0}
\definecolor{screenSecondary}{HTML}{00897B}
\definecolor{screenHighlight}{HTML}{F9A826}
\definecolor{screenInfo}{HTML}{546E7A}
\definecolor{screenSurface}{HTML}{FFFFFF}

\definecolor{printAccent}{cmyk}{0.95,0.55,0,0.05}
\definecolor{printSecondary}{cmyk}{0.82,0,0.56,0.08}
\definecolor{printHighlight}{cmyk}{0,0.35,0.80,0}
\definecolor{printInfo}{cmyk}{0.60,0.47,0.43,0.20}
\definecolor{printSurface}{cmyk}{0,0,0,0}
\definecolor{mathnotePureCyan}{cmyk}{1,0,0,0}

% 16-color palettes for screen (sRGB)
\definecolor{screenTone01}{HTML}{0D47A1}
\definecolor{screenTone02}{HTML}{1565C0}
\definecolor{screenTone03}{HTML}{1A73E8}
\definecolor{screenTone04}{HTML}{2196F3}
\definecolor{screenTone05}{HTML}{00ACC1}
\definecolor{screenTone06}{HTML}{00897B}
\definecolor{screenTone07}{HTML}{2E7D32}
\definecolor{screenTone08}{HTML}{558B2F}
\definecolor{screenTone09}{HTML}{9E9D24}
\definecolor{screenTone10}{HTML}{F9A825}
\definecolor{screenTone11}{HTML}{FFB300}
\definecolor{screenTone12}{HTML}{FB8C00}
\definecolor{screenTone13}{HTML}{F4511E}
\definecolor{screenTone14}{HTML}{D84315}
\definecolor{screenTone15}{HTML}{8E24AA}
\definecolor{screenTone16}{HTML}{6A1B9A}

% 16-color palettes for print (CMYK approximations)
\definecolor{printTone01}{cmyk}{1,0.72,0,0.35}
\definecolor{printTone02}{cmyk}{0.9,0.5,0,0.2}
\definecolor{printTone03}{cmyk}{0.85,0.45,0,0.12}
\definecolor{printTone04}{cmyk}{0.65,0.25,0,0.02}
\definecolor{printTone05}{cmyk}{0.75,0.05,0.1,0.05}
\definecolor{printTone06}{cmyk}{0.85,0,0.35,0.2}
\definecolor{printTone07}{cmyk}{0.75,0,0.8,0.38}
\definecolor{printTone08}{cmyk}{0.6,0,1,0.42}
\definecolor{printTone09}{cmyk}{0.35,0,1,0.45}
\definecolor{printTone10}{cmyk}{0,0.2,1,0.02}
\definecolor{printTone11}{cmyk}{0,0.25,1,0}
\definecolor{printTone12}{cmyk}{0,0.45,1,0}
\definecolor{printTone13}{cmyk}{0,0.7,0.8,0}
\definecolor{printTone14}{cmyk}{0,0.85,0.95,0.1}
\definecolor{printTone15}{cmyk}{0.45,0.9,0,0}
\definecolor{printTone16}{cmyk}{0.6,1,0,0.1}

\colorlet{accent}{screenAccent}
\colorlet{secondary}{screenSecondary}
\colorlet{highlight}{screenHighlight}
\colorlet{inkgray}{screenInfo}
\colorlet{surface}{screenSurface}

\newif\ifmathnote@docstarted
\mathnote@docstartedfalse
\AtBeginDocument{\mathnote@docstartedtrue}

\newcommand{\mathnote@applyhypercolors}{%
  \ifmathnoteprintmode
    \hypersetup{
      colorlinks=false,
      hidelinks,
      pdfborderstyle={/S/U/W 0},
      pdfborder={0 0 0}
    }%
  \else
    \hypersetup{
      colorlinks=true,
      linkcolor=accent,
      citecolor=secondary,
      urlcolor=accent,
      pdfborder={0 0 0}
    }%
  \fi
}

\newcommand{\mathnote@applypalette}{%
  \ifmathnoteprintmode
    \colorlet{accent}{printAccent}%
    \colorlet{secondary}{printSecondary}%
    \colorlet{highlight}{printHighlight}%
    \colorlet{inkgray}{printInfo}%
    \colorlet{surface}{printSurface}%
  \else
    \colorlet{accent}{screenAccent}%
    \colorlet{secondary}{screenSecondary}%
    \colorlet{highlight}{screenHighlight}%
    \colorlet{inkgray}{screenInfo}%
    \colorlet{surface}{screenSurface}%
  \fi
  \colorlet{accentline}{accent!65!black}%
  \colorlet{accentbg}{accent!8!white}%
  \colorlet{secondarybg}{secondary!10!white}%
  \colorlet{highlightbg}{highlight!10!white}%
  \colorlet{inkline}{inkgray!60!black}%
  \colorlet{surfacegrid}{inkgray!6!white}%
  \ifmathnote@docstarted
    \mathnote@applyhypercolors
  \else
    \AtBeginDocument{\mathnote@applyhypercolors}
  \fi
}
\mathnote@applypalette
\newcommand{\MathNoteRefreshColors}{\mathnote@applypalette}
\AtBeginDocument{%
  \hypersetup{%
    pdftitle=\notetitle,
    pdfauthor=\noteauthor,
    pdfsubject=\notesubtitle,
    pdfcreator={MathNote dual-medium template}%
  }%
}

% --------------------------------------------------
% Sectioning and spacing
% --------------------------------------------------
\setlength{\parskip}{0.35em}
\setlength{\parindent}{2em}
\newlength{\mathnote@boxindent}
\setlength{\mathnote@boxindent}{\parindent}
\setcounter{secnumdepth}{3}
\setcounter{tocdepth}{2}


\ctexset{
  section={
    name={第,节},
    format+=\Large\sffamily\bfseries\color{accent},
    beforeskip=1.2em,
    afterskip=0.7em
  },
  subsection={
    format+=\large\sffamily\bfseries\color{secondary},
    beforeskip=1em,
    afterskip=0.4em
  },
  subsubsection={
    format+=\normalsize\sffamily\bfseries\color{inkgray},
    beforeskip=0.8em,
    afterskip=0.2em
  }
}

% --------------------------------------------------
% Header / footer
% --------------------------------------------------
\pagestyle{fancy}
\fancyhf{}
\fancyhead[LE]{\small\sffamily\textcolor{accent}{\notetitle\ >\ \nouppercase{\rightmark}}}
\fancyhead[RO]{\small\sffamily\textcolor{accent}{\notetitle\ >\ \nouppercase{\rightmark}}}
\fancyfoot[LE]{\small\sffamily\textcolor{inkgray}{\thepage}\ \textcolor{mathnotePureCyan}{/}\ \textcolor{inkgray}{\pageref{LastPage}}}
\fancyfoot[RO]{\small\sffamily\textcolor{inkgray}{\thepage}\ \textcolor{mathnotePureCyan}{/}\ \textcolor{inkgray}{\pageref{LastPage}}}
\fancyfoot[LO]{}
\fancyfoot[RE]{}
\renewcommand{\headrulewidth}{0.2pt}
\renewcommand{\footrulewidth}{0pt}
\renewcommand{\sectionmark}[1]{\markright{#1}}

% --------------------------------------------------
% Math helpers
% --------------------------------------------------
\DeclarePairedDelimiter\abs{\lvert}{\rvert}
\DeclarePairedDelimiter\norm{\lVert}{\rVert}
\DeclarePairedDelimiter\ceil{\lceil}{\rceil}
\DeclarePairedDelimiter\floor{\lfloor}{\rfloor}

\newcommand{\R}{\mathbb{R}}
\newcommand{\C}{\mathbb{C}}
\newcommand{\Q}{\mathbb{Q}}
\newcommand{\Z}{\mathbb{Z}}
\newcommand{\N}{\mathbb{N}}
\newcommand{\dd}{\mathop{}\!\mathrm{d}}
\newcommand{\ee}{\mathrm{e}}
\newcommand{\dv}[2]{\frac{\dd #1}{\dd #2}}
\newcommand{\pdv}[2]{\frac{\partial #1}{\partial #2}}

\lstset{
  backgroundcolor=\color{surfacegrid},
  basicstyle=\ttfamily\small,
  keywordstyle=\color{screenTone04}\bfseries,
  commentstyle=\color{screenTone06},
  stringstyle=\color{screenTone12},
  frame=none,
  columns=fullflexible,
  showstringspaces=false
}

\NewDocumentCommand{\keyword}{m}{%
  \textcolor{accent}{\textbf{#1}}%
}

\NewDocumentCommand{\inlinehint}{m}{%
  \textcolor{secondary}{\sffamily\footnotesize #1}%
}

\NewDocumentCommand{\MathNotePaletteSwatch}{mm}{%
  \tikz[baseline=(label.base)]{
    \node[rounded corners=2pt, draw=#1!65!black, fill=#1, minimum width=0.85cm, minimum height=0.4cm] (chip) {};
    \node[right=0.28cm of chip, anchor=west, font=\sffamily\scriptsize\color{inkgray}] (label) {#2};
  }%
}

\NewDocumentCommand{\ModeBadge}{O{accent}m}{%
  \tikz[baseline=(label.base)]\node[label/.style={}] (label) [inner xsep=6pt, inner ysep=1.6pt, rounded corners=2pt, fill=#1!12!white, draw=#1!80!black, font=\sffamily\scriptsize\bfseries\color{#1!30!black}] {#2};%
}

\newcommand{\mathnote@ifblank}[3]{%
  \if\relax\detokenize{#1}\relax
    #2%
  \else
    #3%
  \fi
}

\newenvironment{focuspoints}{%
  \begin{itemize}[label=\tikz{\filldraw[accent] (0,0) circle (2pt);}, leftmargin=1.8em, itemsep=0.2em, topsep=0.1em]
}{\end{itemize}}

\newcounter{roadmapstep}
\newlength{\mathnote@roadmapindent}
\setlength{\mathnote@roadmapindent}{1.4em}
\newcommand{\mathnote@roadmaparrow}{%
  \par\smallskip
  \noindent\hspace{1.7em}\tikz{
    \draw[accent, line width=0.85pt, -{Latex[length=3mm]}] (0,0) -- (0,-0.9);
  }%
  \par\smallskip
}
\NewDocumentEnvironment{roadmap}{O{}}{%
  \par\smallskip
  \setcounter{roadmapstep}{0}%
}{%
  \par\smallskip
}
\newcommand{\RoadmapStep}[1]{%
  \stepcounter{roadmapstep}%
  \ifnum\value{roadmapstep}>1
    \mathnote@roadmaparrow
  \fi
  {%
    \noindent\parfillskip=0pt plus 1fil\ModeBadge[accent]{第\zhnumber{\value{roadmapstep}}步}\par
    \vspace{0.2em}%
    \noindent\hspace{\mathnote@roadmapindent}%
    \begin{minipage}[t]{\dimexpr\linewidth-\mathnote@roadmapindent\relax}
      \raggedright\sloppy #1
    \end{minipage}\par
  }%
}

% --------------------------------------------------
% Box styles
% --------------------------------------------------
\tcbset{
  mathnote box/.style={
    enhanced,
    sharp corners,
    boxrule=0.5pt,
    colback=surface,
    coltitle=inkgray,
    fonttitle=\sffamily\bfseries,
    left=1em,
    right=1em,
    top=0.7em,
    bottom=0.7em,
    before skip=10pt,
    after skip=10pt,
    breakable,
    width=\dimexpr\linewidth-\mathnote@boxindent\relax,
    left skip=\mathnote@boxindent,
    borderline west={1pt}{0pt}{accentline}
  }
}
\newtcolorbox{definitionbox}[2][]{%
  mathnote box,
  title=\mathnote@ifblank{#2}{定义}{#2},
  colback=surface,
  colframe=secondary!70!black,
  coltitle=secondary!15!white,
  fonttitle=\sffamily\bfseries\color{secondary!35!white},
  borderline west={2pt}{0pt}{secondary},
  #1
}

\newtcolorbox{theorembox}[2][]{%
  mathnote box,
  title=\mathnote@ifblank{#2}{定理}{#2},
  colback=surface,
  colframe=accent!70!black,
  coltitle=accent!10!white,
  fonttitle=\sffamily\bfseries\color{accent!35!white},
  borderline west={2pt}{0pt}{accent},
  #1
}

\newtcolorbox{examplebox}[2][]{%
  mathnote box,
  title=\mathnote@ifblank{#2}{例题}{#2},
  colback=surface,
  colframe=highlight!80!black,
  coltitle=highlight!15!white,
  fonttitle=\sffamily\bfseries\color{highlight!40!white},
  borderline west={2pt}{0pt}{highlight},
  #1
}

\newtcolorbox{lemmabox}[2][]{%
  mathnote box,
  title=\mathnote@ifblank{#2}{引理}{#2},
  colback=surface,
  colframe=inkline,
  coltitle=inkgray!30!white,
  fonttitle=\sffamily\bfseries\color{inkgray!45!white},
  borderline west={2pt}{0pt}{inkgray},
  #1
}

\newtcolorbox{notebox}[2][]{%
  mathnote box,
  title=\mathnote@ifblank{#2}{提示}{#2},
  colback=surface,
  colframe=highlight!60!black,
  coltitle=highlight!20!white,
  fonttitle=\sffamily\bfseries\color{highlight!45!white},
  borderline west={2pt}{0pt}{highlight},
  #1
}

\newtcolorbox{summarybox}[2][]{%
  mathnote box,
  title=\mathnote@ifblank{#2}{总结}{#2},
  colback=surface,
  colframe=accent!20!black,
  borderline west={2pt}{0pt}{accent},
  coltitle=accent!10!white,
  fonttitle=\sffamily\bfseries\color{accent!45!white},
  #1
}

\newtcolorbox{conceptbox}[2][]{%
  mathnote box,
  title=\mathnote@ifblank{#2}{概念骨架}{#2},
  colback=surface,
  colframe=secondary!40!black,
  coltitle=secondary!15!white,
  fonttitle=\sffamily\bfseries\color{secondary!40!white},
  borderline west={2pt}{0pt}{secondary},
  #1
}

\newtcolorbox{proofbox}[2][]{%
  mathnote box,
  title=\mathnote@ifblank{#2}{证明}{#2},
  colback=surface,
  colframe=inkline,
  coltitle=inkgray!35!white,
  fonttitle=\sffamily\bfseries\color{inkgray!60!white},
  borderline west={2pt}{0pt}{inkline},
  #1
}

\newtcolorbox{warningbox}[2][]{%
  mathnote box,
  title=\mathnote@ifblank{#2}{排版警示}{#2},
  colback=surface,
  colframe=highlight!80!black,
  borderline west={2pt}{0pt}{highlight},
  coltitle=highlight!20!white,
  fonttitle=\sffamily\bfseries\color{highlight!45!white},
  #1
}

% --------------------------------------------------
% TikZ styles
% --------------------------------------------------
\tikzset{
  mathnote lines/.style={
    line width=0.8pt,
    >=Stealth,
    draw=accentline,
    text=inkgray
  },
  mathnote grid/.style={
    color=inkgray!30,
    line width=0.3pt
  }
}

% --------------------------------------------------
% Tables and lists
% --------------------------------------------------
\newcommand{\mathnote@listbarbegin}[2][0.8em]{%
  \par\noindent
  \begin{tcolorbox}[
    blanker,
    enhanced,
    sharp corners,
    boxrule=0pt,
    colback=surface,
    left=#1,
    right=0pt,
    top=0.25em,
    bottom=0.25em,
    borderline west={1.3pt}{0pt}{#2}
  ]%
  \ignorespaces
}
\newcommand{\mathnote@listbarend}{\end{tcolorbox}\ignorespacesafterend}

\setlist[itemize]{leftmargin=1.8em, itemsep=0.25em, before=\mathnote@listbarbegin{accent}, after=\mathnote@listbarend}
\setlist[enumerate]{leftmargin=2.1em, itemsep=0.3em, label=\textbf{\arabic*.}, before=\mathnote@listbarbegin[1em]{secondary}, after=\mathnote@listbarend}
\setlist[description]{font=\sffamily\bfseries, labelsep=0.5em}

\renewcommand{\arraystretch}{1.2}
\captionsetup{font=small, labelfont=bf}

% --------------------------------------------------
% Utility commands
% --------------------------------------------------
\newcommand{\ScreenOnly}[1]{\ifmathnoteprintmode\else #1\fi}
\newcommand{\PrintOnly}[1]{\ifmathnoteprintmode #1\fi}
\newcommand{\DualMode}[2]{\ifmathnoteprintmode #2\else #1\fi}

\NewDocumentCommand{\PageTag}{O{accent}m}{%
  \begin{tikzpicture}[remember picture, overlay]
    \node[anchor=north east, xshift=-6mm, yshift=-10mm, fill=#1, text=white, rounded corners=2pt, inner xsep=6pt, inner ysep=2pt, font=\sffamily\footnotesize] at (current page.north east) {#2};
  \end{tikzpicture}%
}

\newcommand{\SectionTag}[1]{%
  \textcolor{accent}{\Large\bfseries\sffamily #1}%
}

\makeatother


\renewcommand{\notetitle}{定点固定速度抛射的包络线}
\renewcommand{\noteauthor}{Gemini 数学组}
\renewcommand{\notedate}{\today}

\title{\Huge \bfseries \color{myblue} \notetitle}
\author{\noteauthor}
\date{\notedate}

\begin{document}

% 标题页
\maketitle

% 目录
\tableofcontents
\newpage

% ============================================
% 1. 引言与问题背景
% ============================================
\section{引言与问题背景}

\begin{definitionbox}{包络线}
\textbf{包络线(Envelope)}是指与一族曲线都相切的曲线。在几何上,包络线是这族曲线的“边界”,它恰好与族中的每一条曲线在某一点相切。

对于抛体运动,如果我们从同一点以相同的初速度 $v_0$ 但不同的抛射角 $\theta$ 抛出多个质点,这些质点的轨迹形成一族抛物线。这族抛物线的包络线就是一条特殊的曲线,它表示在给定初速度下,质点能够到达的所有位置点的边界。
\end{definitionbox}

\begin{center}
\begin{tikzpicture}[scale=2.0, x=2.5cm, y=6cm, >=Stealth]
    % 参数设置(先计算范围)
    \pgfmathsetmacro{\vzero}{3}
    \pgfmathsetmacro{\g}{9.8}
    \pgfmathsetmacro{\hmax}{\vzero*\vzero/(2*\g)}
    \pgfmathsetmacro{\xmaxenv}{\vzero*\vzero/\g}
    
    % 坐标轴
    \draw[->, thick, black] (-0.1,0) -- (1.1*\xmaxenv,0) node[right, font=\normalsize] {$x$ (m)};
    \draw[->, thick, black] (0,-0.05) -- (0,1.1*\hmax) node[above, font=\normalsize] {$y$ (m)};
    \node[black] at (0,0) [below left, font=\normalsize] {$O$};
    
    % 网格线(辅助)
    \draw[gray!20, very thin] (0,0) grid (1.05*\xmaxenv,1.05*\hmax);
    
    % 抛射点
    \fill[myred] (0,0) circle (2pt);
    \node[myred, below left, font=\normalsize] at (0,0) {抛射点};
    
    % 多条不同角度的轨迹(示意)
    \foreach \angle in {10, 20, 30, 40, 50, 60, 70, 80}
    {
        \pgfmathsetmacro{\tantheta}{tan(\angle)}
        \pgfmathsetmacro{\costheta}{cos(\angle)}
        \pgfmathsetmacro{\sintheta}{sin(\angle)}
        \pgfmathsetmacro{\xmax}{2*\vzero*\vzero*\sintheta*\costheta/\g}
        
        \draw[myblue!35, thin, domain=0:\xmax, samples=80] 
            plot (\x, {\x*\tantheta - \g*\x*\x/(2*\vzero*\vzero*\costheta*\costheta)});
    }
    
    % 特殊角度轨迹(加粗显示)
    \foreach \angle in {30, 45, 60}
    {
        \pgfmathsetmacro{\tantheta}{tan(\angle)}
        \pgfmathsetmacro{\costheta}{cos(\angle)}
        \pgfmathsetmacro{\sintheta}{sin(\angle)}
        \pgfmathsetmacro{\xmax}{2*\vzero*\vzero*\sintheta*\costheta/\g}
        
        \draw[myblue!70, semithick, domain=0:\xmax, samples=80] 
            plot (\x, {\x*\tantheta - \g*\x*\x/(2*\vzero*\vzero*\costheta*\costheta)});
    }
    
    % 包络线(抛物线,粗线)
    \draw[myred, very thick, line width=2pt, domain=0:\xmaxenv, samples=150] 
        plot (\x, {\hmax - \g*\x*\x/(2*\vzero*\vzero)});
    
    % 包络线标注
    \node[myred, above right, font=\normalsize\bfseries] at (0.6*\xmaxenv, 0.6*\hmax) {包络线};
    \draw[myred, ->, thick] (0.55*\xmaxenv, 0.55*\hmax) -- (0.45*\xmaxenv, 0.45*\hmax);
    
    % 轨迹族标注
    \node[myblue, right, font=\normalsize] at (0.9*\xmaxenv, 0.2*\hmax) {轨迹族};
    
    % 添加一些关键点的标注
    \fill[mygreen] (0, \hmax) circle (1.5pt);
    \draw[dashed, mygreen!60] (0, 0) -- (0, \hmax);
    \node[mygreen, left, font=\small] at (0, \hmax) {$\frac{v_0^2}{2g}$};
\end{tikzpicture}
\captionof{figure}{包络线的直观理解:多条轨迹的边界}
\end{center}

\subsection{问题的物理意义}

在抛体运动中,如果我们固定初速度 $v_0$,只改变抛射角 $\theta$,那么:
\begin{itemize}
    \item 不同的抛射角对应不同的轨迹抛物线
    \item 这些轨迹形成一个“轨迹族”
    \item 包络线表示在给定初速度下,质点能够到达的所有位置点的\textbf{边界}
    \item 包络线内部的区域是“可达区域”,外部的区域是“不可达区域”
\end{itemize}

这个问题在军事、体育、工程等领域都有重要应用,例如:
\begin{itemize}
    \item 确定炮弹的射程范围
    \item 分析投掷物体的可达区域
    \item 设计安全防护区域
\end{itemize}

% ============================================
% 2. 抛体运动基础回顾
% ============================================
\section{抛体运动基础回顾}

\begin{definitionbox}{抛体运动的参数方程}
设质点从原点 $O(0,0)$ 以初速度 $v_0$、抛射角 $\theta$(与水平方向夹角)抛出,重力加速度为 $g$(方向竖直向下)。

\textbf{运动分解:}
\begin{itemize}
    \item 水平方向:匀速直线运动,初速度 $v_{0x} = v_0\cos\theta$
    \item 竖直方向:匀变速直线运动,初速度 $v_{0y} = v_0\sin\theta$,加速度 $-g$
\end{itemize}

\textbf{参数方程(以时间 $t$ 为参数):}
\begin{align}
    x(t) &= v_0\cos\theta \cdot t \label{eq:x}\\
    y(t) &= v_0\sin\theta \cdot t - \frac{1}{2}gt^2 \label{eq:y}
\end{align}

其中 $t \geq 0$,且 $y(t) \geq 0$(落地前)。
\end{definitionbox}

\begin{center}
\begin{tikzpicture}[scale=4.0, x=2.5cm, y=4cm, >=Stealth]
    % 参数设置(先计算范围)
    \pgfmathsetmacro{\vzero}{2}
    \pgfmathsetmacro{\g}{9.8}
    \pgfmathsetmacro{\angle}{40}
    \pgfmathsetmacro{\tantheta}{tan(\angle)}
    \pgfmathsetmacro{\costheta}{cos(\angle)}
    \pgfmathsetmacro{\sintheta}{sin(\angle)}
    \pgfmathsetmacro{\xmax}{2*\vzero*\vzero*\sintheta*\costheta/\g}
    \pgfmathsetmacro{\ymax}{\vzero*\vzero*\sintheta*\sintheta/(2*\g)}
    
    % 坐标轴
    \draw[->, thick, black] (-0.1,0) -- (1.1*\xmax,0) node[right, font=\normalsize] {$x$ (m)};
    \draw[->, thick, black] (0,-0.05) -- (0,1.2*\ymax) node[above, font=\normalsize] {$y$ (m)};
    \node[black] at (0,0) [below left, font=\normalsize] {$O$};
    
    % 网格线(辅助)
    \draw[gray!20, very thin] (0,0) grid (1.05*\xmax,1.15*\ymax);
    
    % 抛射点
    \fill[myred] (0,0) circle (1.5pt);
    
    % 速度矢量分解(缩短箭头长度)
    \pgfmathsetmacro{\vlen}{0.5}
    \pgfmathsetmacro{\vx}{\vlen*cos(\angle)}
    \pgfmathsetmacro{\vy}{\vlen*sin(\angle)}
    
    % 总速度矢量
    \draw[->, myred, very thick, line width=2.5pt] (0,0) -- (\vx, \vy);
    \node[myred, above left, font=\normalsize\bfseries] at (0.5*\vx, 0.5*\vy) {$v_0$};
    
    % 水平分量
    \draw[->, myblue, thick, line width=2pt] (0,0) -- (\vx, 0);
    \node[myblue, below, font=\normalsize] at (0.5*\vx, 0) {$v_{0x} = v_0\cos\theta$};
    
    % 竖直分量
    \draw[->, mygreen, thick, line width=2pt] (\vx, 0) -- (\vx, \vy);
    \node[mygreen, right, font=\normalsize] at (\vx, 0.5*\vy) {$v_{0y} = v_0\sin\theta$};
    
    % 辅助线(虚线)
    \draw[dashed, gray!50] (\vx, 0) -- (\vx, \vy);
    \draw[dashed, gray!50] (0, \vy) -- (\vx, \vy);
    
    % 角度标注
    \draw[myorange, thick] (0.35, 0) arc (0:\angle:0.35);
    \node[myorange, font=\normalsize] at (0.5, 0.12) {$\theta$};
    
    % 轨迹(抛物线)
    \draw[myorange, very thick, line width=2pt, domain=0:\xmax, samples=120] 
        plot (\x, {\x*\tantheta - \g*\x*\x/(2*\vzero*\vzero*\costheta*\costheta)});
    
    % 关键点标注
    \fill[myred] (0.5*\xmax, \ymax) circle (2pt);
    \draw[dashed, myred!60] (0.5*\xmax, 0) -- (0.5*\xmax, \ymax);
    \draw[dashed, myred!60] (0, \ymax) -- (0.5*\xmax, \ymax);
    \node[myred, above, font=\normalsize] at (0.5*\xmax, \ymax) {最高点};
    \node[myred, below, font=\small] at (0.5*\xmax, 0) {$R/2$};
    \node[myred, left, font=\small] at (0, \ymax) {$H$};
    
    \fill[myred] (\xmax, 0) circle (2pt);
    \draw[dashed, myred!60] (\xmax, 0) -- (\xmax, 0.1);
    \node[myred, below right, font=\normalsize] at (\xmax, 0) {落地点};
    \node[myred, below, font=\small] at (\xmax, 0) {$R$};
    
    % 轨迹标注
    \node[myorange, right, font=\normalsize\bfseries] at (0.7*\xmax, 0.5*\ymax) {轨迹};
\end{tikzpicture}
\captionof{figure}{抛体运动的速度分解与轨迹}
\end{center}

\subsection{轨迹方程}

从参数方程 \eqref{eq:x} 和 \eqref{eq:y} 中消去参数 $t$,得到轨迹方程:

由 $x = v_0\cos\theta \cdot t$ 得 $t = \frac{x}{v_0\cos\theta}$,代入 $y$ 的表达式:

\begin{align}
    y &= v_0\sin\theta \cdot \frac{x}{v_0\cos\theta} - \frac{1}{2}g\left(\frac{x}{v_0\cos\theta}\right)^2 \nonumber\\
    &= x\tan\theta - \frac{gx^2}{2v_0^2\cos^2\theta} \nonumber\\
    &= x\tan\theta - \frac{gx^2}{2v_0^2}(1 + \tan^2\theta) \label{eq:trajectory}
\end{align}

这是一个关于 $x$ 的二次函数,轨迹为抛物线。

\textbf{重要结论:}
\begin{itemize}
    \item 射程(水平距离):$R = \frac{v_0^2\sin(2\theta)}{g}$,当 $\theta = 45°$ 时取得最大值 $R_{\max} = \frac{v_0^2}{g}$
    \item 最大高度:$H = \frac{v_0^2\sin^2\theta}{2g}$,当 $\theta = 90°$ 时取得最大值 $H_{\max} = \frac{v_0^2}{2g}$
\end{itemize}

% ============================================
% 3. 固定速度抛射的轨迹族
% ============================================
\section{固定速度抛射的轨迹族}

当我们固定初速度 $v_0$,让抛射角 $\theta$ 变化时,得到一族轨迹曲线。

\begin{definitionbox}{轨迹族}
对于固定的初速度 $v_0$,抛射角 $\theta$ 作为参数,轨迹方程 \eqref{eq:trajectory} 可以写成:
\begin{equation}
    y = x\tan\theta - \frac{gx^2}{2v_0^2}(1 + \tan^2\theta), \quad \theta \in \left(0, \frac{\pi}{2}\right) \label{eq:family}
\end{equation}

这表示一个以 $\theta$(或 $\tan\theta$)为参数的曲线族。每条曲线对应一个特定的抛射角。
\end{definitionbox}

\begin{center}
\begin{tikzpicture}[scale=2.0, x=2.5cm, y=6cm, >=Stealth]
    % 参数设置(先计算范围)
    \pgfmathsetmacro{\vzero}{3}
    \pgfmathsetmacro{\g}{9.8}
    \pgfmathsetmacro{\hmax}{\vzero*\vzero/(2*\g)}
    \pgfmathsetmacro{\xmaxenv}{\vzero*\vzero/\g}
    
    % 坐标轴
    \draw[->, thick, black] (-0.1,0) -- (1.1*\xmaxenv,0) node[right, font=\normalsize] {$x$ (m)};
    \draw[->, thick, black] (0,-0.05) -- (0,1.1*\hmax) node[above, font=\normalsize] {$y$ (m)};
    \node[black] at (0,0) [below left, font=\normalsize] {$O$};
    
    % 网格线(辅助)
    \draw[gray!20, very thin] (0,0) grid (1.05*\xmaxenv,1.05*\hmax);
    
    % 抛射点
    \fill[myred] (0,0) circle (2pt);
    \node[myred, below left, font=\small] at (0,0) {抛射点};
    
    % 多条不同角度的轨迹
    \foreach \angle in {10, 15, 20, 25, 30, 35, 40, 50, 55, 60, 65, 70, 75, 80}
    {
        \pgfmathsetmacro{\tantheta}{tan(\angle)}
        \pgfmathsetmacro{\costheta}{cos(\angle)}
        \pgfmathsetmacro{\sintheta}{sin(\angle)}
        \pgfmathsetmacro{\xmax}{2*\vzero*\vzero*\sintheta*\costheta/\g}
        
        \draw[myblue!40, thin, domain=0:\xmax, samples=60] 
            plot (\x, {\x*\tantheta - \g*\x*\x*(1+\tantheta*\tantheta)/(2*\vzero*\vzero)});
    }
    
    % 特殊角度轨迹(加粗显示)
    % 30度和60度
    \foreach \angle in {30, 60}
    {
        \pgfmathsetmacro{\tantheta}{tan(\angle)}
        \pgfmathsetmacro{\costheta}{cos(\angle)}
        \pgfmathsetmacro{\sintheta}{sin(\angle)}
        \pgfmathsetmacro{\xmax}{2*\vzero*\vzero*\sintheta*\costheta/\g}
        
        \draw[myblue!70, semithick, domain=0:\xmax, samples=80] 
            plot (\x, {\x*\tantheta - \g*\x*\x*(1+\tantheta*\tantheta)/(2*\vzero*\vzero)});
    }
    
    % 45度(特殊颜色)
    \pgfmathsetmacro{\angle}{45}
    \pgfmathsetmacro{\tantheta}{tan(\angle)}
    \pgfmathsetmacro{\costheta}{cos(\angle)}
    \pgfmathsetmacro{\sintheta}{sin(\angle)}
    \pgfmathsetmacro{\xmax}{2*\vzero*\vzero*\sintheta*\costheta/\g}
    \draw[mygreen, thick, line width=1.5pt, domain=0:\xmax, samples=80] 
        plot (\x, {\x*\tantheta - \g*\x*\x*(1+\tantheta*\tantheta)/(2*\vzero*\vzero)});
    
    % 45度角标注
    \pgfmathsetmacro{\angle}{45}
    \pgfmathsetmacro{\tantheta}{tan(\angle)}
    \pgfmathsetmacro{\sintheta}{sin(\angle)}
    \pgfmathsetmacro{\costheta}{cos(\angle)}
    \pgfmathsetmacro{\xmax}{2*\vzero*\vzero*\sintheta*\costheta/\g}
    \pgfmathsetmacro{\ymax}{\vzero*\vzero*\sintheta*\sintheta/(2*\g)}
    \node[mygreen, above, font=\normalsize\bfseries] at (0.5*\xmax, \ymax) {$\theta=45°$};
    
    % 其他角度标注
    \pgfmathsetmacro{\angle}{30}
    \pgfmathsetmacro{\tantheta}{tan(\angle)}
    \pgfmathsetmacro{\sintheta}{sin(\angle)}
    \pgfmathsetmacro{\costheta}{cos(\angle)}
    \pgfmathsetmacro{\xmax}{2*\vzero*\vzero*\sintheta*\costheta/\g}
    \pgfmathsetmacro{\ymax}{\vzero*\vzero*\sintheta*\sintheta/(2*\g)}
    \node[myblue, above left, font=\small] at (0.5*\xmax, \ymax) {$\theta=30°$};
    
    \pgfmathsetmacro{\angle}{60}
    \pgfmathsetmacro{\tantheta}{tan(\angle)}
    \pgfmathsetmacro{\sintheta}{sin(\angle)}
    \pgfmathsetmacro{\costheta}{cos(\angle)}
    \pgfmathsetmacro{\xmax}{2*\vzero*\vzero*\sintheta*\costheta/\g}
    \pgfmathsetmacro{\ymax}{\vzero*\vzero*\sintheta*\sintheta/(2*\g)}
    \node[myblue, above right, font=\small] at (0.5*\xmax, \ymax) {$\theta=60°$};
    
    % 标注
    \node[myblue, right, font=\normalsize] at (0.9*\xmaxenv, 0.3*\hmax) {轨迹族};
\end{tikzpicture}
\captionof{figure}{固定初速度 $v_0$ 下不同抛射角的轨迹族}
\end{center}

\subsection{轨迹族的特点}

观察轨迹族,我们可以发现:
\begin{itemize}
    \item 所有轨迹都从同一点 $(0,0)$ 出发
    \item 不同角度的轨迹有不同的射程和最大高度
    \item 这些轨迹之间存在一条\textbf{包络线},它是所有轨迹的边界
    \item 包络线内部的区域是可达区域,外部是不可达区域
\end{itemize}

% ============================================
% 4. 包络线的数学推导
% ============================================
\section{包络线的数学推导}

\begin{definitionbox}{包络线的数学定义}
设曲线族由方程 $F(x, y, \alpha) = 0$ 给出,其中 $\alpha$ 是参数。如果存在一条曲线 $C$,使得:
\begin{enumerate}
    \item $C$ 上的每一点都在族中某条曲线上
    \item $C$ 在每一点都与族中经过该点的曲线相切
\end{enumerate}
则称 $C$ 为该曲线族的\textbf{包络线}。

包络线的求法:联立方程组
\begin{equation}
    \begin{cases}
        F(x, y, \alpha) = 0 \\
        \frac{\partial F}{\partial \alpha}(x, y, \alpha) = 0
    \end{cases}
\end{equation}
消去参数 $\alpha$,即可得到包络线的方程。
\end{definitionbox}

\subsection{包络线的推导过程}

对于抛体运动的轨迹族,我们使用 $\tan\theta$ 作为参数会更方便。设 $k = \tan\theta$,则轨迹方程 \eqref{eq:family} 变为:

\begin{equation}
    F(x, y, k) = y - xk + \frac{gx^2}{2v_0^2}(1 + k^2) = 0 \label{eq:F}
\end{equation}

对参数 $k$ 求偏导数:

\begin{equation}
    \frac{\partial F}{\partial k} = -x + \frac{gx^2}{2v_0^2} \cdot 2k = -x + \frac{gx^2k}{v_0^2} = 0 \label{eq:partial}
\end{equation}

由 \eqref{eq:partial} 得:
\begin{equation}
    -x + \frac{gx^2k}{v_0^2} = 0 \quad \Rightarrow \quad x\left(-1 + \frac{gxk}{v_0^2}\right) = 0
\end{equation}

当 $x = 0$ 时,对应抛射点,不是包络线上的点(除起点外)。因此:
\begin{equation}
    -1 + \frac{gxk}{v_0^2} = 0 \quad \Rightarrow \quad k = \frac{v_0^2}{gx} \label{eq:k}
\end{equation}

将 \eqref{eq:k} 代入 \eqref{eq:F}:

\begin{align}
    y &= x \cdot \frac{v_0^2}{gx} - \frac{gx^2}{2v_0^2}\left(1 + \left(\frac{v_0^2}{gx}\right)^2\right) \nonumber\\
    &= \frac{v_0^2}{g} - \frac{gx^2}{2v_0^2}\left(1 + \frac{v_0^4}{g^2x^2}\right) \nonumber\\
    &= \frac{v_0^2}{g} - \frac{gx^2}{2v_0^2} - \frac{gx^2}{2v_0^2} \cdot \frac{v_0^4}{g^2x^2} \nonumber\\
    &= \frac{v_0^2}{g} - \frac{gx^2}{2v_0^2} - \frac{v_0^2}{2g} \nonumber\\
    &= \frac{v_0^2}{2g} - \frac{gx^2}{2v_0^2}
\end{align}

\begin{theorembox}{抛体运动轨迹族的包络线方程}
从同一点以固定初速度 $v_0$、不同抛射角 $\theta$ 抛出的质点,其轨迹族的包络线方程为:
\begin{equation}
    y = \frac{v_0^2}{2g} - \frac{gx^2}{2v_0^2} \label{eq:envelope}
\end{equation}

这是一个开口向下的抛物线,其性质为:
\begin{itemize}
    \item 顶点:$\left(0, \frac{v_0^2}{2g}\right)$,这是竖直上抛能达到的最大高度
    \item 对称轴:$x = 0$($y$ 轴)
    \item 与 $x$ 轴的交点:$x = \pm \frac{v_0^2}{g}$,这是水平抛射的最大射程
\end{itemize}
\end{theorembox}

\subsection{推导方法的说明}

我们也可以直接从轨迹方程出发,使用另一种方法:

将轨迹方程 \eqref{eq:trajectory} 整理为关于 $\tan\theta$ 的二次方程:
\begin{equation}
    \frac{gx^2}{2v_0^2}\tan^2\theta - x\tan\theta + \left(y + \frac{gx^2}{2v_0^2}\right) = 0
\end{equation}

对于给定的点 $(x, y)$,如果它位于某条轨迹上,则上述关于 $\tan\theta$ 的方程有实数解,判别式 $\Delta \geq 0$:

\begin{align}
    \Delta &= x^2 - 4 \cdot \frac{gx^2}{2v_0^2} \cdot \left(y + \frac{gx^2}{2v_0^2}\right) \nonumber\\
    &= x^2 - \frac{2gx^2}{v_0^2}\left(y + \frac{gx^2}{2v_0^2}\right) \nonumber\\
    &= x^2 - \frac{2gx^2y}{v_0^2} - \frac{g^2x^4}{v_0^4} \geq 0
\end{align}

整理得:
\begin{equation}
    y \leq \frac{v_0^2}{2g} - \frac{gx^2}{2v_0^2}
\end{equation}

等号成立时,点 $(x, y)$ 恰好在包络线上,这与 \eqref{eq:envelope} 一致。

% ============================================
% 5. 包络线的性质与应用
% ============================================
\section{包络线的性质与应用}

\subsection{包络线的几何性质}

\begin{center}
\begin{tikzpicture}[scale=2.0, x=2.5cm, y=6cm, >=Stealth]
    % 参数设置(先计算范围)
    \pgfmathsetmacro{\vzero}{3}
    \pgfmathsetmacro{\g}{9.8}
    \pgfmathsetmacro{\hmax}{\vzero*\vzero/(2*\g)}
    \pgfmathsetmacro{\xmax}{\vzero*\vzero/\g}
    
    % 坐标轴
    \draw[->, thick, black] (-1.1*\xmax,0) -- (1.1*\xmax,0) node[right, font=\normalsize] {$x$ (m)};
    \draw[->, thick, black] (0,-0.05) -- (0,1.1*\hmax) node[above, font=\normalsize] {$y$ (m)};
    \node[black] at (0,0) [below left, font=\normalsize] {$O$};
    
    % 网格线(辅助)
    \draw[gray!20, very thin] (-1.05*\xmax,0) grid (1.05*\xmax,1.05*\hmax);
    
    % 多条轨迹(较淡,作为背景)
    \foreach \angle in {5, 10, 15, 20, 25, 30, 35, 40, 45, 50, 55, 60, 65, 70, 75, 80, 85}
    {
        \pgfmathsetmacro{\tantheta}{tan(\angle)}
        \pgfmathsetmacro{\costheta}{cos(\angle)}
        \pgfmathsetmacro{\sintheta}{sin(\angle)}
        \pgfmathsetmacro{\xmaxtraj}{2*\vzero*\vzero*\sintheta*\costheta/\g}
        
        \draw[myblue!25, very thin, domain=0:\xmaxtraj, samples=60] 
            plot (\x, {\x*\tantheta - \g*\x*\x*(1+\tantheta*\tantheta)/(2*\vzero*\vzero)});
    }
    
    % 对称的轨迹(负x方向)
    \foreach \angle in {5, 10, 15, 20, 25, 30, 35, 40, 45, 50, 55, 60, 65, 70, 75, 80, 85}
    {
        \pgfmathsetmacro{\tantheta}{tan(\angle)}
        \pgfmathsetmacro{\costheta}{cos(\angle)}
        \pgfmathsetmacro{\sintheta}{sin(\angle)}
        \pgfmathsetmacro{\xmaxtraj}{2*\vzero*\vzero*\sintheta*\costheta/\g}
        
        \draw[myblue!25, very thin, domain=0:\xmaxtraj, samples=60] 
            plot (-\x, {\x*\tantheta - \g*\x*\x*(1+\tantheta*\tantheta)/(2*\vzero*\vzero)});
    }
    
    % 包络线(粗线,最上层)
    \draw[myred, very thick, line width=2.5pt, domain=-\xmax:\xmax, samples=150] 
        plot (\x, {\hmax - \g*\x*\x/(2*\vzero*\vzero)});
    
    % 抛射点
    \fill[myred] (0,0) circle (2pt);
    \node[myred, below left, font=\small] at (0,0) {抛射点};
    
    % 标注顶点
    \fill[myred] (0, \hmax) circle (2.5pt);
    \draw[dashed, myred!60, thick] (0, 0) -- (0, \hmax);
    \node[myred, left, font=\normalsize] at (0, 0.5*\hmax) {$\frac{v_0^2}{2g}$};
    \node[myred, above, font=\normalsize\bfseries] at (0, \hmax) {顶点};
    
    % 标注与x轴交点
    \fill[myred] (\xmax, 0) circle (2pt);
    \fill[myred] (-\xmax, 0) circle (2pt);
    \draw[dashed, myred!60, thick] (\xmax, 0) -- (\xmax, 0.08);
    \draw[dashed, myred!60, thick] (-\xmax, 0) -- (-\xmax, 0.08);
    \node[black, below, font=\normalsize] at (\xmax, 0) {$\frac{v_0^2}{g}$};
    \node[black, below, font=\normalsize] at (-\xmax, 0) {$-\frac{v_0^2}{g}$};
    
    % 标注
    \node[mygreen, right, font=\normalsize\bfseries] at (0.4*\xmax, 0.4*\hmax) {可达区域};
    \node[myred, above, font=\normalsize\bfseries] at (0, 0.8*\hmax) {包络线};
    
    % 添加对称轴标注
    \draw[<->, mypurple!60, thick] (-0.15*\xmax, 0.15*\hmax) -- (0.15*\xmax, 0.15*\hmax);
    \node[mypurple, above, font=\small] at (0, 0.15*\hmax) {对称轴};
\end{tikzpicture}
\captionof{figure}{包络线的几何性质与可达区域}
\end{center}

\begin{theorembox}{包络线的几何性质}
包络线 $y = \frac{v_0^2}{2g} - \frac{gx^2}{2v_0^2}$ 具有以下性质:

\begin{enumerate}
    \item \textbf{顶点}:$\left(0, \frac{v_0^2}{2g}\right)$
    \begin{itemize}
        \item 这是竖直上抛($\theta = 90°$)能达到的最大高度
        \item 也是所有轨迹中能达到的最高点
    \end{itemize}
    
    \item \textbf{对称轴}:$x = 0$($y$ 轴)
    \begin{itemize}
        \item 包络线关于 $y$ 轴对称
        \item 这反映了抛体运动的对称性
    \end{itemize}
    
    \item \textbf{与坐标轴的交点}:
    \begin{itemize}
        \item 与 $y$ 轴交点:$(0, \frac{v_0^2}{2g})$
        \item 与 $x$ 轴交点:$(\pm \frac{v_0^2}{g}, 0)$
        \item 水平距离 $\frac{v_0^2}{g}$ 是水平抛射($\theta = 0°$)的最大射程
    \end{itemize}
    
    \item \textbf{开口方向}:向下
    \begin{itemize}
        \item 包络线是开口向下的抛物线
        \item 包络线内部的区域是可达区域
    \end{itemize}
\end{enumerate}
\end{theorembox}

\subsection{物理意义}

包络线的物理意义非常重要:

\begin{itemize}
    \item \textbf{可达区域的边界}:包络线内部的区域表示在给定初速度 $v_0$ 下,质点能够到达的所有位置点
    \item \textbf{不可达区域}:包络线外部的区域表示无论以什么角度抛射,都无法到达
    \item \textbf{最优抛射角}:包络线上的每个点都对应一个特定的抛射角,这个角度是到达该点的唯一角度(或两个对称角度)
\end{itemize}

\begin{center}
\begin{tikzpicture}[scale=2.0, x=2.5cm, y=2.5cm, >=Stealth]
    % 参数设置
    \pgfmathsetmacro{\vlen}{1.4}
    \pgfmathsetmacro{\xmax}{2.0}
    \pgfmathsetmacro{\ymax}{2.0}
    
    % 坐标轴
    \draw[->, thick, black] (-0.1,0) -- (\xmax,0) node[right, font=\normalsize] {$x$};
    \draw[->, thick, black] (0,-0.1) -- (0,\ymax) node[above, font=\normalsize] {$y$};
    \node[black] at (0,0) [below left, font=\normalsize] {$O$};
    
    % 网格线(辅助)
    \draw[gray!20, very thin] (0,0) grid (\xmax,\ymax);
    
    % 抛射点
    \fill[myred] (0,0) circle (1.5pt);
    \node[myred, below left, font=\small] at (0,0) {抛射点};
    
    % 参数设置
    \pgfmathsetmacro{\vlen}{1.4}
    
    % 多个不同角度的速度矢量(较淡)
    \foreach \angle in {10, 20, 30, 40, 50, 60, 70, 80}
    {
        \pgfmathsetmacro{\vx}{\vlen*cos(\angle)}
        \pgfmathsetmacro{\vy}{\vlen*sin(\angle)}
        
        \draw[->, myblue!40, semithick] (0,0) -- (\vx, \vy);
    }
    
    % 特殊角度的速度矢量(加粗)
    \foreach \angle in {15, 30, 45, 60, 75}
    {
        \pgfmathsetmacro{\vx}{\vlen*cos(\angle)}
        \pgfmathsetmacro{\vy}{\vlen*sin(\angle)}
        
        \draw[->, myblue!70, thick] (0,0) -- (\vx, \vy);
        
        % 分解线(虚线)
        \draw[mygreen!40, thin, dashed] (0,0) -- (\vx, 0);
        \draw[myorange!40, thin, dashed] (\vx, 0) -- (\vx, \vy);
    }
    
    % 主要标注(45度角)
    \pgfmathsetmacro{\angle}{45}
    \pgfmathsetmacro{\vx}{\vlen*cos(\angle)}
    \pgfmathsetmacro{\vy}{\vlen*sin(\angle)}
    
    \draw[->, myred, very thick, line width=2.5pt] (0,0) -- (\vx, \vy);
    \node[myred, above left, font=\normalsize\bfseries] at (0.5*\vx, 0.5*\vy) {$v_0$};
    
    \draw[->, mygreen, thick, line width=2pt] (0,0) -- (\vx, 0);
    \node[mygreen, below, font=\normalsize] at (0.5*\vx, 0) {$v_0\cos\theta$};
    
    \draw[->, myorange, thick, line width=2pt] (\vx, 0) -- (\vx, \vy);
    \node[myorange, right, font=\normalsize] at (\vx, 0.5*\vy) {$v_0\sin\theta$};
    
    % 角度标注
    \draw[myorange, thick] (0.3, 0) arc (0:\angle:0.3);
    \node[myorange, font=\normalsize] at (0.45, 0.12) {$\theta$};
    
    % 图例
    \node[myblue, right, font=\normalsize] at (1.3, 1.8) {不同角度的初速度};
    \draw[->, mygreen, thick] (1.1, 1.5) -- (1.5, 1.5);
    \node[mygreen, right, font=\normalsize] at (1.5, 1.5) {水平分量};
    \draw[->, myorange, thick] (1.1, 1.2) -- (1.5, 1.2);
    \node[myorange, right, font=\normalsize] at (1.5, 1.2) {竖直分量};
    
    % 添加速度大小相等的说明(1/4扇形)
    \draw[myblue!60, dashed, thick] (0,0) -- (\vlen, 0) arc (0:90:\vlen) -- (0,0);
    \node[myblue, right, font=\small] at (1.0, 0.2) {所有速度大小相等};
\end{tikzpicture}
\captionof{figure}{不同抛射角的速度分解}
\end{center}

% ============================================
% 6. 例题与应用
% ============================================
\section{例题与应用}

\begin{examplebox}{例题1:求包络线方程}
\textbf{题目:}从原点以初速度 $v_0 = 20\text{ m/s}$ 抛出质点,重力加速度 $g = 10\text{ m/s}^2$。求所有可能轨迹的包络线方程,并确定可达区域。

\textbf{解:}

\textbf{步骤1:}写出轨迹族方程

设抛射角为 $\theta$,轨迹方程为:
\[ y = x\tan\theta - \frac{gx^2}{2v_0^2}(1 + \tan^2\theta) \]

代入 $v_0 = 20$,$g = 10$:
\[ y = x\tan\theta - \frac{10x^2}{2 \times 20^2}(1 + \tan^2\theta) = x\tan\theta - \frac{x^2}{80}(1 + \tan^2\theta) \]

\textbf{步骤2:}使用包络线求法

设 $k = \tan\theta$,则:
\[ F(x, y, k) = y - xk + \frac{x^2}{80}(1 + k^2) = 0 \]

对 $k$ 求偏导:
\[ \frac{\partial F}{\partial k} = -x + \frac{x^2k}{40} = 0 \]

当 $x \neq 0$ 时:
\[ k = \frac{40}{x} \]

\textbf{步骤3:}代入求包络线方程

\[ y = x \cdot \frac{40}{x} - \frac{x^2}{80}\left(1 + \left(\frac{40}{x}\right)^2\right) = 40 - \frac{x^2}{80} - \frac{1600}{80} = 40 - \frac{x^2}{80} - 20 = 20 - \frac{x^2}{80} \]

\textbf{答案:}包络线方程为 $y = 20 - \frac{x^2}{80}$。

\textbf{步骤4:}分析可达区域

\begin{itemize}
    \item 顶点:$(0, 20)$,最大高度为 $20$ 米
    \item 与 $x$ 轴交点:$x = \pm 40$,最大水平距离为 $40$ 米
    \item 可达区域:包络线下方(包括边界)的区域
\end{itemize}
\end{examplebox}

\begin{examplebox}{例题2:判断点是否可达}
\textbf{题目:}在例题1的条件下,判断点 $P(30, 10)$ 是否可达?如果可达,求到达该点的抛射角。

\textbf{解:}

\textbf{方法1:}判断点是否在包络线下方

包络线方程为 $y = 20 - \frac{x^2}{80}$。

当 $x = 30$ 时,包络线上的 $y$ 值为:
\[ y = 20 - \frac{30^2}{80} = 20 - \frac{900}{80} = 20 - 11.25 = 8.75 \]

点 $P(30, 10)$ 的 $y$ 坐标为 $10 > 8.75$,说明该点在包络线上方,\textbf{不可达}。

\textbf{方法2:}使用判别式法

将点 $P(30, 10)$ 代入轨迹方程:
\[ 10 = 30\tan\theta - \frac{30^2}{80}(1 + \tan^2\theta) \]

整理得:
\[ 10 = 30k - \frac{900}{80}(1 + k^2) = 30k - \frac{45}{4}(1 + k^2) \]

\[ 40 = 120k - 45(1 + k^2) = 120k - 45 - 45k^2 \]

\[ 45k^2 - 120k + 85 = 0 \]

判别式:
\[ \Delta = 120^2 - 4 \times 45 \times 85 = 14400 - 15300 = -900 < 0 \]

判别式小于零,说明不存在实数解,因此点 $P(30, 10)$ \textbf{不可达}。
\end{examplebox}

\begin{examplebox}{例题3:求到达包络线上某点的抛射角}
\textbf{题目:}在例题1的条件下,求到达包络线上点 $Q(20, 15)$ 的抛射角。

\textbf{解:}

\textbf{步骤1:}验证点在包络线上

当 $x = 20$ 时,包络线上的 $y$ 值为:
\[ y = 20 - \frac{20^2}{80} = 20 - 5 = 15 \]

点 $Q(20, 15)$ 确实在包络线上。

\textbf{步骤2:}求对应的 $k = \tan\theta$

由包络线的推导过程,$k = \frac{v_0^2}{gx} = \frac{40}{20} = 2$

\textbf{步骤3:}求抛射角

\[ \theta = \arctan(2) \approx 63.4° \]

\textbf{答案:}到达点 $Q(20, 15)$ 的抛射角约为 $63.4°$。

\textbf{验证:}将 $\theta = \arctan(2)$ 代入轨迹方程验证即可。
\end{examplebox}

\begin{notebox}{重要注意事项}
\begin{enumerate}
    \item \textbf{包络线的适用范围}:包络线方程是在理想条件下推导的(忽略空气阻力、假设重力加速度恒定等)。实际应用中需要考虑这些因素。
    
    \item \textbf{可达区域的判断}:判断点 $(x, y)$ 是否可达,只需检查是否满足 $y \leq \frac{v_0^2}{2g} - \frac{gx^2}{2v_0^2}$。
    
    \item \textbf{包络线上的点}:包络线上的每个点(除顶点外)通常对应两个对称的抛射角,一个在 $0°$ 到 $45°$ 之间,一个在 $45°$ 到 $90°$ 之间。
    
    \item \textbf{最大射程}:虽然单个轨迹的最大射程在 $\theta = 45°$ 时取得,但包络线与 $x$ 轴的交点对应的是水平抛射($\theta = 0°$)的射程。
    
    \item \textbf{单位一致性}:计算时注意物理量的单位要一致,通常使用国际单位制(SI)。
\end{enumerate}
\end{notebox}

% ============================================
% 总结
% ============================================
\section{总结}

本笔记介绍了固定速度抛射的包络线问题,主要内容包括:

\begin{itemize}
    \item \textbf{包络线的概念}:与一族曲线都相切的曲线,表示可达区域的边界
    \item \textbf{抛体运动基础}:参数方程、轨迹方程、射程和最大高度
    \item \textbf{轨迹族}:固定初速度、不同抛射角形成的抛物线族
    \item \textbf{包络线方程}:$y = \frac{v_0^2}{2g} - \frac{gx^2}{2v_0^2}$
    \item \textbf{几何性质}:顶点、对称轴、与坐标轴的交点
    \item \textbf{应用}:判断点是否可达、求最优抛射角等
\end{itemize}

包络线问题将数学中的曲线族理论与物理中的抛体运动相结合,是数学物理交叉应用的典型例子。掌握包络线的求法和性质,有助于深入理解抛体运动的本质特征。

% ============================================
% 文档结束
% ============================================
\end{document}
