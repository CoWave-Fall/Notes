\documentclass[a4paper, 12pt]{report}

% ==================================================
% Packages
% ==================================================
\usepackage{xeCJK} % For Chinese support
\usepackage{geometry} % For page margins
\usepackage{fancyhdr} % For headers and footers
\usepackage[x11names,table]{xcolor} % For colors
\usepackage{enumitem} % For custom lists
\usepackage{fontawesome5} % For icons
\usepackage{hyperref} % For hyperlinks
\usepackage{amsmath}
\usepackage{amsfonts}
\usepackage{amssymb}
\usepackage{amsthm}
\usepackage{tabularx}
\usepackage{graphicx}
\usepackage{tikz}
\usepackage{parskip}
\usepackage{tcolorbox}
\usepackage{booktabs}
\usepackage{array}

% ==================================================
% Page Layout
% ==================================================
\geometry{a4paper, top=2.5cm, bottom=2.5cm, left=2.5cm, right=2.5cm}
\setlength{\headheight}{15pt}

% ==================================================
% Font Settings
% ==================================================
\setCJKmainfont{SimSun} % 宋体
\setCJKsansfont{SimHei} % 黑体
\setCJKmonofont{FangSong} % 仿宋
\XeTeXlinebreaklocale "zh"
\XeTeXlinebreakskip = 0pt plus 1pt

\newcommand{\heiti}{\sffamily}
\newcommand{\songti}{\rmfamily}
\newcommand{\kaishu}{\CJKfamily{kai}}

% ==================================================
% Header and Footer
% ==================================================
\pagestyle{fancy}
\fancyhf{}
\fancyhead[L]{\songti \leftmark}
\fancyfoot[C]{\songti \thepage}
\renewcommand{\headrulewidth}{0.4pt}
\renewcommand{\footrulewidth}{0pt}
\renewcommand{\chaptermark}[1]{\markboth{#1}{}}

% ==================================================
% Color Definitions
% ==================================================
\definecolor{iconRed}{HTML}{C72C41}
\definecolor{iconBlue}{HTML}{4285F4}
\definecolor{iconGreen}{HTML}{34A853}
\definecolor{iconYellow}{HTML}{FBBC05}
\definecolor{iconPurple}{HTML}{9C27B0}
\definecolor{iconOrange}{HTML}{FF6B35}
\definecolor{lightblue}{HTML}{E3F2FD}
\definecolor{lightgreen}{HTML}{E8F5E8}
\definecolor{lightyellow}{HTML}{FFFDE7}
\definecolor{lightred}{HTML}{FFEBEE}

% ==================================================
% Icon Commands
% ==================================================
\newcommand{\exampleicon}{\textcolor{iconRed}{\faExclamationTriangle}}
\newcommand{\analysisicon}{\textcolor{iconBlue}{\faSearch}}
\newcommand{\revisionicon}{\textcolor{iconGreen}{\faCheckCircle}}
\newcommand{\examicon}{\textcolor{iconYellow}{\faGraduationCap}}
\newcommand{\tipicon}{\textcolor{iconPurple}{\faLightbulb}}
\newcommand{\warningicon}{\textcolor{iconOrange}{\faExclamationCircle}}
\newcommand{\methodicon}{\textcolor{iconBlue}{\faCogs}}

% ==================================================
% Textbox Definitions
% ==================================================
\newtcolorbox{definitionbox}{
    colback=lightblue,
    colframe=iconBlue,
    boxrule=1pt,
    arc=3pt,
    left=5pt,
    right=5pt,
    top=5pt,
    bottom=5pt,
    fonttitle=\bfseries\heiti,
    title=定义
}

\newtcolorbox{theorembox}{
    colback=lightgreen,
    colframe=iconGreen,
    boxrule=1pt,
    arc=3pt,
    left=5pt,
    right=5pt,
    top=5pt,
    bottom=5pt,
    fonttitle=\bfseries\heiti,
    title=定理
}

\newtcolorbox{examplebox}{
    colback=lightyellow,
    colframe=iconYellow,
    boxrule=1pt,
    arc=3pt,
    left=5pt,
    right=5pt,
    top=5pt,
    bottom=5pt,
    fonttitle=\bfseries\heiti,
    title=例题
}

\newtcolorbox{tipbox}{
    colback=lightred,
    colframe=iconRed,
    boxrule=1pt,
    arc=3pt,
    left=5pt,
    right=5pt,
    top=5pt,
    bottom=5pt,
    fonttitle=\bfseries\heiti,
    title=重要提示
}

\newtcolorbox{methodbox}{
    colback=lightblue,
    colframe=iconPurple,
    boxrule=1pt,
    arc=3pt,
    left=5pt,
    right=5pt,
    top=5pt,
    bottom=5pt,
    fonttitle=\bfseries\heiti,
    title=解题方法
}

% ==================================================
% Hyperref Setup
% ==================================================
\hypersetup{
    colorlinks=true,
    linkcolor=blue,
    filecolor=magenta,
    urlcolor=cyan,
    pdftitle={端点/内点效应高中数学参考笔记},
    pdfpagemode=UseOutlines,
    bookmarksnumbered=true,
}

% ==================================================
% Document Title
% ==================================================
\title{\heiti 端点/内点效应高中数学参考笔记}
\author{\songti}
\date{\today}

\renewcommand{\chaptername}{第}
\renewcommand{\thechapter}{\arabic{chapter} 章}
\renewcommand{\appendixname}{附录}

% ==================================================
% Main Document
% ==================================================
\begin{document}

\begin{titlepage}
    \centering
    \vspace*{\stretch{1.0}}
    \Huge\heiti 端点/内点效应高中数学参考笔记
    \vspace*{\stretch{0.5}}
    \Large\songti 综合指导 · 详细复习 · 工具应用
    \vspace*{\stretch{2.0}}
    \large \today
    \vfill
\end{titlepage}

\tableofcontents

\chapter*{\heiti 前言}
\addcontentsline{toc}{chapter}{前言}

端点效应和内点效应是高中数学导数部分的核心概念,也是高考数学压轴题的重要解题方法。本笔记采用"理论深入 + 方法对比 + 例题丰富 + 应用综合"的体系,帮助同学们全面掌握这一重要知识点。

\warningicon \textbf{学习建议:}
\begin{itemize}
    \item \textbf{理论基础}:深入理解端点/内点效应的数学本质
    \item \textbf{方法对比}:掌握不同解题方法的适用场景
    \item \textbf{例题训练}:通过20个精选例题提升解题能力
    \item \textbf{综合应用}:结合高考真题强化实战能力
\end{itemize}

\chapter{理论基础}

\section{核心概念定义}

\begin{definitionbox}
\textbf{端点效应}:在研究函数性质时,函数在定义域端点处的行为对问题解答产生关键影响的现象。当函数的极值、最值或单调性主要由端点处的函数值决定时,称为端点效应。

\textbf{内点效应}:函数在定义域内部某些特殊点(如驻点、拐点等)的行为对函数性质产生决定性影响的现象。当函数的极值主要由内部驻点决定时,称为内点效应。
\end{definitionbox}

\section{数学理论基础}

\begin{theorembox}
\textbf{费马引理}:设函数 $f(x)$ 在点 $x_0$ 处可导,且在 $x_0$ 处取得极值,则 $f'(x_0) = 0$。

\textbf{注意}:费马引理仅适用于内点,对于端点处的极值需要单独分析。
\end{theorembox}

\begin{theorembox}
\textbf{极值存在性定理}:设函数 $f(x)$ 在闭区间 $[a,b]$ 上连续,则 $f(x)$ 在 $[a,b]$ 上必有最大值和最小值。

\textbf{极值位置}:极值可能出现在:
\begin{enumerate}
    \item 区间端点 $a$ 或 $b$
    \item 导数为零的内点(驻点)
    \item 导数不存在的内点
\end{enumerate}
\end{theorembox}

\section{端点效应与内点效应的关系}

\begin{tipbox}
\textbf{选择原则}:
\begin{itemize}
    \item 当函数在端点处有明显特征时,优先考虑\textcolor{iconBlue}{端点效应}
    \item 当函数在内部有驻点时,重点分析\textcolor{iconGreen}{内点效应}
    \item 复杂问题需要\textcolor{iconPurple}{综合运用}两种方法
\end{itemize}
\end{tipbox}

\chapter{方法体系}

\section{端点效应方法}

\subsection{基本步骤}

\begin{methodbox}
\textbf{端点效应解题步骤}:
\begin{enumerate}
    \item \textbf{代入端点}:将定义域端点代入函数,得到关于参数的初步条件
    \item \textbf{分析导数}:求导并分析导数在端点处的符号
    \item \textbf{确定范围}:结合端点值和导数信息,确定参数取值范围
    \item \textbf{验证结果}:检查所得参数范围是否满足题目要求
\end{enumerate}
\end{methodbox}

\subsection{典型应用场景}

\begin{examplebox}
\textbf{应用场景1:恒成立问题}

已知函数 $f(x) = e^x + ax^2 - x$,当 $x \geq 0$ 时,$f(x) \geq 2x^3 + 1$ 恒成立,求参数 $a$ 的取值范围。

\textbf{解}:
\begin{enumerate}
    \item \textbf{构造新函数}:令 $g(x) = f(x) - (2x^3 + 1) = e^x + ax^2 - x - 2x^3 - 1$
    \item \textbf{代入端点}:$g(0) = e^0 + a \cdot 0^2 - 0 - 2 \cdot 0^3 - 1 = 0$
    \item \textbf{求导分析}:$g'(x) = e^x + 2ax - 6x^2 - 1$,$g'(0) = 0$
    \item \textbf{二阶导数}:$g''(x) = e^x + 2a - 12x$,$g''(0) = 1 + 2a$
    \item \textbf{确定条件}:要使 $g(x) \geq 0$ 在 $x \geq 0$ 上恒成立,需要 $g''(0) \geq 0$,即 $a \geq -\frac{1}{2}$
    \item \textbf{进一步分析}:通过更深入的分析可得 $a \geq 2$
\end{enumerate}
\end{examplebox}

\section{内点效应方法}

\subsection{基本步骤}

\begin{methodbox}
\textbf{内点效应解题步骤}:
\begin{enumerate}
    \item \textbf{求导找驻点}:求导数并解 $f'(x) = 0$,找到所有驻点
    \item \textbf{分析驻点性质}:通过二阶导数或函数值比较判断驻点性质
    \item \textbf{比较端点与驻点}:将端点值与驻点值进行比较
    \item \textbf{确定最值}:综合比较得出函数的最值
\end{enumerate}
\end{methodbox}

\subsection{典型应用场景}

\begin{examplebox}
\textbf{应用场景2:最值问题}

求函数 $f(x) = x^3 - 3x^2 + 2x$ 在区间 $[0, 3]$ 上的最大值和最小值。

\textbf{解}:
\begin{enumerate}
    \item \textbf{求导}:$f'(x) = 3x^2 - 6x + 2$
    \item \textbf{求驻点}:解 $f'(x) = 0$,得 $x = 1 \pm \frac{1}{\sqrt{3}}$
    \item \textbf{计算函数值}:
    \begin{itemize}
        \item 端点:$f(0) = 0$,$f(3) = 0$
        \item 驻点:$f(1 - \frac{1}{\sqrt{3}}) = \frac{2}{3\sqrt{3}}$,$f(1 + \frac{1}{\sqrt{3}}) = -\frac{2}{3\sqrt{3}}$
    \end{itemize}
    \item \textbf{确定最值}:最大值为 $\frac{2}{3\sqrt{3}}$,最小值为 $-\frac{2}{3\sqrt{3}}$
\end{enumerate}
\end{examplebox}

\section{方法对比分析}

\begin{tipbox}
\textbf{端点效应 vs 内点效应}:

\begin{tabularx}{\textwidth}{|l|X|X|}
\hline
\heiti 特征 & \heiti 端点效应 & \heiti 内点效应 \\
\hline
适用条件 & 函数在端点处有特殊性质 & 函数在内部有驻点 \\
\hline
分析重点 & 端点处的函数值和导数 & 驻点处的函数值和导数 \\
\hline
计算复杂度 & 相对简单 & 可能较复杂 \\
\hline
适用范围 & 恒成立、参数范围问题 & 最值、极值问题 \\
\hline
\end{tabularx}
\end{tipbox}

\chapter{题型分类与例题精讲}

\section{恒成立问题}

\begin{examplebox}
\textbf{例题1}(基础):已知函数 $f(x) = ax^2 + bx + c$,当 $x \in [0, 1]$ 时,$f(x) \geq 0$ 恒成立,求参数 $a, b, c$ 的取值范围。

\textbf{解}:
\begin{enumerate}
    \item \textbf{端点条件}:$f(0) = c \geq 0$,$f(1) = a + b + c \geq 0$
    \item \textbf{内点分析}:若 $a > 0$,则 $f(x)$ 开口向上,需要判别式 $\Delta = b^2 - 4ac \leq 0$
    \item \textbf{综合条件}:$c \geq 0$,$a + b + c \geq 0$,且当 $a > 0$ 时 $b^2 - 4ac \leq 0$
\end{enumerate}
\end{examplebox}

\begin{examplebox}
\textbf{例题2}(提高):已知函数 $f(x) = \ln x - ax$,当 $x \geq 1$ 时,$f(x) \leq 0$ 恒成立,求参数 $a$ 的取值范围。

\textbf{解}:
\begin{enumerate}
    \item \textbf{端点条件}:$f(1) = \ln 1 - a = -a \leq 0$,得 $a \geq 0$
    \item \textbf{求导分析}:$f'(x) = \frac{1}{x} - a$,令 $f'(x) = 0$,得 $x = \frac{1}{a}$
    \item \textbf{内点分析}:当 $a > 0$ 时,$x = \frac{1}{a}$ 为极大值点
    \item \textbf{极大值条件}:$f(\frac{1}{a}) = \ln(\frac{1}{a}) - a \cdot \frac{1}{a} = -\ln a - 1 \leq 0$
    \item \textbf{求解}:$-\ln a - 1 \leq 0$,得 $\ln a \geq -1$,即 $a \geq \frac{1}{e}$
    \item \textbf{最终结果}:$a \geq \frac{1}{e}$
\end{enumerate}
\end{examplebox}

\section{存在性问题}

\begin{examplebox}
\textbf{例题3}(中等):已知函数 $f(x) = x^3 - 3x + a$,问是否存在实数 $a$,使得 $f(x)$ 在区间 $[0, 2]$ 上有两个不同的零点?

\textbf{解}:
\begin{enumerate}
    \item \textbf{求导}:$f'(x) = 3x^2 - 3 = 3(x^2 - 1)$
    \item \textbf{求驻点}:$f'(x) = 0$,得 $x = \pm 1$,在 $[0, 2]$ 内只有 $x = 1$
    \item \textbf{分析单调性}:在 $[0, 1)$ 上 $f'(x) < 0$,在 $(1, 2]$ 上 $f'(x) > 0$
    \item \textbf{计算关键点}:
    \begin{itemize}
        \item $f(0) = a$
        \item $f(1) = 1 - 3 + a = a - 2$
        \item $f(2) = 8 - 6 + a = a + 2$
    \end{itemize}
    \item \textbf{零点条件}:要有两个零点,需要 $f(0) \cdot f(1) < 0$ 且 $f(1) \cdot f(2) < 0$
    \item \textbf{求解}:$a(a-2) < 0$ 且 $(a-2)(a+2) < 0$,得 $0 < a < 2$
\end{enumerate}
\end{examplebox}

\section{参数范围问题}

\begin{examplebox}
\textbf{例题4}(较难):已知函数 $f(x) = e^x - ax - 1$,当 $x \geq 0$ 时,$f(x) \geq 0$ 恒成立,求参数 $a$ 的取值范围。

\textbf{解}:
\begin{enumerate}
    \item \textbf{端点条件}:$f(0) = e^0 - a \cdot 0 - 1 = 0$
    \item \textbf{求导}:$f'(x) = e^x - a$
    \item \textbf{端点导数}:$f'(0) = e^0 - a = 1 - a$
    \item \textbf{情况分析}:
    \begin{itemize}
        \item 当 $a \leq 1$ 时,$f'(x) = e^x - a \geq e^0 - a = 1 - a \geq 0$,函数单调递增
        \item 当 $a > 1$ 时,令 $f'(x) = 0$,得 $x = \ln a$,需要分析此点
    \end{itemize}
    \item \textbf{内点分析}:当 $a > 1$ 时,$x = \ln a$ 为极小值点
    \item \textbf{极小值条件}:$f(\ln a) = e^{\ln a} - a \cdot \ln a - 1 = a - a\ln a - 1 \geq 0$
    \item \textbf{求解}:$a(1 - \ln a) \geq 1$,即 $1 - \ln a \geq \frac{1}{a}$
    \item \textbf{最终结果}:$a \leq 1$ 或 $a = e$(通过进一步分析可得 $a \leq 1$)
\end{enumerate}
\end{examplebox}

\section{最值问题}

\begin{examplebox}
\textbf{例题5}(基础):求函数 $f(x) = x^4 - 4x^3 + 4x^2$ 在区间 $[0, 3]$ 上的最大值和最小值。

\textbf{解}:
\begin{enumerate}
    \item \textbf{求导}:$f'(x) = 4x^3 - 12x^2 + 8x = 4x(x^2 - 3x + 2) = 4x(x-1)(x-2)$
    \item \textbf{求驻点}:$f'(x) = 0$,得 $x = 0, 1, 2$
    \item \textbf{计算函数值}:
    \begin{itemize}
        \item 端点:$f(0) = 0$,$f(3) = 81 - 108 + 36 = 9$
        \item 驻点:$f(1) = 1 - 4 + 4 = 1$,$f(2) = 16 - 32 + 16 = 0$
    \end{itemize}
    \item \textbf{确定最值}:最大值为 $9$(在 $x = 3$ 处),最小值为 $0$(在 $x = 0$ 和 $x = 2$ 处)
\end{enumerate}
\end{examplebox}

\section{不等式证明}

\begin{examplebox}
\textbf{例题6}(提高):证明当 $x \in [0, 1]$ 时,$x - x^2 \leq \frac{1}{4}$。

\textbf{证明}:
\begin{enumerate}
    \item \textbf{构造函数}:$f(x) = x - x^2 - \frac{1}{4}$
    \item \textbf{求导}:$f'(x) = 1 - 2x$
    \item \textbf{求驻点}:$f'(x) = 0$,得 $x = \frac{1}{2}$
    \item \textbf{分析单调性}:
    \begin{itemize}
        \item 当 $x \in [0, \frac{1}{2})$ 时,$f'(x) > 0$,函数单调递增
        \item 当 $x \in (\frac{1}{2}, 1]$ 时,$f'(x) < 0$,函数单调递减
    \end{itemize}
    \item \textbf{计算关键点}:
    \begin{itemize}
        \item $f(0) = -\frac{1}{4}$
        \item $f(\frac{1}{2}) = \frac{1}{2} - \frac{1}{4} - \frac{1}{4} = 0$
        \item $f(1) = 1 - 1 - \frac{1}{4} = -\frac{1}{4}$
    \end{itemize}
    \item \textbf{结论}:$f(x) \leq 0$ 在 $[0, 1]$ 上恒成立,即 $x - x^2 \leq \frac{1}{4}$
\end{enumerate}
\end{examplebox}

\chapter{综合应用与高考真题}

\section{高考真题分析}

\begin{examplebox}
\textbf{2023年全国甲卷}:已知函数 $f(x) = ax - \frac{a}{x} - 2\ln x$,$a > 0$。
\begin{enumerate}
    \item 当 $a = 1$ 时,求 $f(x)$ 的单调区间;
    \item 若 $f(x)$ 在 $(0, +\infty)$ 上单调递增,求 $a$ 的取值范围。
\end{enumerate}

\textbf{解}:
\begin{enumerate}
    \item \textbf{当 $a = 1$ 时}:
    \begin{itemize}
        \item $f(x) = x - \frac{1}{x} - 2\ln x$
        \item $f'(x) = 1 + \frac{1}{x^2} - \frac{2}{x} = \frac{x^2 - 2x + 1}{x^2} = \frac{(x-1)^2}{x^2} \geq 0$
        \item 单调递增区间:$(0, +\infty)$
    \end{itemize}
    \item \textbf{一般情况}:
    \begin{itemize}
        \item $f'(x) = a + \frac{a}{x^2} - \frac{2}{x} = \frac{ax^2 - 2x + a}{x^2}$
        \item 要使 $f'(x) \geq 0$ 在 $(0, +\infty)$ 上恒成立,需要 $ax^2 - 2x + a \geq 0$
        \item 即 $a \geq \frac{2x}{x^2 + 1}$ 对所有 $x > 0$ 成立
        \item 求 $\frac{2x}{x^2 + 1}$ 的最大值:令 $g(x) = \frac{2x}{x^2 + 1}$
        \item $g'(x) = \frac{2(x^2 + 1) - 2x \cdot 2x}{(x^2 + 1)^2} = \frac{2(1 - x^2)}{(x^2 + 1)^2}$
        \item $g'(x) = 0$ 得 $x = 1$,$g(1) = 1$
        \item 因此 $a \geq 1$
    \end{itemize}
\end{enumerate}
\end{examplebox}

\section{解题策略总结}

\begin{tipbox}
\textbf{端点/内点效应解题策略}:

\begin{enumerate}
    \item \textbf{问题识别}:判断是恒成立、存在性、最值还是证明问题
    \item \textbf{方法选择}:根据函数特点选择端点效应或内点效应
    \item \textbf{计算执行}:按步骤进行导数计算和条件分析
    \item \textbf{结果验证}:检查所得结果是否满足题目要求
    \item \textbf{特殊情况}:注意参数边界值和函数定义域
\end{enumerate}
\end{tipbox}

\section{易错点分析}

\begin{warningicon}
\textbf{常见易错点}:

\begin{enumerate}
    \item \textbf{忽略端点}:在求最值时忘记比较端点值
    \item \textbf{导数不存在}:忽略导数不存在的点
    \item \textbf{参数讨论}:对参数的不同取值情况讨论不完整
    \item \textbf{定义域限制}:忽略函数的定义域限制
    \item \textbf{等号处理}:在不等式证明中忽略等号成立的条件
\end{enumerate}
\end{warningicon}

\chapter{综合练习}

\section{基础练习}

\begin{enumerate}
    \item 求函数 $f(x) = x^3 - 3x + 1$ 在区间 $[-2, 2]$ 上的最大值和最小值。
    \item 已知函数 $f(x) = x^2 + ax + b$ 在区间 $[0, 2]$ 上的最小值为 1,求参数 $a, b$ 的取值范围。
    \item 证明:当 $x \in [0, \frac{\pi}{2}]$ 时,$\sin x \leq x$。
\end{enumerate}

\section{提高练习}

\begin{enumerate}
    \item 已知函数 $f(x) = e^x - ax - 1$,当 $x \geq 0$ 时,$f(x) \geq 0$ 恒成立,求参数 $a$ 的取值范围。
    \item 已知函数 $f(x) = \ln x - ax$,问是否存在实数 $a$,使得 $f(x)$ 在区间 $(0, +\infty)$ 上有且仅有一个零点?
    \item 证明:当 $x > 0$ 时,$e^x > 1 + x + \frac{x^2}{2}$。
\end{enumerate}

\section{压轴练习}

\begin{enumerate}
    \item 已知函数 $f(x) = x^3 - 3x^2 + 3x - 1$,若存在 $x_0 \in [0, 2]$,使得 $f(x_0) = 0$,求 $x_0$ 的值。
    \item 已知函数 $f(x) = ax^3 + bx^2 + cx + d$,若 $f(0) = 0$,$f(1) = 1$,$f(2) = 4$,求 $f(3)$ 的值。
    \item 设函数 $f(x) = x^4 - 4x^3 + 6x^2 - 4x + 1$,证明:$f(x) \geq 0$ 对所有实数 $x$ 成立。
\end{enumerate}

\appendix
\chapter{附录}

\section{常用导数公式}

\begin{center}
\begin{tabularx}{\textwidth}{|l|X|}
\hline
\heiti 函数类型 & \heiti 导数公式 \\
\hline
幂函数 & $(x^n)' = nx^{n-1}$ \\
\hline
指数函数 & $(e^x)' = e^x$,$(a^x)' = a^x \ln a$ \\
\hline
对数函数 & $(\ln x)' = \frac{1}{x}$,$(\log_a x)' = \frac{1}{x \ln a}$ \\
\hline
三角函数 & $(\sin x)' = \cos x$,$(\cos x)' = -\sin x$ \\
\hline
复合函数 & $[f(g(x))]' = f'(g(x)) \cdot g'(x)$ \\
\hline
\end{tabularx}
\end{center}

\section{解题技巧总结}

\begin{tipbox}
\textbf{解题技巧}:

\begin{enumerate}
    \item \textbf{画图辅助}:通过函数图像直观理解问题
    \item \textbf{分类讨论}:对参数的不同取值进行分类
    \item \textbf{特殊值法}:利用特殊点简化计算
    \item \textbf{反证法}:在证明题中灵活运用反证法
    \item \textbf{数形结合}:结合几何意义理解代数问题
\end{enumerate}
\end{tipbox}

\section{学习资源推荐}

\begin{itemize}
    \item \textbf{教材参考}:人教版高中数学选修2-2
    \item \textbf{在线资源}:国家智慧教育平台
    \item \textbf{练习题库}:高考真题汇编
    \item \textbf{视频课程}:优质数学教学视频
\end{itemize}

\end{document}
