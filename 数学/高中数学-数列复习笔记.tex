% --------------------------------------------------
% 高中数学·数列 专题复习笔记
% 依托 mathnote-preamble.tex 与“数学笔记模板”排版风格
% --------------------------------------------------
\documentclass[a4paper,11pt]{ctexart}
\InputIfFileExists{mathnote-preamble.tex}{}{%
  \InputIfFileExists{../mathnote-preamble.tex}{}{%
    \InputIfFileExists{../../mathnote-preamble.tex}{}{%
      \PackageError{mathnote-notes}{mathnote-preamble.tex 未找到}{请确认 mathnote-preamble.tex 与当前文稿的相对路径配置正确。}%
    }%
  }%
}

% 如以打印版为主,可保留;若以屏幕阅读为主,可注释掉
\MathNoteEnablePrint
% 如需右上角审核标记,可启用
\MathNoteEnableReviewStamp

% 印刷模式下略微压缩垂直空白,适配纸质阅读
\ifmathnoteprintmode
  \setlength{\parskip}{0.25em}
  \ctexset{
    section={
      beforeskip=1.0em,
      afterskip=0.5em
    },
    subsection={
      beforeskip=0.8em,
      afterskip=0.35em
    },
    subsubsection={
      beforeskip=0.6em,
      afterskip=0.2em
    }
  }
  \tcbset{
    mathnote box/.style/.append style={
      top=0.6em,
      bottom=0.6em,
      before skip=8pt,
      after skip=8pt
    }
  }
\fi

\renewcommand{\notetitle}{高中数学数列专题复习}
\renewcommand{\noteauthor}{生成式人工智能,人工审阅通过}
\renewcommand{\notedate}{\today}
\renewcommand{\notesubtitle}{通项与求和模型\quad 典型题型全景梳理}
\renewcommand{\noteversion}{v1.0}

% 本专题中一般列表采用普通样式,避免过多色块
\setlist[itemize]{leftmargin=1.8em, itemsep=0.25em, before=\relax, after=\relax}
\setlist[enumerate]{leftmargin=2.1em, itemsep=0.3em, before=\relax, after=\relax}

\newcommand{\SeqMiniHeading}[1]{%
  \par\smallskip
  \noindent\textbf{#1}\par\smallskip
}

\begin{document}

% 封面
\thispagestyle{empty}
\PageTag[secondary]{高中数学·数列}
\setcounter{page}{1}

\noindent
\begin{minipage}[t][0.7\textheight][c]{\textwidth}
  \raggedleft
  \vspace*{\fill}
  {\sffamily\bfseries\Huge\color{accent}\notetitle\par}
  \vspace{0.6em}
  {\Large\color{inkgray}\notesubtitle\par}
  \vspace{0.5em}
  {\large\color{inkgray}\noteauthor\par}
  \vspace{0.4em}
  {\normalsize\color{inkgray}\noteversion\quad|\quad\notedate\par}
  \vspace*{\fill}
\end{minipage}

\begin{center}
\begin{minipage}{0.84\textwidth}
  \textcolor{inkgray}{\sffamily\small 复习导航}\\
  \begin{itemize}
    \item 以 \keyword{知识网络} + \keyword{典型模型} + \keyword{高考风格例题} 的结构构成一张可反复翻阅的数列「工具地图」;
    \item 贯穿「\inlinehint{通项}–\inlinehint{求和}–\inlinehint{不等式}–\inlinehint{归纳法}」主线,适当衔接高等数学中的极限与级数思想;
    \item 每一类模型配套「\ModeBadge{常见陷阱}」与「\ModeBadge[highlight]{解题套路}」,方便考前快速回忆。
  \end{itemize}
\end{minipage}
\end{center}
\vspace{1em}

\clearpage
\thispagestyle{plain}
\phantomsection
\tableofcontents

\clearpage
% === Part 1: 数列基础与整体认知 ===
\section{数列基础与整体认知}

\SeqMiniHeading{数列的基本概念}
数列是按一定\keyword{次序规则}排列的一列数,通常记为 $\{a_n\}$ 或 $\{a_n\}_{n=1}^{\infty}$。
\begin{itemize}
  \item \keyword{通项公式}:给出 $a_n$ 关于 $n$ 的显式表示,如 $a_n=f(n)$;
  \item \keyword{递推关系}:用前若干项表示后面的项,如 $a_{n+1}=g(a_n)$;
  \item \keyword{前 $n$ 项和}:$S_n = a_1 + a_2 + \cdots + a_n$,常写为 $\{S_n\}$。
\end{itemize}

\begin{definitionbox}{单调、有界与收敛}
  设数列 $\{a_n\}$:
  \begin{itemize}
    \item 若对任意 $n$,都有 $a_{n+1}\ge a_n$(或 $\le$),则称其为\keyword{单调递增}(或单调递减);
    \item 若存在实数 $m,M$,使得对所有 $n$ 都有 $m\le a_n\le M$,称 $\{a_n\}$ \keyword{有界};
    \item 若存在实数 $L$,使得当 $n\to\infty$ 时 $a_n\to L$,则称 $\{a_n\}$ \keyword{收敛}到 $L$。
  \end{itemize}
  在高等数学中,数列极限是\keyword{函数极限}与\keyword{级数收敛性}的基础。
\end{definitionbox}

\begin{summarybox}{考试中的常见数列类型}
  \begin{multicols}{2}
    \begin{itemize}
      \item 等差数列与其变式(分段、绝对值);
      \item 等比数列与其变式(交错、绝对值);
      \item 线性递推数列(特征根法);
      \item 斐波那契及其推广;
      \item 含奇偶项的分段递推数列;
      \item 与函数、导数、极限相关的构造数列。
    \end{itemize}
  \end{multicols}
\end{summarybox}

\SeqMiniHeading{与高等数学的联系}
\begin{itemize}
  \item 数列 $\{a_n\}$ 的极限 $\lim_{n\to\infty} a_n$ 是极限与连续性课程中的基本对象;
  \item 求和 $\sum_{k=1}^n a_k$ 在高等数学中推广为\keyword{无穷级数} $\sum_{k=1}^{\infty} a_k$;
  \item 一些求和模型(如裂项相消、错位相减)会自然出现在幂级数与泰勒展开的推导中。
\end{itemize}

\clearpage
% === Part 2: 等差与等比数列基础 ===
\section{等差与等比数列基础}

\subsection{等差数列}
\begin{definitionbox}{等差数列}
  若数列 $\{a_n\}$ 满足
  \[
    a_{n+1}-a_n = d\quad(\text{常数})
  \]
  则称其为\keyword{等差数列},$d$ 为\keyword{公差}。
  \begin{itemize}
    \item 通项:$a_n = a_1 + (n-1)d$;
    \item 前 $n$ 项和:$S_n = \dfrac{n(a_1+a_n)}{2}$ 或 $S_n = \dfrac{n(2a_1+(n-1)d)}{2}$。
  \end{itemize}
\end{definitionbox}

\begin{examplebox}{典型题·等差数列基础}
  已知等差数列 $\{a_n\}$ 满足 $a_1=3,a_5=11$。
  \begin{enumerate}
    \item 求公差 $d$ 与通项公式 $a_n$;
    \item 求 $S_{10}$。
  \end{enumerate}
  \textbf{解:}
  \begin{enumerate}
    \item 由 $a_5=a_1+4d$ 得 $11=3+4d$,解得 $d=2$,从而 $a_n=3+2(n-1)=2n+1$;
    \item $S_{10}=\dfrac{10(a_1+a_{10})}{2}=\dfrac{10(3+21)}{2}=120$。
  \end{enumerate}
\end{examplebox}

\subsection{等比数列}
\begin{definitionbox}{等比数列}
  若数列 $\{b_n\}$ 满足
  \[
    \frac{b_{n+1}}{b_n} = q\quad(q\ne0)
  \]
  则称其为\keyword{等比数列},$q$ 为\keyword{公比}。
  \begin{itemize}
    \item 通项:$b_n = b_1 q^{n-1}$;
    \item 前 $n$ 项和($q\ne1$):
    \[
      S_n = b_1\frac{1-q^{n}}{1-q}.
    \]
  \end{itemize}
\end{definitionbox}

\begin{examplebox}{典型题·等比数列基础}
  已知等比数列 $\{b_n\}$ 满足 $b_1=2,b_3=8$。
  \begin{enumerate}
    \item 求公比 $q$ 与通项 $b_n$;
    \item 求 $\displaystyle S_n=\sum_{k=1}^n b_k$。
  \end{enumerate}
  \textbf{解:}
  \begin{enumerate}
    \item $b_3=b_1q^2\Rightarrow 8=2q^2\Rightarrow q^2=4$。若题目未做额外约束,一般取 $q>0$,故 $q=2$,$b_n=2\cdot2^{n-1}=2^n$;
    \item $S_n=2\cdot\dfrac{1-2^{n}}{1-2}=2(2^{n}-1)$。
  \end{enumerate}
\end{examplebox}

\begin{summarybox}{等差/等比数列对照}
  \begin{center}
  \begin{tabularx}{0.98\textwidth}{>{\raggedright\arraybackslash}p{2.4cm}X}
    \toprule
    \textbf{类型} & \textbf{核心特征与常用技巧} \\
    \midrule
    等差数列 &
      $a_{n+1}-a_n=d$,常用\keyword{差分}与\keyword{线性表示},
      通常与\keyword{一次函数}、\keyword{线性不等式}建模联系密切。 \\[0.3em]
    等比数列 &
      $\dfrac{b_{n+1}}{b_n}=q$,常用\keyword{比值}与\keyword{对数}思想,
      在高等数学中,对应于\keyword{指数函数}与\keyword{指数型增长/衰减}模型。 \\
    \bottomrule
  \end{tabularx}
  \end{center}
\end{summarybox}

\clearpage
% === Part 3: 通项求法与递推模型 ===
\section{通项求法与递推模型}

\subsection{利用 \texorpdfstring{$S_n$}{Sn} 与 \texorpdfstring{$a_n$}{an} 的关系求通项}

\begin{definitionbox}{基本关系}
  设 $S_n$ 为 $\{a_n\}$ 的前 $n$ 项和,则
  \[
    a_n = S_n - S_{n-1}\quad(n\ge2),\qquad a_1=S_1.
  \]
  若已知 $S_n$ 的解析式,只要能够\keyword{化简} $S_n-S_{n-1}$,即可得到 $a_n$ 的通项。
\end{definitionbox}

\begin{examplebox}{典型题·由 $S_n$ 求通项}
  已知数列 $\{a_n\}$ 的前 $n$ 项和为
  \[
    S_n = n^2+3n.
  \]
  求通项 $a_n$。

  \textbf{解:}
  \[
    a_n = S_n - S_{n-1} = (n^2+3n)-\bigl((n-1)^2+3(n-1)\bigr) = 2n+2.
  \]
  故 $a_n=2n+2$,是一个等差数列。
\end{examplebox}

\begin{sideinfobox}{常见结构识别}
  \begin{itemize}
    \item $S_n$ 含有 $n,n^2,n^3$ 等多项式时,$a_n$ 往往仍是多项式(等差或二阶差分常数);
    \item $S_n$ 含有 $q^n$ 时,$a_n$ 常为等比或等比与等差的\keyword{叠加};
    \item 解题时要刻意训练「\keyword{从和到项}」的\inlinehint{差分眼光}。
  \end{itemize}
\end{sideinfobox}

\subsection{构造法求通项}

\begin{definitionbox}{构造法的基本思路}
  构造法的核心是:\keyword{人为创造一个熟悉的结构}。
  \begin{itemize}
    \item 通过构造新的数列(如 $b_n=a_n+\lambda$、$b_n=a_n/n$、$b_n=a_n-q a_{n-1}$)简化递推;
    \item 通过\keyword{差分}或\keyword{比值}将复杂数列还原为等差/等比;
    \item 通过与\keyword{函数图像}或\keyword{组合模型}建立联系,理解数列变化节奏。
  \end{itemize}
\end{definitionbox}

\begin{examplebox}{典型题·构造新数列}
  数列 $\{a_n\}$ 满足递推关系
  \[
    a_{n+1}=a_n+2n,\quad a_1=1.
  \]
  求通项 $a_n$。

  \textbf{解法一:累加思想}\\
  由递推式连加:
  \[
    a_n = a_1+\sum_{k=1}^{n-1}2k = 1 + (n-1)n = n^2-n+1.
  \]

  \textbf{解法二:构造差分}\\
  注意到 $a_{n+1}-a_n=2n$,即
  \[
    a_{n+1}-(n+1)^2 = a_n-n^2.
  \]
  令 $b_n=a_n-n^2$,则 $b_{n+1}=b_n$,于是 $b_n\equiv b_1=a_1-1^2=0$。从而 $a_n=n^2$ 与上式对照可修正为 $a_n=n^2-n+1$。
\end{examplebox}

\begin{sideinfobox}{构造法常见套路}
  \begin{itemize}
    \item 将 $a_n$ 与 $n$ 的某个简单函数($n,n^2,q^n$ 等)组合;
    \item 对递推式进行\keyword{移项}、\keyword{配方}或\keyword{系数调整},试图出现「不变式」;
    \item 与高等数学中\keyword{线性非齐次微分方程}的解法类似(解 = 通解 + 特解)。
  \end{itemize}
\end{sideinfobox}

\subsection{特征根法求通项}

\begin{definitionbox}{线性齐次递推与特征方程}
  设数列 $\{a_n\}$ 满足
  \[
    a_{n+2}+p a_{n+1}+q a_n=0,
  \]
  其中 $p,q$ 为常数。称其为\keyword{二阶线性齐次递推数列}。其\keyword{特征方程}为
  \[
    \lambda^2+p\lambda+q=0.
  \]
  \begin{itemize}
    \item 若有两个不等实根 $\lambda_1,\lambda_2$,则 $a_n=A\lambda_1^n+B\lambda_2^n$;
    \item 若为重根 $\lambda$,则 $a_n=(A+Bn)\lambda^n$。
  \end{itemize}
  在高等数学中,这与线性微分方程 $y''+py'+qy=0$ 的解法是同一套路。
\end{definitionbox}

\begin{examplebox}{典型题·特征根法}
  数列 $\{a_n\}$ 满足
  \[
    a_{n+2}=5a_{n+1}-6a_n,\quad a_1=2,a_2=5.
  \]
  求通项 $a_n$。

  \textbf{解:}特征方程
  \[
    \lambda^2-5\lambda+6=0\Rightarrow (\lambda-2)(\lambda-3)=0,
  \]
  得 $\lambda_1=2,\lambda_2=3$,故有 $a_n=A\cdot2^n+B\cdot3^n$。
  利用初值:
  \[
    \begin{cases}
      a_1=2A+3B=2,\\
      a_2=4A+9B=5,
    \end{cases}
    \Rightarrow A=1,B=0.
  \]
  故 $a_n=2^n$。
\end{examplebox}

\begin{sideinfobox}{常见变式}
  \begin{itemize}
    \item 当递推式带常数项时,可先求\keyword{齐次方程通解},再构造\keyword{特解};
    \item 若系数含 $n$,通常需结合\keyword{构造法}或\keyword{累加法},特征根不再直接使用。
  \end{itemize}
\end{sideinfobox}

\subsection{斐波那契数列模型}

\begin{definitionbox}{斐波那契数列}
  经典斐波那契数列定义为
  \[
    F_1=1,F_2=1,\quad F_{n+2}=F_{n+1}+F_n.
  \]
  它是特征根法的一个重要示例,其特征方程为
  \[
    \lambda^2-\lambda-1=0.
  \]
\end{definitionbox}

\begin{examplebox}{典型题·斐波那契求通项}
  利用特征根法求斐波那契数列的通项。

  \textbf{解:}特征方程
  \[
    \lambda^2-\lambda-1=0
  \]
  的两根为
  \[
    \lambda_{1,2}=\frac{1\pm\sqrt{5}}{2}.
  \]
  故 $F_n=A\lambda_1^n+B\lambda_2^n$。由 $F_1=1,F_2=1$ 解得
  \[
    F_n=\frac{1}{\sqrt{5}}\Bigl(\lambda_1^n-\lambda_2^n\Bigr),
  \]
  这就是著名的 Binet 公式。
\end{examplebox}

\begin{sideinfobox}{斐波那契模型在考试中的常见题型}
  \begin{itemize}
    \item 含有斐波那契数列的\keyword{比值极限},如 $\displaystyle\lim_{n\to\infty}\frac{F_{n+1}}{F_n}$;
    \item 用斐波那契数列建模\keyword{阶梯走法}、\keyword{铺砖问题}等计数问题;
    \item 在求和题中,利用递推式和错位相减构造\keyword{相消结构}。
  \end{itemize}
\end{sideinfobox}

\clearpage
% === Part 4: 特殊结构:奇偶项与公共项 ===
\section{特殊结构:奇偶项与公共项}

\subsection{奇偶项模型}

\begin{definitionbox}{奇偶项分列思想}
  若数列递推在奇数项与偶数项上呈现不同规律,可以分别考虑子数列:
  \[
    \{a_{2n-1}\},\quad\{a_{2n}\}.
  \]
  解题时通常将原数列拆成两个\keyword{等差/等比或线性递推}数列。
\end{definitionbox}

\begin{examplebox}{典型题·奇偶项递推}
  已知数列 $\{a_n\}$ 满足
  \[
    a_{n+2}=a_n+2,\quad a_1=1,a_2=2.
  \]
  求通项 $a_n$。

  \textbf{解:}分别考虑奇数项和偶数项:
  \[
    a_3=a_1+2=3,\ a_5=a_3+2=5,\dots
  \]
  可见
  \[
    a_{2n-1}=2n-1;
  \]
  同理
  \[
    a_4=a_2+2=4,\ a_6=a_4+2=6,\dots
  \]
  得
  \[
    a_{2n}=2n.
  \]
  综合得到
  \[
    a_n=
    \begin{cases}
      n, & n\text{ 为偶数},\\
      n, & n\text{ 为奇数},
    \end{cases}
  \]
  即 $a_n=n$。
\end{examplebox}

\begin{notebox}{解题提示}
  \begin{itemize}
    \item 看到「隔一个相加/相减」的递推式时,优先尝试奇偶项分列;
    \item 对子数列使用等差/等比或特征根法,再合成为原数列;
    \item 常与\keyword{不等式放缩}结合,分别估计奇偶项的大小。
  \end{itemize}
\end{notebox}

\subsection{数列的公共项模型}

\begin{definitionbox}{公共项}
  若两数列 $\{a_n\}$、$\{b_n\}$ 中存在若干相同的数,这些相同的数称为\keyword{公共项}。
  \begin{itemize}
    \item 常见问题:求公共项的\keyword{个数}、\keyword{最大公共项}或\keyword{第 $k$ 个公共项};
    \item 解法核心:\keyword{列方程}——令 $a_m=b_n$。
  \end{itemize}
\end{definitionbox}

\begin{examplebox}{典型题·等差数列的公共项}
  已知等差数列
  \[
    a_n=2n+1,\quad b_n=5n-4.
  \]
  求两数列的所有公共项。

  \textbf{解:}令 $2m+1=5n-4$,得
  \[
    2m=5n-5=5(n-1)\Rightarrow m=\frac{5}{2}(n-1).
  \]
  要求 $m$ 为正整数,故 $n-1$ 必须为偶数,令 $n-1=2k$,则
  \[
    n=2k+1,\quad m=5k.
  \]
  公共项为
  \[
    a_m=a_{5k}=2(5k)+1=10k+1.
  \]
  因此,两数列的公共项构成等差数列 $\{10k+1\}$。
\end{examplebox}

\clearpage
% === Part 5: 求和策略总览 ===
\section{求和策略总览}

\begin{summarybox}{常见求和模型一览表}
  \begin{center}
  \begin{tabularx}{0.98\textwidth}{>{\raggedright\arraybackslash}p{3cm}X}
    \toprule
    \textbf{模型} & \textbf{核心思路与典型形式} \\
    \midrule
    累加模型 &
      直接利用 $S_n$ 定义,将递推式或通项式累加,常见于形如 $a_{n+1}-a_n=f(n)$。 \\
    累乘模型 &
      对乘积 $\prod(1+k_n)$ 常取对数转化为加法,或识别为等比结构。 \\
    分组求和 &
      将项按 2 项或多项一组,化简后再求和,例如 $\sum (a_{2k-1}+a_{2k})$。 \\
    并项求和 &
      将两个和式合并,如 $\sum a_n+\sum b_n=\sum (a_n+b_n)$,或对齐指标。 \\
    错位相减 &
      构造 $S_n$ 与 $qS_n$,再相减得到大量相消项,典型于等比或类等比求和。 \\
    裂项相消 &
      将一般项拆成差分形式 $u_n=v_n-v_{n+1}$,从而形成首尾相消结构。 \\
    归纳求和 &
      对形如 $\sum f(k)$ 的显式公式,用数学归纳法\keyword{证明或猜测}闭式。 \\
    不等式放缩 &
      用单调性、夹逼定理或柯西不等式估计和的范围。 \\
    \bottomrule
  \end{tabularx}
  \end{center}
\end{summarybox}

\subsection{累加累乘模型}

\begin{examplebox}{累加模型}
  数列 $\{a_n\}$ 满足 $a_{n+1}-a_n=\dfrac{1}{n}$,$a_1=0$。求 $a_n$。

  \textbf{解:}
  \[
    a_n = a_1+\sum_{k=1}^{n-1}\frac{1}{k} = \sum_{k=1}^{n-1}\frac{1}{k}.
  \]
  该式在高等数学中对应于\keyword{调和级数}部分和,常与积分比较法联系。
\end{examplebox}

\begin{examplebox}{累乘模型}
  求
  \[
    P_n=\prod_{k=1}^{n}\left(1+\frac{1}{k}\right).
  \]

  \textbf{解:}注意到
  \[
    1+\frac{1}{k}=\frac{k+1}{k},
  \]
  故
  \[
    P_n=\prod_{k=1}^{n}\frac{k+1}{k}=\frac{2}{1}\cdot\frac{3}{2}\cdots\frac{n+1}{n}=n+1.
  \]
  这实际上是一个\keyword{裂项相消}的乘积模型。
\end{examplebox}

\subsection{分组求和与并项求和模型}

\begin{examplebox}{分组求和}
  求和
  \[
    S_n = \sum_{k=1}^{n}(2k-1)(2k+1).
  \]

  \textbf{解:}先化简一般项:
  \[
    (2k-1)(2k+1)=4k^2-1.
  \]
  利用\keyword{分组思想},可写作
  \[
    S_n=\sum_{k=1}^n4k^2-\sum_{k=1}^n1
    =4\sum_{k=1}^n k^2-n
    =\frac{4n(n+1)(2n+1)}{6}-n.
  \]
  若已知 $\displaystyle\sum_{k=1}^n k^2$ 的公式,也可用\keyword{归纳法模型}证明。
\end{examplebox}

\begin{examplebox}{并项求和}
  已知
  \[
    S_n=\sum_{k=1}^n\left(\frac{1}{k}-\frac{1}{k+1}\right).
  \]
  求 $S_n$。

  \textbf{解:}写出前几项:
  \[
    S_n=\left(1-\frac{1}{2}\right)+\left(\frac{1}{2}-\frac{1}{3}\right)+\cdots+\left(\frac{1}{n}-\frac{1}{n+1}\right)
    =1-\frac{1}{n+1}.
  \]
  这就是典型的\keyword{裂项相消}模型。
\end{examplebox}

\subsection{错位相减法求和模型}

\begin{definitionbox}{错位相减法}
  当求和式中包含等比或近似等比结构时,经常构造
  \[
    qS_n-S_n
  \]
  以实现\keyword{错位相减},让中间大部分项相消。
\end{definitionbox}

\begin{examplebox}{典型题·错位相减}
  求
  \[
    S_n=\frac{1}{2}+\frac{1}{2^2}+\cdots+\frac{1}{2^n}.
  \]

  \textbf{解:}令 $S_n$ 如上,则
  \[
    \frac{1}{2}S_n=\frac{1}{2^2}+\frac{1}{2^3}+\cdots+\frac{1}{2^{n+1}}.
  \]
  相减得
  \[
    S_n-\frac{1}{2}S_n=\frac{1}{2}-\frac{1}{2^{n+1}}
    \Rightarrow S_n=1-\frac{1}{2^{n}}.
  \]
  在高等数学中,令 $n\to\infty$ 可得到无穷级数 $\sum_{k=1}^{\infty}\frac{1}{2^k}=1$。
\end{examplebox}

\subsection{裂项相消法求和模型}

\begin{definitionbox}{裂项思想}
  通过\keyword{部分分式分解}或\keyword{配方},将一般项写成
  \[
    t_n=u_n-u_{n+1},
  \]
  从而
  \[
    \sum_{k=1}^{n}t_k=(u_1-u_2)+(u_2-u_3)+\cdots+(u_n-u_{n+1})=u_1-u_{n+1}.
  \]
\end{definitionbox}

\begin{examplebox}{典型题·裂项相消}
  求
  \[
    S_n=\sum_{k=1}^n\frac{1}{k(k+1)}.
  \]

  \textbf{解:}先裂项:
  \[
    \frac{1}{k(k+1)}=\frac{1}{k}-\frac{1}{k+1}.
  \]
  故
  \[
    S_n=\sum_{k=1}^n\left(\frac{1}{k}-\frac{1}{k+1}\right)=1-\frac{1}{n+1}.
  \]
\end{examplebox}

\clearpage
% === Part 6: 数学归纳法模型 ===
\section{数学归纳法模型}

\begin{definitionbox}{数学归纳法}
  对自然数命题 $P(n)$,若
  \begin{itemize}
    \item \textbf{起步}:$P(1)$ 成立;
    \item \textbf{归纳}:对任意 $k\ge1$,由 $P(k)$ 成立可推出 $P(k+1)$ 成立,
  \end{itemize}
  则对所有 $n\in\N^*$,命题 $P(n)$ 成立。
  在数列问题中,常用于证明\keyword{通项公式}或\keyword{求和公式}。
\end{definitionbox}

\begin{examplebox}{典型题·用归纳法证明求和公式}
  证明
  \[
    1+2+\cdots+n=\frac{n(n+1)}{2}.
  \]

  \textbf{证:}
  \begin{enumerate}
    \item 当 $n=1$ 时,左边为 $1$,右边为 $\dfrac{1\cdot2}{2}=1$,命题成立;
    \item 假设当 $n=k$ 时命题成立,即
    \[
      1+2+\cdots+k=\frac{k(k+1)}{2}.
    \]
    则当 $n=k+1$ 时,有
    \[
      1+2+\cdots+k+(k+1)=\frac{k(k+1)}{2}+(k+1)
      =\frac{(k+1)(k+2)}{2},
    \]
    即命题对 $k+1$ 亦成立。
  \end{enumerate}
  由数学归纳法,结论对一切正整数 $n$ 成立。
\end{examplebox}

\begin{examplebox}{典型题·归纳证明数列不等式}
  设 $a_n=2^n$,证明
  \[
    a_n\ge n+1,\quad \forall n\in\N^*.
  \]

  \textbf{证:}
  \begin{enumerate}
    \item 当 $n=1$ 时,$a_1=2\ge2$,命题成立;
    \item 假设对 $n=k$ 成立,即 $2^k\ge k+1$,则
    \[
      2^{k+1}=2\cdot2^k\ge2(k+1)=k+1+k+1\ge k+2,
    \]
    故对 $n=k+1$ 亦成立。
  \end{enumerate}
\end{examplebox}

\clearpage
% === Part 7: 数列不等式与放缩模型 ===
\section{数列不等式与放缩模型}

\begin{definitionbox}{放缩的基本思想}
  放缩就是通过构造\keyword{上下界}或\keyword{等价形式},把复杂的数列不等式转化为更\keyword{熟悉}的形式。
  \begin{itemize}
    \item 对单调数列配合\keyword{夹逼}或\keyword{上、下界}讨论;
    \item 对和式使用\keyword{柯西不等式}、\keyword{均值不等式}进行估计;
    \item 与高等数学中\keyword{比较判别法}、\keyword{柯西收敛准则}等思想相通。
  \end{itemize}
\end{definitionbox}

\begin{examplebox}{典型题·单调与上界}
  数列 $a_n=\left(1+\dfrac{1}{n}\right)^n$。证明 $\{a_n\}$ 单调递增且有上界。

  \textbf{提示思路:}
  \begin{focuspoints}
    \item 利用
    \[
      \left(1+\frac{1}{n}\right)^{n+1}=\left(1+\frac{1}{n}\right)^n\left(1+\frac{1}{n}\right),
    \]
    比较 $a_{n+1}$ 与 $a_n$ 的大小;
    \item 构造上界(如 $a_n<3$),可用二项式展开或函数单调性;
    \item 在高等数学中,可证明 $\lim_{n\to\infty}a_n=e$。
  \end{focuspoints}
\end{examplebox}

\begin{examplebox}{典型题·和式不等式放缩}
  设 $a_n=\dfrac{1}{\sqrt{n}}$,证明
  \[
    \sum_{k=1}^{n}\frac{1}{\sqrt{k}}\le 2\sqrt{n}.
  \]

  \textbf{证明思路:}
  \begin{focuspoints}
    \item 可以利用\keyword{积分比较}:注意到
    \[
      \int_{0}^{n}\frac{1}{\sqrt{x+1}}\dd x=2(\sqrt{n+1}-1);
    \]
    \item 对每一项构造上界:$\dfrac{1}{\sqrt{k}}\le2(\sqrt{k}-\sqrt{k-1})$,再累加相消;
    \item 累加后得到 $S_n\le2\sqrt{n}$。
  \end{focuspoints}
\end{examplebox}

\clearpage
% === Part 8: 综合策略与复习路线 ===
\section{综合策略与复习路线}

\begin{NoteRoadmap}{数列解题 Roadmap}
  \RoadmapStep{%
    \textbf{识别数列类型}\\[0.15em]
    初步判断:是否等差/等比?是否线性递推?是否含有奇偶项结构或公共项问题?
  }
  \RoadmapStep{%
    \textbf{锁定主模型}\\[0.15em]
    在\keyword{通项}与\keyword{求和}之间切换视角,选择合适的模型(特征根、构造法、S\_n–a\_n 等)。
  }
  \RoadmapStep{%
    \textbf{尝试结构化变形}\\[0.15em]
    通过\keyword{裂项}、\keyword{错位相减}、\keyword{分组}或\keyword{对数}将复杂表达式拆解。
  }
  \RoadmapStep{%
    \textbf{检查单调性与极限}\\[0.15em]
    结合不等式放缩,判断数列收敛性或和式大小,联想到高数中的极限定理。
  }
  \RoadmapStep{%
    \textbf{归纳与回顾}\\[0.15em]
    对重要公式使用\keyword{数学归纳法}或\keyword{极限方法}再次验证,形成稳定记忆。
  }
\end{NoteRoadmap}

\begin{examplebox}{综合例题·多模型结合}
  设数列 $\{a_n\}$ 满足
  \[
    a_1=1,\quad a_{n+1}=a_n+\frac{1}{n(n+1)}.
  \]
  \begin{enumerate}
    \item 写出 $a_n$ 的前几项,猜想通项公式;
    \item 用数学归纳法证明你的猜想;
    \item 求 $\displaystyle\lim_{n\to\infty}a_n$。
  \end{enumerate}

  \textbf{思路概览:}
  \begin{focuspoints}
    \item 先裂项:$\dfrac{1}{n(n+1)}=\dfrac{1}{n}-\dfrac{1}{n+1}$,识别为\keyword{裂项相消}模型;
    \item 累加递推式得到 $a_n=1+\sum_{k=1}^{n-1}\dfrac{1}{k(k+1)}$;
    \item 应用相消得到 $a_n=2-\dfrac{1}{n}$,再用归纳法验证;
    \item 由 $\displaystyle\lim_{n\to\infty}a_n=2$,自然过渡到高等数学中极限的概念。
  \end{focuspoints}
\end{examplebox}

\begin{NoteChecklist}{考前速记清单}
  \item 等差、等比的\keyword{通项与和公式};
  \item $S_n$ 与 $a_n$ 的\keyword{差分关系};
  \item 构造法常见变形:$a_n+\lambda$、$a_n/n$、$a_{n+1}-qa_n$;
  \item 特征根法:二阶线性齐次递推的\keyword{模板型结论};
  \item 斐波那契:递推式、比值极限与 Binet 公式;
  \item 奇偶项与公共项的\keyword{分列与方程}思想;
  \item 求和模型:分组、错位相减、裂项相消、累乘转对数;
  \item 数学归纳法的两步:\keyword{起步}与\keyword{归纳};
  \item 不等式放缩:单调性、夹逼、比较与积分估计;
  \item 将每一道题归类到上述某一个或多个模型中。
\end{NoteChecklist}

\end{document}
