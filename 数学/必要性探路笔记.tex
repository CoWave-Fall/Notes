\documentclass[a4paper,12pt]{ctexart}

\ProvidesFile{mathnote-preamble.tex}[2024/11/13 Math note modern preamble]
\RequirePackage{iftex}
\ifPDFTeX
  \PackageError{mathnote-preamble}{请使用 \XeLaTeX\ 或 LuaLaTeX 编译该模板}{在 MikTeX / TeX Live 中切换到 \XeLaTeX\ (推荐) 或 LuaLaTeX 后重新编译。}
\fi
\makeatletter
\newif\ifmathnote@fontdirfound
\mathnote@fontdirfoundfalse
\def\mathnote@fontdircandidates{fonts/,./fonts/,../fonts/,../../fonts/,../../../fonts/}
\def\mathnote@fontdir{}
\@for\mathnote@cand:=\mathnote@fontdircandidates\do{%
  \ifmathnote@fontdirfound\else
    \IfFileExists{\mathnote@cand NotoSerif-VF.ttf}{%
      \edef\mathnote@fontdir{\mathnote@cand}%
      \mathnote@fontdirfoundtrue
    }{%
      \IfFileExists{\mathnote@cand SourceHanSerifSC-Regular.otf}{%
        \edef\mathnote@fontdir{\mathnote@cand}%
        \mathnote@fontdirfoundtrue
      }{}%
    }%
  \fi
}
\ifmathnote@fontdirfound\else
  \edef\mathnote@fontdir{fonts/}%
\fi
\@ifundefined{MathNoteFontDir}{%
  \edef\MathNoteFontDir{\mathnote@fontdir}%
}{}

% --------------------------------------------------
% User switches
% --------------------------------------------------
\newif\ifmathnoteprintmode
\mathnoteprintmodefalse
\newcommand{\MathNoteEnablePrint}{%
  \mathnoteprintmodetrue
  \mathnote@applypalette
}
\newif\ifmathnotereviewstamp
\mathnotereviewstampfalse
\newcommand{\mathnote@reviewstamppathbase}{assest/lzlxV-reviewed}
\newcommand{\mathnote@reviewstampsvg}{\mathnote@reviewstamppathbase.svg}
\newcommand{\mathnote@reviewstamppdf}{\mathnote@reviewstamppathbase.pdf}
\newcommand{\mathnote@reviewstampinclude}{}
\newif\ifmathnote@reviewstampplaced
\mathnote@reviewstampplacedfalse
\IfFileExists{\mathnote@reviewstamppdf}{%
  \renewcommand{\mathnote@reviewstampinclude}{%
    \includegraphics[width=20mm,height=20mm,keepaspectratio]{\mathnote@reviewstamppdf}%
  }%
}{%
  \IfFileExists{\mathnote@reviewstampsvg}{%
    \renewcommand{\mathnote@reviewstampinclude}{%
      \includesvg[width=20mm,height=20mm,keepaspectratio]{\mathnote@reviewstamppathbase}%
    }%
  }{}%
}
\newcommand{\mathnote@enablereviewstamp}{%
  \ifx\mathnote@reviewstampinclude\@empty
    \PackageWarning{mathnote-preamble}{Review stamp graphic \mathnote@reviewstampsvg\space (or PDF fallback) not found}%
  \else
    \mathnote@reviewstampplacedfalse
    \AddToHook{shipout/foreground}{%
      \ifmathnote@reviewstampplaced\else
        \begin{tikzpicture}[remember picture, overlay]
          \node[anchor=north east, xshift=-8mm, yshift=-8mm] at (current page.north east){%
            \mathnote@reviewstampinclude
          };
        \end{tikzpicture}%
        \global\mathnote@reviewstampplacedtrue
      \fi
    }%
  \fi
}
\newcommand{\MathNoteEnableReviewStamp}{%
  \mathnotereviewstamptrue
  \mathnote@enablereviewstamp
}

% --------------------------------------------------
% Metadata defaults (can be overwritten in each file)
% --------------------------------------------------
\providecommand{\notetitle}{数学学习笔记}
\providecommand{\noteauthor}{作者}
\providecommand{\notedate}{\today}
\providecommand{\notesubtitle}{现代数学排版示例}
\providecommand{\noteversion}{v1.0}

% --------------------------------------------------
% Core packages
% --------------------------------------------------
\usepackage{geometry}
\geometry{
  paper=a4paper,
  top=2.35cm,
  bottom=2.4cm,
  left=2.1cm,
  right=2.1cm,
  headheight=16pt,
  headsep=14pt
}

\usepackage{fontspec}
\usepackage{metalogo}
\defaultfontfeatures{Ligatures=TeX, Scale=MatchLowercase}

\newcommand{\mathnote@fontfile}[1]{\MathNoteFontDir#1}
\newif\ifmathnote@haslocalserif
\newif\ifmathnote@haslocalsans
\newif\ifmathnote@haslocalmono
\newif\ifmathnote@haslocalcjkserif
\newif\ifmathnote@haslocalcjksans
\newif\ifmathnote@haslocalcjkmono
\newif\ifmathnote@haslocalkai
\newif\ifmathnote@haslocalshserif
\newif\ifmathnote@haslocalshsans
\IfFileExists{\mathnote@fontfile{NotoSerif-VF.ttf}}{\mathnote@haslocalseriftrue}{\mathnote@haslocalseriffalse}
\IfFileExists{\mathnote@fontfile{NotoSans-VF.ttf}}{\mathnote@haslocalsanstrue}{\mathnote@haslocalsansfalse}
\IfFileExists{\mathnote@fontfile{NotoSansMono-VF.ttf}}{\mathnote@haslocalmonotrue}{\mathnote@haslocalmonofalse}
\IfFileExists{\mathnote@fontfile{NotoSerifCJK-VF.ttc}}{\mathnote@haslocalcjkseriftrue}{\mathnote@haslocalcjkseriffalse}
\IfFileExists{\mathnote@fontfile{NotoSansCJK-VF.ttc}}{\mathnote@haslocalcjksanstrue}{\mathnote@haslocalcjksansfalse}
\IfFileExists{\mathnote@fontfile{NotoSansMonoCJK-VF.ttc}}{\mathnote@haslocalcjkmonotrue}{\mathnote@haslocalcjkmonofalse}
\IfFileExists{\mathnote@fontfile{LXGWWenKaiSC-Regular.ttf}}{\mathnote@haslocalkaitrue}{\mathnote@haslocalkaifalse}
\IfFileExists{\mathnote@fontfile{SourceHanSerifSC-Regular.otf}}{\mathnote@haslocalshseriftrue}{\mathnote@haslocalshseriffalse}
\IfFileExists{\mathnote@fontfile{SourceHanSansSC-Regular.otf}}{\mathnote@haslocalshsanstrue}{\mathnote@haslocalshsansfalse}

\newcommand{\mathnote@cjkitalicfont}{}
\newcommand{\mathnote@cjkitalicfeatures}{Language = Chinese Simplified}
\ifmathnote@haslocalkai
  \def\mathnote@cjkitalicfont{LXGWWenKaiSC-Regular}
  \def\mathnote@cjkitalicfeatures{Path = {\MathNoteFontDir}, Extension = .ttf, Language = Chinese Simplified}
\else
  \IfFontExistsTF{LXGW WenKai SC}{%
    \def\mathnote@cjkitalicfont{LXGW WenKai SC}%
    \def\mathnote@cjkitalicfeatures{Language = Chinese Simplified}
  }{%
    \def\mathnote@cjkitalicfont{}%
    \def\mathnote@cjkitalicfeatures{Language = Chinese Simplified}
  }%
\fi
\ifx\mathnote@cjkitalicfont\@empty
  \def\mathnote@cjkitalicfont{FandolKai}
  \def\mathnote@cjkitalicfeatures{Language = Chinese Simplified}
\fi

\newcommand{\mathnote@setlatinfonts}{%
  \ifmathnote@haslocalserif
    \setmainfont{Noto Serif}[
      Path = {\MathNoteFontDir},
      Extension = .ttf,
      UprightFont = NotoSerif-VF,
      ItalicFont = NotoSerif-Italic-VF,
      BoldFont = NotoSerif-VF,
      BoldFeatures = {RawFeature={+wght=760}},
      BoldItalicFont = NotoSerif-Italic-VF,
      BoldItalicFeatures = {RawFeature={+wght=760}}
    ]%
  \else
    \IfFontExistsTF{Noto Serif}{%
      \setmainfont{Noto Serif}[
        ItalicFont = {Noto Serif Italic},
        BoldFont = {Noto Serif Bold},
        BoldItalicFont = {Noto Serif Bold Italic}
      ]%
    }{%
      \setmainfont{TeX Gyre Pagella}%
    }%
  \fi
  \ifmathnote@haslocalsans
    \setsansfont{Noto Sans}[
      Path = {\MathNoteFontDir},
      Extension = .ttf,
      UprightFont = NotoSans-VF,
      ItalicFont = NotoSans-Italic-VF,
      BoldFont = NotoSans-VF,
      BoldFeatures = {RawFeature={+wght=760}},
      BoldItalicFont = NotoSans-Italic-VF,
      BoldItalicFeatures = {RawFeature={+wght=760}}
    ]%
  \else
    \IfFontExistsTF{Noto Sans}{%
      \setsansfont{Noto Sans}[
        ItalicFont = {Noto Sans Italic},
        BoldFont = {Noto Sans Bold},
        BoldItalicFont = {Noto Sans Bold Italic}
      ]%
    }{%
      \setsansfont{TeX Gyre Heros}%
    }%
  \fi
  \ifmathnote@haslocalmono
    \setmonofont{Noto Sans Mono}[
      Path = {\MathNoteFontDir},
      Extension = .ttf,
      UprightFont = NotoSansMono-VF,
      BoldFont = NotoSansMono-VF,
      BoldFeatures = {RawFeature={+wght=740}},
      ItalicFont = NotoSansMono-VF,
      ItalicFeatures = {FakeSlant=0.2}
    ]%
  \else
    \IfFontExistsTF{Noto Sans Mono}{%
      \setmonofont{Noto Sans Mono}[
        BoldFont = {Noto Sans Mono Bold}
      ]%
    }{%
      \setmonofont{TeX Gyre Cursor}%
    }%
  \fi
}

\newcommand{\mathnote@setcjkfonts}{%
  \ifmathnote@haslocalshserif
    \setCJKmainfont{SourceHanSerifSC-Regular}[
      Path = {\MathNoteFontDir},
      Extension = .otf,
      Language = Chinese Simplified,
      BoldFont = SourceHanSerifSC-Bold,
      ItalicFont = {\mathnote@cjkitalicfont},
      ItalicFeatures = {\mathnote@cjkitalicfeatures}
    ]%
  \else
    \IfFontExistsTF{Source Han Serif SC}{%
      \setCJKmainfont{Source Han Serif SC}[Language=Chinese Simplified, ItalicFont={\mathnote@cjkitalicfont}, ItalicFeatures={\mathnote@cjkitalicfeatures}]%
    }{%
      \ifmathnote@haslocalcjkserif
        \setCJKmainfont{NotoSerifCJK-VF}[
          Path = {\MathNoteFontDir},
          Extension = .ttc,
          Language = Chinese Simplified,
          UprightFont = NotoSerifCJK-VF,
          UprightFeatures = {FontIndex=2},
          BoldFont = NotoSerifCJK-VF,
          BoldFeatures = {FontIndex=2,RawFeature={+wght=780}},
          AutoFakeSlant = 0.18,
          ItalicFont = {\mathnote@cjkitalicfont},
          ItalicFeatures = {\mathnote@cjkitalicfeatures}
        ]%
      \else
        \IfFontExistsTF{Noto Serif CJK SC}{%
          \setCJKmainfont{Noto Serif CJK SC}[Language=Chinese Simplified, ItalicFont={\mathnote@cjkitalicfont}, ItalicFeatures={\mathnote@cjkitalicfeatures}]
        }{%
          \setCJKmainfont{FandolSong}[BoldFont={FandolSong-Bold}, ItalicFont={\mathnote@cjkitalicfont}, ItalicFeatures={\mathnote@cjkitalicfeatures}]
        }%
      \fi
    }%
  \fi
  \ifmathnote@haslocalshsans
    \setCJKsansfont{SourceHanSansSC-Regular}[
      Path = {\MathNoteFontDir},
      Extension = .otf,
      Language = Chinese Simplified,
      BoldFont = SourceHanSansSC-Bold,
      ItalicFont = {\mathnote@cjkitalicfont},
      ItalicFeatures = {\mathnote@cjkitalicfeatures}
    ]%
    \setCJKfamilyfont{hei}{SourceHanSansSC-Regular}[
      Path = {\MathNoteFontDir},
      Extension = .otf,
      BoldFont = SourceHanSansSC-Bold
    ]%
  \else
    \IfFontExistsTF{Source Han Sans SC}{%
      \setCJKsansfont{Source Han Sans SC}[Language=Chinese Simplified, ItalicFont={\mathnote@cjkitalicfont}, ItalicFeatures={\mathnote@cjkitalicfeatures}]
      \setCJKfamilyfont{hei}{Source Han Sans SC}[Language=Chinese Simplified]
    }{%
      \ifmathnote@haslocalcjksans
        \setCJKsansfont{NotoSansCJK-VF}[
          Path = {\MathNoteFontDir},
          Extension = .ttc,
          Language = Chinese Simplified,
          UprightFont = NotoSansCJK-VF,
          UprightFeatures = {FontIndex=2},
          BoldFont = NotoSansCJK-VF,
          BoldFeatures = {FontIndex=2,RawFeature={+wght=780}},
          ItalicFont = {\mathnote@cjkitalicfont},
          ItalicFeatures = {\mathnote@cjkitalicfeatures}
        ]%
        \setCJKfamilyfont{hei}{NotoSansCJK-VF}[
          Path = {\MathNoteFontDir},
          Extension = .ttc,
          UprightFont = NotoSansCJK-VF,
          UprightFeatures = {FontIndex=2},
          BoldFont = NotoSansCJK-VF,
          BoldFeatures = {FontIndex=2,RawFeature={+wght=820}}
        ]%
      \else
        \IfFontExistsTF{Noto Sans CJK SC}{%
          \setCJKsansfont{Noto Sans CJK SC}[Language=Chinese Simplified, ItalicFont={\mathnote@cjkitalicfont}, ItalicFeatures={\mathnote@cjkitalicfeatures}]
          \setCJKfamilyfont{hei}{Noto Sans CJK SC}[Language=Chinese Simplified]
        }{%
          \setCJKsansfont{FandolHei}[ItalicFont={\mathnote@cjkitalicfont}, ItalicFeatures={\mathnote@cjkitalicfeatures}]
          \setCJKfamilyfont{hei}{FandolHei}
        }%
      \fi
    }%
  \fi
  \ifmathnote@haslocalshsans
    \setCJKmonofont{SourceHanSansSC-Regular}[
      Path = {\MathNoteFontDir},
      Extension = .otf,
      Language = Chinese Simplified,
      BoldFont = SourceHanSansSC-Bold,
      ItalicFont = {\mathnote@cjkitalicfont},
      ItalicFeatures = {\mathnote@cjkitalicfeatures}
    ]%
  \else
    \IfFontExistsTF{Source Han Sans SC}{%
      \setCJKmonofont{Source Han Sans SC}[Language=Chinese Simplified, ItalicFont={\mathnote@cjkitalicfont}, ItalicFeatures={\mathnote@cjkitalicfeatures}]
    }{%
      \ifmathnote@haslocalcjkmono
        \setCJKmonofont{NotoSansMonoCJK-VF}[
          Path = {\MathNoteFontDir},
          Extension = .ttc,
          Language = Chinese Simplified,
          UprightFont = NotoSansMonoCJK-VF,
          UprightFeatures = {FontIndex=2},
          BoldFont = NotoSansMonoCJK-VF,
          BoldFeatures = {FontIndex=2,RawFeature={+wght=760}},
          ItalicFont = {\mathnote@cjkitalicfont},
          ItalicFeatures = {\mathnote@cjkitalicfeatures}
        ]%
      \else
        \IfFontExistsTF{Noto Sans Mono CJK SC}{%
          \setCJKmonofont{Noto Sans Mono CJK SC}[Language=Chinese Simplified, ItalicFont={\mathnote@cjkitalicfont}, ItalicFeatures={\mathnote@cjkitalicfeatures}]
        }{%
          \setCJKmonofont{FandolFang}[ItalicFont={\mathnote@cjkitalicfont}, ItalicFeatures={\mathnote@cjkitalicfeatures}]
        }%
      \fi
    }%
  \fi
  \ifmathnote@haslocalkai
    \setCJKfamilyfont{kai}{LXGW WenKai SC}[
      Path = {\MathNoteFontDir},
      Extension = .ttf,
      UprightFont = LXGWWenKaiSC-Regular,
      BoldFont = LXGWWenKaiSC-Medium,
      AutoFakeBold = 1.25,
      Language = Chinese Simplified
    ]%
    \setCJKfamilyfont{zhkai}{LXGW WenKai SC}[
      Path = {\MathNoteFontDir},
      Extension = .ttf,
      UprightFont = LXGWWenKaiSC-Regular,
      BoldFont = LXGWWenKaiSC-Medium,
      AutoFakeBold = 1.25,
      Language = Chinese Simplified
    ]%
  \else
    \IfFontExistsTF{LXGW WenKai SC}{%
      \setCJKfamilyfont{kai}{LXGW WenKai SC}[Language=Chinese Simplified]
      \setCJKfamilyfont{zhkai}{LXGW WenKai SC}[Language=Chinese Simplified]
    }{%
      \setCJKfamilyfont{kai}{FandolKai}[Language=Chinese Simplified]
      \setCJKfamilyfont{zhkai}{FandolKai}[Language=Chinese Simplified]
    }%
  \fi
}

\mathnote@setlatinfonts
\mathnote@setcjkfonts
\renewcommand{\kaishu}{\CJKfamily{kai}}
\xeCJKsetup{
  CheckSingle = true,
  RubberPunctSkip = true,
  PunctStyle = plain
}
\xeCJKsetwidth{,}{0.5em}
\xeCJKsetwidth{。}{1em}
\newlength{\mathnote@commaspace}
\setlength{\mathnote@commaspace}{0.5em}
\catcode`,=\active
\protected\def,{,\kern\mathnote@commaspace}
\clubpenalty=10000
\widowpenalty=10000
\displaywidowpenalty=10000

\usepackage{microtype}
\usepackage{setspace}
\setstretch{1.15}

\usepackage{amsmath, amssymb, amsthm, mathtools}
\usepackage{bm}
\usepackage{siunitx}
\usepackage{enumitem}
\usepackage{tikz}
\usetikzlibrary{calc, arrows.meta, decorations.pathmorphing, positioning}
\usepackage{xparse}
\usepackage{etoolbox}
\usepackage{graphicx}
\usepackage{svg}
\usepackage{caption}
\usepackage{booktabs}
\usepackage{tabularx}
\usepackage{multicol}
\usepackage{listings}
\usepackage{tcolorbox}
\tcbuselibrary{skins, breakable, hooks, listingsutf8}
\usepackage{zhnumber}
\usepackage{fancyhdr}
\usepackage{lastpage}
\usepackage{hyperref}
\usepackage{bookmark}

% Block-style paragraph headings to avoid run-in overfull boxes
\renewcommand{\paragraph}{%
  \@startsection{paragraph}{4}{\z@}%
    {1.5ex \@plus 0.5ex \@minus 0.2ex}%
    {0.65em}%
    {\normalfont\normalsize\bfseries}%
}

% --------------------------------------------------
% Colors
% --------------------------------------------------
\definecolor{screenAccent}{HTML}{1565C0}
\definecolor{screenSecondary}{HTML}{00897B}
\definecolor{screenHighlight}{HTML}{F9A826}
\definecolor{screenInfo}{HTML}{546E7A}
\definecolor{screenSurface}{HTML}{FFFFFF}

\definecolor{printAccent}{cmyk}{0.95,0.55,0,0.05}
\definecolor{printSecondary}{cmyk}{0.82,0,0.56,0.08}
\definecolor{printHighlight}{cmyk}{0,0.35,0.80,0}
\definecolor{printInfo}{cmyk}{0.60,0.47,0.43,0.20}
\definecolor{printSurface}{cmyk}{0,0,0,0}
\definecolor{mathnotePureCyan}{cmyk}{1,0,0,0}

% 16-color palettes for screen (sRGB)
\definecolor{screenTone01}{HTML}{0D47A1}
\definecolor{screenTone02}{HTML}{1565C0}
\definecolor{screenTone03}{HTML}{1A73E8}
\definecolor{screenTone04}{HTML}{2196F3}
\definecolor{screenTone05}{HTML}{00ACC1}
\definecolor{screenTone06}{HTML}{00897B}
\definecolor{screenTone07}{HTML}{2E7D32}
\definecolor{screenTone08}{HTML}{558B2F}
\definecolor{screenTone09}{HTML}{9E9D24}
\definecolor{screenTone10}{HTML}{F9A825}
\definecolor{screenTone11}{HTML}{FFB300}
\definecolor{screenTone12}{HTML}{FB8C00}
\definecolor{screenTone13}{HTML}{F4511E}
\definecolor{screenTone14}{HTML}{D84315}
\definecolor{screenTone15}{HTML}{8E24AA}
\definecolor{screenTone16}{HTML}{6A1B9A}

% 16-color palettes for print (CMYK approximations)
\definecolor{printTone01}{cmyk}{1,0.72,0,0.35}
\definecolor{printTone02}{cmyk}{0.9,0.5,0,0.2}
\definecolor{printTone03}{cmyk}{0.85,0.45,0,0.12}
\definecolor{printTone04}{cmyk}{0.65,0.25,0,0.02}
\definecolor{printTone05}{cmyk}{0.75,0.05,0.1,0.05}
\definecolor{printTone06}{cmyk}{0.85,0,0.35,0.2}
\definecolor{printTone07}{cmyk}{0.75,0,0.8,0.38}
\definecolor{printTone08}{cmyk}{0.6,0,1,0.42}
\definecolor{printTone09}{cmyk}{0.35,0,1,0.45}
\definecolor{printTone10}{cmyk}{0,0.2,1,0.02}
\definecolor{printTone11}{cmyk}{0,0.25,1,0}
\definecolor{printTone12}{cmyk}{0,0.45,1,0}
\definecolor{printTone13}{cmyk}{0,0.7,0.8,0}
\definecolor{printTone14}{cmyk}{0,0.85,0.95,0.1}
\definecolor{printTone15}{cmyk}{0.45,0.9,0,0}
\definecolor{printTone16}{cmyk}{0.6,1,0,0.1}

\colorlet{accent}{screenAccent}
\colorlet{secondary}{screenSecondary}
\colorlet{highlight}{screenHighlight}
\colorlet{inkgray}{screenInfo}
\colorlet{surface}{screenSurface}

\newif\ifmathnote@docstarted
\mathnote@docstartedfalse
\AtBeginDocument{\mathnote@docstartedtrue}

\newcommand{\mathnote@applyhypercolors}{%
  \ifmathnoteprintmode
    \hypersetup{
      colorlinks=false,
      hidelinks,
      pdfborderstyle={/S/U/W 0},
      pdfborder={0 0 0}
    }%
  \else
    \hypersetup{
      colorlinks=true,
      linkcolor=accent,
      citecolor=secondary,
      urlcolor=accent,
      pdfborder={0 0 0}
    }%
  \fi
}

\newcommand{\mathnote@applypalette}{%
  \ifmathnoteprintmode
    \colorlet{accent}{printAccent}%
    \colorlet{secondary}{printSecondary}%
    \colorlet{highlight}{printHighlight}%
    \colorlet{inkgray}{printInfo}%
    \colorlet{surface}{printSurface}%
  \else
    \colorlet{accent}{screenAccent}%
    \colorlet{secondary}{screenSecondary}%
    \colorlet{highlight}{screenHighlight}%
    \colorlet{inkgray}{screenInfo}%
    \colorlet{surface}{screenSurface}%
  \fi
  \colorlet{accentline}{accent!65!black}%
  \colorlet{accentbg}{accent!8!white}%
  \colorlet{secondarybg}{secondary!10!white}%
  \colorlet{highlightbg}{highlight!10!white}%
  \colorlet{inkline}{inkgray!60!black}%
  \colorlet{surfacegrid}{inkgray!6!white}%
  \ifmathnote@docstarted
    \mathnote@applyhypercolors
  \else
    \AtBeginDocument{\mathnote@applyhypercolors}
  \fi
}
\mathnote@applypalette
\newcommand{\MathNoteRefreshColors}{\mathnote@applypalette}
\AtBeginDocument{%
  \hypersetup{%
    pdftitle=\notetitle,
    pdfauthor=\noteauthor,
    pdfsubject=\notesubtitle,
    pdfcreator={MathNote dual-medium template}%
  }%
}

% --------------------------------------------------
% Sectioning and spacing
% --------------------------------------------------
\setlength{\parskip}{0.35em}
\setlength{\parindent}{2em}
\newlength{\mathnote@boxindent}
\setlength{\mathnote@boxindent}{\parindent}
\setcounter{secnumdepth}{3}
\setcounter{tocdepth}{2}


\ctexset{
  section={
    name={第,节},
    format+=\Large\sffamily\bfseries\color{accent},
    beforeskip=1.2em,
    afterskip=0.7em
  },
  subsection={
    format+=\large\sffamily\bfseries\color{secondary},
    beforeskip=1em,
    afterskip=0.4em
  },
  subsubsection={
    format+=\normalsize\sffamily\bfseries\color{inkgray},
    beforeskip=0.8em,
    afterskip=0.2em
  }
}

% --------------------------------------------------
% Header / footer
% --------------------------------------------------
\pagestyle{fancy}
\fancyhf{}
\fancyhead[LE]{\small\sffamily\textcolor{accent}{\notetitle\ >\ \nouppercase{\rightmark}}}
\fancyhead[RO]{\small\sffamily\textcolor{accent}{\notetitle\ >\ \nouppercase{\rightmark}}}
\fancyfoot[LE]{\small\sffamily\textcolor{inkgray}{\thepage}\ \textcolor{mathnotePureCyan}{/}\ \textcolor{inkgray}{\pageref{LastPage}}}
\fancyfoot[RO]{\small\sffamily\textcolor{inkgray}{\thepage}\ \textcolor{mathnotePureCyan}{/}\ \textcolor{inkgray}{\pageref{LastPage}}}
\fancyfoot[LO]{}
\fancyfoot[RE]{}
\renewcommand{\headrulewidth}{0.2pt}
\renewcommand{\footrulewidth}{0pt}
\renewcommand{\sectionmark}[1]{\markright{#1}}

% --------------------------------------------------
% Math helpers
% --------------------------------------------------
\DeclarePairedDelimiter\abs{\lvert}{\rvert}
\DeclarePairedDelimiter\norm{\lVert}{\rVert}
\DeclarePairedDelimiter\ceil{\lceil}{\rceil}
\DeclarePairedDelimiter\floor{\lfloor}{\rfloor}

\newcommand{\R}{\mathbb{R}}
\newcommand{\C}{\mathbb{C}}
\newcommand{\Q}{\mathbb{Q}}
\newcommand{\Z}{\mathbb{Z}}
\newcommand{\N}{\mathbb{N}}
\newcommand{\dd}{\mathop{}\!\mathrm{d}}
\newcommand{\ee}{\mathrm{e}}
\newcommand{\dv}[2]{\frac{\dd #1}{\dd #2}}
\newcommand{\pdv}[2]{\frac{\partial #1}{\partial #2}}

\lstset{
  backgroundcolor=\color{surfacegrid},
  basicstyle=\ttfamily\small,
  keywordstyle=\color{screenTone04}\bfseries,
  commentstyle=\color{screenTone06},
  stringstyle=\color{screenTone12},
  frame=none,
  columns=fullflexible,
  showstringspaces=false
}

\NewDocumentCommand{\keyword}{m}{%
  \textcolor{accent}{\textbf{#1}}%
}

\NewDocumentCommand{\inlinehint}{m}{%
  \textcolor{secondary}{\sffamily\footnotesize #1}%
}

\NewDocumentCommand{\MathNotePaletteSwatch}{mm}{%
  \tikz[baseline=(label.base)]{
    \node[rounded corners=2pt, draw=#1!65!black, fill=#1, minimum width=0.85cm, minimum height=0.4cm] (chip) {};
    \node[right=0.28cm of chip, anchor=west, font=\sffamily\scriptsize\color{inkgray}] (label) {#2};
  }%
}

\NewDocumentCommand{\ModeBadge}{O{accent}m}{%
  \tikz[baseline=(label.base)]\node[label/.style={}] (label) [inner xsep=6pt, inner ysep=1.6pt, rounded corners=2pt, fill=#1!12!white, draw=#1!80!black, font=\sffamily\scriptsize\bfseries\color{#1!30!black}] {#2};%
}

\newcommand{\mathnote@ifblank}[3]{%
  \if\relax\detokenize{#1}\relax
    #2%
  \else
    #3%
  \fi
}

\newenvironment{focuspoints}{%
  \begin{itemize}[label=\tikz{\filldraw[accent] (0,0) circle (2pt);}, leftmargin=1.8em, itemsep=0.2em, topsep=0.1em]
}{\end{itemize}}

\newcounter{roadmapstep}
\newlength{\mathnote@roadmapindent}
\setlength{\mathnote@roadmapindent}{1.4em}
\newcommand{\mathnote@roadmaparrow}{%
  \par\smallskip
  \noindent\hspace{1.7em}\tikz{
    \draw[accent, line width=0.85pt, -{Latex[length=3mm]}] (0,0) -- (0,-0.9);
  }%
  \par\smallskip
}
\NewDocumentEnvironment{roadmap}{O{}}{%
  \par\smallskip
  \setcounter{roadmapstep}{0}%
}{%
  \par\smallskip
}
\newcommand{\RoadmapStep}[1]{%
  \stepcounter{roadmapstep}%
  \ifnum\value{roadmapstep}>1
    \mathnote@roadmaparrow
  \fi
  {%
    \noindent\parfillskip=0pt plus 1fil\ModeBadge[accent]{第\zhnumber{\value{roadmapstep}}步}\par
    \vspace{0.2em}%
    \noindent\hspace{\mathnote@roadmapindent}%
    \begin{minipage}[t]{\dimexpr\linewidth-\mathnote@roadmapindent\relax}
      \raggedright\sloppy #1
    \end{minipage}\par
  }%
}

% --------------------------------------------------
% Box styles
% --------------------------------------------------
\tcbset{
  mathnote box/.style={
    enhanced,
    sharp corners,
    boxrule=0.5pt,
    colback=surface,
    coltitle=inkgray,
    fonttitle=\sffamily\bfseries,
    left=1em,
    right=1em,
    top=0.7em,
    bottom=0.7em,
    before skip=10pt,
    after skip=10pt,
    breakable,
    width=\dimexpr\linewidth-\mathnote@boxindent\relax,
    left skip=\mathnote@boxindent,
    borderline west={1pt}{0pt}{accentline}
  }
}
\newtcolorbox{definitionbox}[2][]{%
  mathnote box,
  title=\mathnote@ifblank{#2}{定义}{#2},
  colback=surface,
  colframe=secondary!70!black,
  coltitle=secondary!15!white,
  fonttitle=\sffamily\bfseries\color{secondary!35!white},
  borderline west={2pt}{0pt}{secondary},
  #1
}

\newtcolorbox{theorembox}[2][]{%
  mathnote box,
  title=\mathnote@ifblank{#2}{定理}{#2},
  colback=surface,
  colframe=accent!70!black,
  coltitle=accent!10!white,
  fonttitle=\sffamily\bfseries\color{accent!35!white},
  borderline west={2pt}{0pt}{accent},
  #1
}

\newtcolorbox{examplebox}[2][]{%
  mathnote box,
  title=\mathnote@ifblank{#2}{例题}{#2},
  colback=surface,
  colframe=highlight!80!black,
  coltitle=highlight!15!white,
  fonttitle=\sffamily\bfseries\color{highlight!40!white},
  borderline west={2pt}{0pt}{highlight},
  #1
}

\newtcolorbox{lemmabox}[2][]{%
  mathnote box,
  title=\mathnote@ifblank{#2}{引理}{#2},
  colback=surface,
  colframe=inkline,
  coltitle=inkgray!30!white,
  fonttitle=\sffamily\bfseries\color{inkgray!45!white},
  borderline west={2pt}{0pt}{inkgray},
  #1
}

\newtcolorbox{notebox}[2][]{%
  mathnote box,
  title=\mathnote@ifblank{#2}{提示}{#2},
  colback=surface,
  colframe=highlight!60!black,
  coltitle=highlight!20!white,
  fonttitle=\sffamily\bfseries\color{highlight!45!white},
  borderline west={2pt}{0pt}{highlight},
  #1
}

\newtcolorbox{summarybox}[2][]{%
  mathnote box,
  title=\mathnote@ifblank{#2}{总结}{#2},
  colback=surface,
  colframe=accent!20!black,
  borderline west={2pt}{0pt}{accent},
  coltitle=accent!10!white,
  fonttitle=\sffamily\bfseries\color{accent!45!white},
  #1
}

\newtcolorbox{conceptbox}[2][]{%
  mathnote box,
  title=\mathnote@ifblank{#2}{概念骨架}{#2},
  colback=surface,
  colframe=secondary!40!black,
  coltitle=secondary!15!white,
  fonttitle=\sffamily\bfseries\color{secondary!40!white},
  borderline west={2pt}{0pt}{secondary},
  #1
}

\newtcolorbox{proofbox}[2][]{%
  mathnote box,
  title=\mathnote@ifblank{#2}{证明}{#2},
  colback=surface,
  colframe=inkline,
  coltitle=inkgray!35!white,
  fonttitle=\sffamily\bfseries\color{inkgray!60!white},
  borderline west={2pt}{0pt}{inkline},
  #1
}

\newtcolorbox{warningbox}[2][]{%
  mathnote box,
  title=\mathnote@ifblank{#2}{排版警示}{#2},
  colback=surface,
  colframe=highlight!80!black,
  borderline west={2pt}{0pt}{highlight},
  coltitle=highlight!20!white,
  fonttitle=\sffamily\bfseries\color{highlight!45!white},
  #1
}

% --------------------------------------------------
% TikZ styles
% --------------------------------------------------
\tikzset{
  mathnote lines/.style={
    line width=0.8pt,
    >=Stealth,
    draw=accentline,
    text=inkgray
  },
  mathnote grid/.style={
    color=inkgray!30,
    line width=0.3pt
  }
}

% --------------------------------------------------
% Tables and lists
% --------------------------------------------------
\newcommand{\mathnote@listbarbegin}[2][0.8em]{%
  \par\noindent
  \begin{tcolorbox}[
    blanker,
    enhanced,
    sharp corners,
    boxrule=0pt,
    colback=surface,
    left=#1,
    right=0pt,
    top=0.25em,
    bottom=0.25em,
    borderline west={1.3pt}{0pt}{#2}
  ]%
  \ignorespaces
}
\newcommand{\mathnote@listbarend}{\end{tcolorbox}\ignorespacesafterend}

\setlist[itemize]{leftmargin=1.8em, itemsep=0.25em, before=\mathnote@listbarbegin{accent}, after=\mathnote@listbarend}
\setlist[enumerate]{leftmargin=2.1em, itemsep=0.3em, label=\textbf{\arabic*.}, before=\mathnote@listbarbegin[1em]{secondary}, after=\mathnote@listbarend}
\setlist[description]{font=\sffamily\bfseries, labelsep=0.5em}

\renewcommand{\arraystretch}{1.2}
\captionsetup{font=small, labelfont=bf}

% --------------------------------------------------
% Utility commands
% --------------------------------------------------
\newcommand{\ScreenOnly}[1]{\ifmathnoteprintmode\else #1\fi}
\newcommand{\PrintOnly}[1]{\ifmathnoteprintmode #1\fi}
\newcommand{\DualMode}[2]{\ifmathnoteprintmode #2\else #1\fi}

\NewDocumentCommand{\PageTag}{O{accent}m}{%
  \begin{tikzpicture}[remember picture, overlay]
    \node[anchor=north east, xshift=-6mm, yshift=-10mm, fill=#1, text=white, rounded corners=2pt, inner xsep=6pt, inner ysep=2pt, font=\sffamily\footnotesize] at (current page.north east) {#2};
  \end{tikzpicture}%
}

\newcommand{\SectionTag}[1]{%
  \textcolor{accent}{\Large\bfseries\sffamily #1}%
}

\makeatother


\setCJKmainfont{SimSun}
\setCJKsansfont{SimHei}
\setCJKmonofont{FangSong}
\geometry{left=2.5cm, right=2.5cm, top=2.5cm, bottom=2.5cm}
\pgfplotsset{compat=1.18}

\hypersetup{
    colorlinks=true,
    linkcolor=blue,
    filecolor=magenta,
    urlcolor=cyan,
    pdfborder={0 0 0}
}

\pagestyle{fancy}
\fancyhf{}
\fancyhead[L]{高中数学参考笔记}
\fancyhead[R]{必要性探路}
\fancyfoot[C]{\thepage}

\titleformat{\section}{\Large\bfseries\color{blue!80!black}}{\thesection}{1em}{}
\titleformat{\subsection}{\large\bfseries\color{green!70!black}}{\thesubsection}{1em}{}
\titleformat{\subsubsection}{\bfseries\color{red!70!black}}{\thesubsubsection}{1em}{}

% 定义颜色方案(兼容黑白打印)
\definecolor{mainblue}{RGB}{0, 102, 204}
\definecolor{accentgreen}{RGB}{0, 153, 76}
\definecolor{accentred}{RGB}{204, 0, 51}
\definecolor{textgray}{RGB}{64, 64, 64}
\definecolor{lightgray}{RGB}{230, 230, 230}

% 定理环境
\theoremstyle{definition}
\newtheorem{definition}{定义}[section]
\newtheorem{theorem}{定理}[section]
\newtheorem{lemma}{引理}[section]
\newtheorem{corollary}{推论}[section]
\newtheorem{proposition}{命题}[section]

\theoremstyle{remark}
\newtheorem{remark}{注记}[section]
\newtheorem{example}{例题}[section]
\newtheorem{exercise}{练习}[section]

% 自定义命令
\newcommand{\highlight}[1]{\textcolor{accentred}{\textbf{#1}}}
\newcommand{\method}[1]{\textcolor{mainblue}{\textbf{#1}}}
\newcommand{\keypoint}[1]{\textcolor{accentgreen}{\textbf{#1}}}
\newcommand{\warning}[1]{\textcolor{accentred}{\textbf{注意:#1}}}

\renewcommand{\notetitle}{必要性探路方法在高中数学中的应用}
\renewcommand{\noteauthor}{高中数学参考笔记}
\renewcommand{\notedate}{\today}

\title{\textbf{\Large \notetitle}\\
\large 综合性·指导性·详细性·复习式·工具与应用风格}
\author{\noteauthor}
\date{\notedate}

\begin{document}

\maketitle

% 目录
\tableofcontents
\newpage

% 前言
\section*{前言}
\addcontentsline{toc}{section}{前言}

本笔记以"必要性探路"为核心,系统梳理高中数学中这一重要解题策略的理论基础、方法技巧和实际应用。必要性探路是一种先寻找必要条件,再验证充分性的解题方法,在不等式证明、函数分析、数列问题、解析几何和三角函数等领域具有广泛应用。

本笔记采用彩色设计,重要概念和方法用不同颜色标注,同时确保黑白打印时的清晰度。每个知识点都配有详细的例题和解析,旨在帮助读者深入理解并掌握这一重要的数学解题方法。

\newpage

% 第一章:理论基础
\section{必要性探路的理论基础}

\subsection{基本概念}

\begin{definition}[命题与条件]
设 $p$ 和 $q$ 是两个命题:
\begin{itemize}
    \item 如果 $p \Rightarrow q$ 为真,则称 $p$ 是 $q$ 的\highlight{充分条件},$q$ 是 $p$ 的\highlight{必要条件}
    \item 如果 $p \Leftrightarrow q$ 为真,则称 $p$ 和 $q$ 互为\highlight{充要条件}
\end{itemize}
\end{definition}

\begin{definition}[必要性探路]
\method{必要性探路}是一种解题策略,其核心思想是:
\begin{enumerate}
    \item 先分析问题成立的必要条件
    \item 通过必要条件缩小讨论范围
    \item 验证这些条件是否充分
    \item 综合推理得出最终结论
\end{enumerate}
\end{definition}

\subsection{逻辑基础}

\begin{theorem}[必要条件的作用]
如果 $p$ 是 $q$ 的必要条件,那么 $q$ 为真时,$p$ 必然为真。因此,通过分析 $p$ 为真时的条件,可以缩小 $q$ 为真时的可能范围。
\end{theorem}

\begin{example}[基本应用]
证明:对于任意实数 $x$,有 $x^2 \geq 0$。

\textbf{分析}:要证明 $x^2 \geq 0$,我们需要分析其成立的必要条件。

\textbf{解}:
\begin{enumerate}
    \item \keypoint{寻找必要条件}:由于平方数的性质,任何实数的平方都非负
    \item \keypoint{验证充分性}:对于任意实数 $x$,$x^2 \geq 0$ 恒成立
    \item \keypoint{得出结论}:原不等式成立
\end{enumerate}
\end{example}

\subsection{方法步骤}

\begin{method}[必要性探路的通用步骤]
\begin{enumerate}
    \item \textbf{分析问题}:明确题目要求证明或求解的结论
    \item \textbf{寻找必要条件}:思考哪些条件是结论成立所必需的
    \item \textbf{验证充分性}:检查这些必要条件是否足以推出结论
    \item \textbf{综合推理}:结合必要条件和充分性验证,完成证明或求解
\end{enumerate}
\end{method}

\newpage

% 第二章:方法技巧
\section{必要性探路的方法技巧}

\subsection{特殊值选取策略}

\begin{definition}[特殊值选取原则]
在必要性探路中,选择合适的特殊值至关重要:
\begin{itemize}
    \item \highlight{端点值}:对于闭区间上的函数,选择区间端点
    \item \highlight{对称点}:对于对称函数,选择对称中心或对称轴上的点
    \item \highlight{特殊角度}:对于三角函数,选择 $0°, 30°, 45°, 60°, 90°$ 等
    \item \highlight{特殊数值}:对于指数、对数函数,选择 $0, 1, e$ 等
\end{itemize}
\end{definition}

\begin{example}[端点值应用]
已知函数 $f(x) = ax^2 + bx + c$ 在区间 $[0, 1]$ 上恒大于 $0$,求参数 $a$ 的取值范围。

\textbf{解}:
\begin{enumerate}
    \item \keypoint{选取特殊值}:选择端点 $x = 0$ 和 $x = 1$
    \item \keypoint{建立必要条件}:
    \begin{align}
    f(0) = c > 0 \\
    f(1) = a + b + c > 0
    \end{align}
    \item \keypoint{进一步分析}:还需要考虑函数在区间内的最小值
    \item \keypoint{验证充分性}:在所得范围内验证函数确实恒大于 $0$
\end{enumerate}
\end{example}

\subsection{参数范围确定}

\begin{method}[参数范围确定技巧]
\begin{enumerate}
    \item \textbf{边界分析}:通过特殊值确定参数的边界
    \item \textbf{单调性分析}:利用函数的单调性确定参数范围
    \item \textbf{极值分析}:通过极值点确定参数的临界值
    \item \textbf{综合验证}:结合多种方法验证参数范围的正确性
\end{enumerate}
\end{method}

\subsection{充分性验证方法}

\begin{theorem}[充分性验证的重要性]
通过特殊值得到的参数范围只是必要条件,必须进一步验证其充分性,确保在所得范围内原问题确实成立。
\end{theorem}

\begin{example}[充分性验证]
对于不等式 $ax^2 + bx + c \geq 0$ 对所有实数 $x$ 成立,通过 $x = 0$ 得到 $c \geq 0$,但这只是必要条件。还需要验证 $a > 0$ 且判别式 $\Delta \leq 0$。
\end{example}

\newpage

% 第三章:不等式证明中的应用
\section{不等式证明中的必要性探路}

\subsection{基本不等式}

\begin{example}[算术-几何平均不等式]
证明:对于任意正数 $a, b$,有 $\frac{a + b}{2} \geq \sqrt{ab}$。

\textbf{解}:
\begin{enumerate}
    \item \keypoint{分析问题}:要证明 $\frac{a + b}{2} \geq \sqrt{ab}$
    \item \keypoint{寻找必要条件}:考虑平方差公式 $(a - b)^2 \geq 0$
    \item \keypoint{展开分析}:$(a - b)^2 = a^2 - 2ab + b^2 \geq 0$,即 $a^2 + b^2 \geq 2ab$
    \item \keypoint{两边加 $2ab$}:$a^2 + 2ab + b^2 \geq 4ab$,即 $(a + b)^2 \geq 4ab$
    \item \keypoint{取平方根}:$a + b \geq 2\sqrt{ab}$,即 $\frac{a + b}{2} \geq \sqrt{ab}$
    \item \keypoint{验证充分性}:当且仅当 $a = b$ 时等号成立
\end{enumerate}
\end{example}

\subsection{柯西不等式}

\begin{theorem}[柯西不等式]
对于任意实数 $a_1, a_2, \ldots, a_n$ 和 $b_1, b_2, \ldots, b_n$,有:
$$(a_1^2 + a_2^2 + \cdots + a_n^2)(b_1^2 + b_2^2 + \cdots + b_n^2) \geq (a_1b_1 + a_2b_2 + \cdots + a_nb_n)^2$$
\end{theorem}

\begin{example}[柯西不等式的证明]
\textbf{解}:
\begin{enumerate}
    \item \keypoint{分析问题}:要证明柯西不等式
    \item \keypoint{寻找必要条件}:考虑向量内积的性质
    \item \keypoint{向量方法}:设向量 $\vec{a} = (a_1, a_2, \ldots, a_n)$,$\vec{b} = (b_1, b_2, \ldots, b_n)$
    \item \keypoint{内积不等式}:$|\vec{a} \cdot \vec{b}| \leq |\vec{a}| \cdot |\vec{b}|$
    \item \keypoint{展开计算}:得到柯西不等式
    \item \keypoint{验证充分性}:当向量共线时等号成立
\end{enumerate}
\end{example}

\subsection{含参数不等式}

\begin{example}[含参数不等式恒成立]
已知不等式 $ax^2 + 2x + 1 > 0$ 对所有实数 $x$ 成立,求参数 $a$ 的取值范围。

\textbf{解}:
\begin{enumerate}
    \item \keypoint{选取特殊值}:选择 $x = 0$,得 $1 > 0$(恒成立)
    \item \keypoint{分析二次函数}:当 $a = 0$ 时,不等式为 $2x + 1 > 0$,不恒成立
    \item \keypoint{必要条件}:$a > 0$ 且判别式 $\Delta = 4 - 4a < 0$
    \item \keypoint{求解范围}:$a > 1$
    \item \keypoint{验证充分性}:当 $a > 1$ 时,二次函数开口向上且无实根,确实恒大于 $0$
\end{enumerate}
\end{example}

\newpage

% 第四章:函数与导数中的应用
\section{函数与导数中的必要性探路}

\subsection{函数单调性}

\begin{example}[函数单调性判定]
设函数 $f(x) = x^3 - 3x + 1$,求其单调区间。

\textbf{解}:
\begin{enumerate}
    \item \keypoint{求导数}:$f'(x) = 3x^2 - 3 = 3(x^2 - 1) = 3(x - 1)(x + 1)$
    \item \keypoint{寻找临界点}:令 $f'(x) = 0$,得 $x = \pm 1$
    \item \keypoint{分析符号}:
    \begin{itemize}
        \item 当 $x < -1$ 时,$f'(x) > 0$,函数单调递增
        \item 当 $-1 < x < 1$ 时,$f'(x) < 0$,函数单调递减
        \item 当 $x > 1$ 时,$f'(x) > 0$,函数单调递增
    \end{itemize}
    \item \keypoint{得出结论}:函数在 $(-\infty, -1)$ 和 $(1, +\infty)$ 上单调递增,在 $(-1, 1)$ 上单调递减
\end{enumerate}
\end{example}

\subsection{极值问题}

\begin{example}[极值点求解]
求函数 $g(x) = x^4 - 4x^2 + 4$ 的极值点。

\textbf{解}:
\begin{enumerate}
    \item \keypoint{求导数}:$g'(x) = 4x^3 - 8x = 4x(x^2 - 2) = 4x(x - \sqrt{2})(x + \sqrt{2})$
    \item \keypoint{寻找临界点}:令 $g'(x) = 0$,得 $x = 0, \pm\sqrt{2}$
    \item \keypoint{二阶导数检验}:$g''(x) = 12x^2 - 8$
    \begin{itemize}
        \item $g''(0) = -8 < 0$,所以 $x = 0$ 是极大值点
        \item $g''(\pm\sqrt{2}) = 16 > 0$,所以 $x = \pm\sqrt{2}$ 是极小值点
    \end{itemize}
    \item \keypoint{计算极值}:
    \begin{itemize}
        \item 极大值:$g(0) = 4$
        \item 极小值:$g(\pm\sqrt{2}) = 0$
    \end{itemize}
\end{enumerate}
\end{example}

\subsection{恒成立问题}

\begin{example}[恒成立问题]
已知函数 $f(x) = \ln(ax + 1) + 1 - \frac{x}{1 + x}$,若 $f(x) \geq \ln 2$ 在 $[1, +\infty)$ 上恒成立,求参数 $a$ 的取值范围。

\textbf{解}:
\begin{enumerate}
    \item \keypoint{选取特殊值}:选择 $x = 1$,得 $f(1) = \ln(a + 1) + 1 - \frac{1}{2} = \ln(a + 1) + \frac{1}{2}$
    \item \keypoint{建立必要条件}:$f(1) \geq \ln 2$,即 $\ln(a + 1) + \frac{1}{2} \geq \ln 2$
    \item \keypoint{求解不等式}:$\ln(a + 1) \geq \ln 2 - \frac{1}{2} = \ln(2e^{-\frac{1}{2}})$
    \item \keypoint{得到范围}:$a + 1 \geq 2e^{-\frac{1}{2}}$,即 $a \geq 2e^{-\frac{1}{2}} - 1$
    \item \keypoint{验证充分性}:在所得范围内验证函数确实满足条件
\end{enumerate}
\end{example}

\newpage

% 第五章:数列问题中的应用
\section{数列问题中的必要性探路}

\subsection{等差数列}

\begin{example}[等差数列性质]
已知等差数列 $\{a_n\}$ 的首项为 $a_1$,公差为 $d$,求其前 $n$ 项和 $S_n$。

\textbf{解}:
\begin{enumerate}
    \item \keypoint{分析问题}:要求等差数列的前 $n$ 项和公式
    \item \keypoint{寻找必要条件}:等差数列的定义和性质
    \item \keypoint{通项公式}:$a_n = a_1 + (n-1)d$
    \item \keypoint{求和公式推导}:
    \begin{align}
    S_n &= a_1 + a_2 + \cdots + a_n \\
    &= a_1 + (a_1 + d) + (a_1 + 2d) + \cdots + [a_1 + (n-1)d] \\
    &= na_1 + d[1 + 2 + \cdots + (n-1)] \\
    &= na_1 + d \cdot \frac{(n-1)n}{2} \\
    &= \frac{n}{2}[2a_1 + (n-1)d]
    \end{align}
    \item \keypoint{验证充分性}:公式适用于所有等差数列
\end{enumerate}
\end{example}

\subsection{等比数列}

\begin{example}[等比数列性质]
已知等比数列 $\{b_n\}$ 的首项为 $b_1$,公比为 $q$,求其前 $n$ 项和 $S_n$。

\textbf{解}:
\begin{enumerate}
    \item \keypoint{分析问题}:要求等比数列的前 $n$ 项和公式
    \item \keypoint{寻找必要条件}:等比数列的定义和性质
    \item \keypoint{通项公式}:$b_n = b_1 q^{n-1}$
    \item \keypoint{求和公式推导}:
    \begin{align}
    S_n &= b_1 + b_1q + b_1q^2 + \cdots + b_1q^{n-1} \\
    qS_n &= b_1q + b_1q^2 + \cdots + b_1q^{n-1} + b_1q^n
    \end{align}
    两式相减得:$(1-q)S_n = b_1(1-q^n)$
    \item \keypoint{得到公式}:当 $q \neq 1$ 时,$S_n = b_1 \cdot \frac{1-q^n}{1-q}$
    \item \keypoint{特殊情况}:当 $q = 1$ 时,$S_n = nb_1$
\end{enumerate}
\end{example}

\subsection{递推数列}

\begin{example}[递推数列求解]
已知数列 $\{c_n\}$ 满足 $c_1 = 1$,$c_{n+1} = 2c_n + 1$,求通项公式。

\textbf{解}:
\begin{enumerate}
    \item \keypoint{分析递推关系}:$c_{n+1} = 2c_n + 1$
    \item \keypoint{寻找必要条件}:需要将递推关系转化为可求解的形式
    \item \keypoint{构造辅助数列}:设 $d_n = c_n + 1$,则 $d_{n+1} = 2d_n$
    \item \keypoint{求解辅助数列}:$d_n = d_1 \cdot 2^{n-1} = 2 \cdot 2^{n-1} = 2^n$
    \item \keypoint{得到原数列}:$c_n = d_n - 1 = 2^n - 1$
    \item \keypoint{验证充分性}:检验 $c_1 = 1$ 和递推关系
\end{enumerate}
\end{example}

\newpage

% 第六章:解析几何中的应用
\section{解析几何中的必要性探路}

\subsection{直线与圆}

\begin{example}[直线与圆的位置关系]
求直线 $y = kx + b$ 与圆 $(x-a)^2 + (y-b)^2 = r^2$ 相交的条件。

\textbf{解}:
\begin{enumerate}
    \item \keypoint{分析问题}:要求直线与圆相交的条件
    \item \keypoint{联立方程}:将直线方程代入圆方程
    \item \keypoint{得到二次方程}:$(x-a)^2 + (kx+b-b)^2 = r^2$
    即:$(1+k^2)x^2 + 2(a+kb)x + a^2 + b^2 - r^2 = 0$
    \item \keypoint{判别式分析}:$\Delta = 4(a+kb)^2 - 4(1+k^2)(a^2+b^2-r^2)$
    \item \keypoint{相交条件}:$\Delta > 0$,即 $|ka-b| < r\sqrt{1+k^2}$
    \item \keypoint{几何意义}:圆心到直线的距离小于半径
\end{enumerate}
\end{example}

\subsection{圆锥曲线}

\begin{example}[椭圆的性质]
已知椭圆 $\frac{x^2}{a^2} + \frac{y^2}{b^2} = 1$,求其焦点坐标。

\textbf{解}:
\begin{enumerate}
    \item \keypoint{分析椭圆方程}:标准椭圆方程
    \item \keypoint{寻找必要条件}:椭圆的定义和性质
    \item \keypoint{焦点性质}:椭圆上任意一点到两焦点距离之和为 $2a$
    \item \keypoint{焦点坐标}:$c = \sqrt{a^2 - b^2}$,焦点为 $(\pm c, 0)$
    \item \keypoint{验证充分性}:验证焦点确实满足椭圆的性质
\end{enumerate}
\end{example}

\subsection{最值问题}

\begin{example}[距离最值问题]
求点 $P(2, 3)$ 到直线 $3x + 4y - 5 = 0$ 的距离。

\textbf{解}:
\begin{enumerate}
    \item \keypoint{分析问题}:求点到直线的距离
    \item \keypoint{距离公式}:$d = \frac{|Ax_0 + By_0 + C|}{\sqrt{A^2 + B^2}}$
    \item \keypoint{代入计算}:$d = \frac{|3 \cdot 2 + 4 \cdot 3 - 5|}{\sqrt{3^2 + 4^2}} = \frac{|6 + 12 - 5|}{\sqrt{25}} = \frac{13}{5}$
    \item \keypoint{验证结果}:距离为正数,符合几何意义
\end{enumerate}
\end{example}

\newpage

% 第七章:三角函数中的应用
\section{三角函数中的必要性探路}

\subsection{三角恒等式}

\begin{example}[基本恒等式]
证明:$\sin^2 x + \cos^2 x = 1$。

\textbf{解}:
\begin{enumerate}
    \item \keypoint{分析问题}:要证明基本的三角恒等式
    \item \keypoint{寻找必要条件}:利用单位圆的性质
    \item \keypoint{单位圆方法}:在单位圆上,任意一点 $(x, y)$ 满足 $x^2 + y^2 = 1$
    \item \keypoint{三角函数定义}:$x = \cos \theta$,$y = \sin \theta$
    \item \keypoint{代入得到}:$\cos^2 \theta + \sin^2 \theta = 1$
    \item \keypoint{验证充分性}:对所有角度都成立
\end{enumerate}
\end{example}

\subsection{三角不等式}

\begin{example}[三角不等式]
证明:对于任意角度 $x$,有 $|\sin x| \leq 1$。

\textbf{解}:
\begin{enumerate}
    \item \keypoint{分析问题}:要证明正弦函数的有界性
    \item \keypoint{寻找必要条件}:利用基本恒等式
    \item \keypoint{基本恒等式}:$\sin^2 x + \cos^2 x = 1$
    \item \keypoint{分析 $\sin^2 x$}:由于 $\cos^2 x \geq 0$,所以 $\sin^2 x \leq 1$
    \item \keypoint{取平方根}:$|\sin x| \leq 1$
    \item \keypoint{验证充分性}:当 $x = \frac{\pi}{2} + k\pi$ 时等号成立
\end{enumerate}
\end{example}

\subsection{三角方程}

\begin{example}[三角方程求解]
求解方程 $\sin x = \frac{1}{2}$。

\textbf{解}:
\begin{enumerate}
    \item \keypoint{分析方程}:$\sin x = \frac{1}{2}$
    \item \keypoint{寻找必要条件}:利用正弦函数的性质
    \item \keypoint{特殊角值}:$\sin \frac{\pi}{6} = \frac{1}{2}$
    \item \keypoint{周期性}:$\sin x = \sin(\frac{\pi}{6})$ 的解为 $x = \frac{\pi}{6} + 2k\pi$ 或 $x = \frac{5\pi}{6} + 2k\pi$
    \item \keypoint{验证解}:代入原方程验证
\end{enumerate}
\end{example}

\newpage

% 第八章:综合应用
\section{综合应用与高考真题}

\subsection{跨领域综合应用}

\begin{example}[函数与不等式综合]
已知函数 $f(x) = e^x - ax - 1$,若 $f(x) \geq 0$ 对所有实数 $x$ 成立,求参数 $a$ 的取值范围。

\textbf{解}:
\begin{enumerate}
    \item \keypoint{分析问题}:指数函数与线性函数的组合
    \item \keypoint{选取特殊值}:选择 $x = 0$,得 $f(0) = 1 - 0 - 1 = 0$
    \item \keypoint{必要条件}:$f(0) \geq 0$ 恒成立
    \item \keypoint{求导数}:$f'(x) = e^x - a$
    \item \keypoint{极值分析}:令 $f'(x) = 0$,得 $x = \ln a$
    \item \keypoint{最小值条件}:$f(\ln a) = a - a\ln a - 1 \geq 0$
    \item \keypoint{求解范围}:通过分析得到 $a \leq 1$
    \item \keypoint{验证充分性}:当 $a \leq 1$ 时,函数确实非负
\end{enumerate}
\end{example}

\subsection{高考真题解析}

\begin{example}[2023年高考真题]
已知函数 $f(x) = \ln(1+x) - \frac{ax}{1+x}$,若 $f(x) \geq 0$ 在 $[0,+\infty)$ 上恒成立,求实数 $a$ 的取值范围。

\textbf{解}:
\begin{enumerate}
    \item \keypoint{选取特殊值}:选择 $x = 0$,得 $f(0) = 0$
    \item \keypoint{求导数}:$f'(x) = \frac{1}{1+x} - \frac{a(1+x) - ax}{(1+x)^2} = \frac{1-a}{(1+x)^2}$
    \item \keypoint{单调性分析}:
    \begin{itemize}
        \item 当 $a \leq 1$ 时,$f'(x) \geq 0$,函数单调递增
        \item 当 $a > 1$ 时,$f'(x) < 0$,函数单调递减
    \end{itemize}
    \item \keypoint{必要条件}:$a \leq 1$
    \item \keypoint{验证充分性}:当 $a \leq 1$ 时,$f(x) \geq f(0) = 0$
    \item \keypoint{最终答案}:$a \in (-\infty, 1]$
\end{enumerate}
\end{example}

\newpage

% 第九章:练习巩固
\section{练习巩固}

\subsection{基础练习}

\begin{exercise}[不等式证明]
证明:对于任意正数 $a, b, c$,有 $\frac{a+b+c}{3} \geq \sqrt[3]{abc}$。
\end{exercise}

\begin{exercise}[函数单调性]
求函数 $f(x) = x^3 - 6x^2 + 9x + 1$ 的单调区间。
\end{exercise}

\begin{exercise}[数列问题]
已知数列 $\{a_n\}$ 满足 $a_1 = 1$,$a_{n+1} = 2a_n + 1$,求通项公式。
\end{exercise}

\begin{exercise}[解析几何]
求点 $P(1, 2)$ 到直线 $2x - y + 3 = 0$ 的距离。
\end{exercise}

\begin{exercise}[三角函数]
求解方程 $\cos 2x = \frac{1}{2}$。
\end{exercise}

\subsection{提高练习}

\begin{exercise}[综合应用]
已知函数 $f(x) = x^2 + ax + b$,若 $f(x) \geq 0$ 对所有实数 $x$ 成立,求 $a, b$ 满足的条件。
\end{exercise}

\begin{exercise}[参数问题]
已知不等式 $x^2 + 2ax + 1 > 0$ 对所有实数 $x$ 成立,求参数 $a$ 的取值范围。
\end{exercise}

\begin{exercise}[最值问题]
求函数 $f(x) = x^4 - 2x^2 + 1$ 在区间 $[-2, 2]$ 上的最大值和最小值。
\end{exercise}

\subsection{答案与解析}

\begin{remark}[练习答案]
\begin{enumerate}
    \item 利用算术-几何平均不等式
    \item 求导数,分析符号变化
    \item 构造辅助数列求解
    \item 利用点到直线距离公式
    \item 利用二倍角公式求解
    \item 利用判别式条件
    \item 利用判别式小于零
    \item 求导数,分析极值点
\end{enumerate}
\end{remark}

\newpage

% 总结
\section*{总结}
\addcontentsline{toc}{section}{总结}

本笔记系统介绍了"必要性探路"这一重要的数学解题策略,从理论基础到实际应用,涵盖了高中数学的各个主要领域。通过大量的例题和练习,帮助读者深入理解并掌握这一方法。

\keypoint{核心要点}:
\begin{itemize}
    \item 必要性探路是一种先寻找必要条件,再验证充分性的解题方法
    \item 特殊值的选取是成功应用该方法的关键
    \item 充分性验证是确保答案正确性的重要步骤
    \item 该方法在不等式、函数、数列、几何、三角等领域都有广泛应用
\end{itemize}

\method{学习建议}:
\begin{itemize}
    \item 熟练掌握基本概念和逻辑关系
    \item 通过大量练习提高特殊值选取的能力
    \item 注重充分性验证,避免遗漏
    \item 结合其他解题方法,形成完整的解题策略
\end{itemize}

希望本笔记能够帮助读者更好地理解和掌握"必要性探路"这一重要的数学解题方法,在学习和考试中取得更好的成绩。

\end{document}
