% mathnote-structure.tex
% 段落风格、章节标题样式以及页眉页脚等文档结构相关设置。

% Block-style paragraph headings to avoid run-in overfull boxes
\renewcommand{\paragraph}{%
  \@startsection{paragraph}{4}{\z@}%
    {1.5ex \@plus 0.5ex \@minus 0.2ex}%
    {0.65em}%
    {\normalfont\normalsize\bfseries}%
}

% --------------------------------------------------
% Sectioning and spacing
% --------------------------------------------------
\setlength{\parskip}{0.35em}
\setlength{\parindent}{2em}
\newlength{\mathnote@boxindent}
\setlength{\mathnote@boxindent}{\parindent}
\setcounter{secnumdepth}{3}
\setcounter{tocdepth}{2}

% 若未加载 ctex(如 article 等),使用安全的西文标题回退方案
\newcommand{\mathnote@setupsectionsfallback}{%
  \renewcommand{\section}{%
    \@startsection{section}{1}{\z@}%
      {1.2em}%
      {0.7em}%
      {\normalfont\Large\sffamily\bfseries\color{accent}}%
  }%
  \renewcommand{\subsection}{%
    \@startsection{subsection}{2}{\z@}%
      {1em}%
      {0.4em}%
      {\normalfont\large\sffamily\bfseries\color{secondary}}%
  }%
  \renewcommand{\subsubsection}{%
    \@startsection{subsubsection}{3}{\z@}%
      {0.8em}%
      {0.2em}%
      {\normalfont\normalsize\sffamily\bfseries\color{inkgray}}%
  }%
}
\@ifundefined{ctexset}{%
  % 未加载 ctex,启用简单回退
  \mathnote@setupsectionsfallback
}{%
  % 已加载 ctex,则通过 ctexset 设定彩色章节标题
  \ctexset{
    section={
      name={第,节},
      format+=\Large\sffamily\bfseries\color{accent},
      beforeskip=1.2em,
      afterskip=0.7em
    },
    subsection={
      format+=\large\sffamily\bfseries\color{secondary},
      beforeskip=1em,
      afterskip=0.4em
    },
    subsubsection={
      format+=\normalsize\sffamily\bfseries\color{inkgray},
      beforeskip=0.8em,
      afterskip=0.2em
    }
  }
}

% --------------------------------------------------
% Header / footer
% --------------------------------------------------
\newcommand{\mathnote@headertext}{\small\sffamily\textcolor{accent}{\notetitle\mathnote@maybeversion\ >\ \nouppercase{\rightmark}}}
\newcommand{\mathnote@footertext}{\small\sffamily\textcolor{inkgray}{\thepage}\ \textcolor{mathnotePureCyan}{/}\ \textcolor{inkgray}{\pageref{LastPage}}}
\pagestyle{fancy}
\fancyhf{}
\if@twoside
  \fancyhead[LE]{\mathnote@headertext}
  \fancyhead[RO]{\mathnote@headertext}
  \fancyfoot[LE]{\mathnote@footertext}
  \fancyfoot[RO]{\mathnote@footertext}
\else
  % 单面:页眉页脚左右交替,方便装订
  \fancyhead[L]{\ifodd\value{page}\relax\else \mathnote@headertext\fi}
  \fancyhead[R]{\ifodd\value{page}\mathnote@headertext\fi}
  \fancyfoot[L]{\ifodd\value{page}\relax\else \mathnote@footertext\fi}
  \fancyfoot[R]{\ifodd\value{page}\mathnote@footertext\fi}
\fi
\renewcommand{\headrulewidth}{0.2pt}
\renewcommand{\footrulewidth}{0pt}
\renewcommand{\sectionmark}[1]{\markright{#1}}
\fancypagestyle{plain}{%
  \fancyhf{}
  \fancyfoot[C]{\mathnote@footertext}
  \renewcommand{\headrulewidth}{0pt}
  \renewcommand{\footrulewidth}{0pt}
}

