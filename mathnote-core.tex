% mathnote-core.tex
% 核心配置:编译引擎约束、字体目录自动探测、用户开关、
% 元信息默认值,以及大部分基础宏包加载(字体 / 中文 / 图形等)。
%
% 该文件假定已在 mathnote-preamble.tex 中置于 \makeatletter ... \makeatother 内部,
% 可单独阅读理解逻辑,但不建议直接独立使用。

\newif\ifmathnote@fontdirfound
\mathnote@fontdirfoundfalse
% 尝试在相对路径中查找本地字体目录;若未找到则回退到 fonts/
\def\mathnote@fontdircandidates{fonts/,./fonts/,../fonts/,../../fonts/,../../../fonts/}
\def\mathnote@fontdir{}
\@for\mathnote@cand:=\mathnote@fontdircandidates\do{%
  \ifmathnote@fontdirfound\else
    \IfFileExists{\mathnote@cand NotoSerif-VF.ttf}{%
      \edef\mathnote@fontdir{\mathnote@cand}%
      \mathnote@fontdirfoundtrue
    }{%
      \IfFileExists{\mathnote@cand SourceHanSerifSC-Regular.otf}{%
        \edef\mathnote@fontdir{\mathnote@cand}%
        \mathnote@fontdirfoundtrue
      }{}%
    }%
  \fi
}
\ifmathnote@fontdirfound\else
  \edef\mathnote@fontdir{fonts/}%
\fi
% 若用户未显式定义 \MathNoteFontDir,则使用自动探测结果
\@ifundefined{MathNoteFontDir}{%
  \edef\MathNoteFontDir{\mathnote@fontdir}%
}{}

% --------------------------------------------------
% User switches —— 面向用户的开关与模式
% --------------------------------------------------
\newif\ifmathnoteprintmode
\mathnoteprintmodefalse
% 屏幕 -> 印刷模式切换接口;颜色逻辑见 mathnote-colors.tex
\newcommand{\MathNoteEnablePrint}{%
  \mathnoteprintmodetrue
  \mathnote@applypalette
}

% 审稿章开关;默认关闭,仅在显式调用时启用
\newif\ifmathnotereviewstamp
\mathnotereviewstampfalse
% 尝试在多级目录中自动寻找审核章图标,兼容根目录与子目录编译
\newif\ifmathnote@reviewstampfound
\mathnote@reviewstampfoundfalse
% 若用户未自定义 \MathNoteReviewStampBase,则留空以使用自动探测
\@ifundefined{MathNoteReviewStampBase}{%
  \newcommand{\MathNoteReviewStampBase}{}%
}{}
\newcommand{\mathnote@reviewstamppathbase}{}
\def\mathnote@reviewstampdircandidates{assest/,./assest/,../assest/,../../assest/,../../../assest/,../../../../assest/}
\@ifundefined{mathnote@rootpath}{}{%
  \edef\mathnote@reviewstampdircandidates{%
    \mathnote@rootpath\detokenize{assest/},%
    \mathnote@rootpath\detokenize{./assest/},%
    \mathnote@rootpath\detokenize{../assest/},%
    \mathnote@rootpath\detokenize{../../assest/},%
    \mathnote@reviewstampdircandidates
  }%
}
\if\relax\detokenize{\MathNoteReviewStampBase}\relax
  \@for\mathnote@cand:=\mathnote@reviewstampdircandidates\do{%
    \ifmathnote@reviewstampfound\else
      \IfFileExists{\mathnote@cand lzlxV-reviewed.pdf}{%
        \edef\mathnote@reviewstamppathbase{\mathnote@cand lzlxV-reviewed}%
        \mathnote@reviewstampfoundtrue
      }{%
        \IfFileExists{\mathnote@cand lzlxV-reviewed.svg}{%
          \edef\mathnote@reviewstamppathbase{\mathnote@cand lzlxV-reviewed}%
          \mathnote@reviewstampfoundtrue
        }{}%
      }%
    \fi
  }%
\else
  \edef\mathnote@reviewstamppathbase{\MathNoteReviewStampBase}%
  \mathnote@reviewstampfoundtrue
\fi
\newcommand{\mathnote@reviewstampsvg}{\mathnote@reviewstamppathbase.svg}
\newcommand{\mathnote@reviewstamppdf}{\mathnote@reviewstamppathbase.pdf}
\newcommand{\mathnote@reviewstampinclude}{}
\newif\ifmathnote@reviewstampplaced
\mathnote@reviewstampplacedfalse
\IfFileExists{\mathnote@reviewstamppdf}{%
  \renewcommand{\mathnote@reviewstampinclude}{%
    \includegraphics[width=20mm,height=20mm,keepaspectratio]{\mathnote@reviewstamppdf}%
  }%
}{%
  \IfFileExists{\mathnote@reviewstampsvg}{%
    \renewcommand{\mathnote@reviewstampinclude}{%
      \includesvg[width=20mm,height=20mm,keepaspectratio]{\mathnote@reviewstamppathbase}%
    }%
  }{}%
}
% 将审稿章叠加到第一页右上角;若找不到资源则只给出警告
\newcommand{\mathnote@enablereviewstamp}{%
  \ifx\mathnote@reviewstampinclude\@empty
    \PackageWarning{mathnote-preamble}{Review stamp graphic \mathnote@reviewstampsvg\space (or PDF fallback) not found}%
  \else
    \mathnote@reviewstampplacedfalse
    \AddToHook{shipout/foreground}{%
      \ifmathnote@reviewstampplaced\else
        \begin{tikzpicture}[remember picture, overlay]
          \node[anchor=north east, xshift=-8mm, yshift=-8mm] at (current page.north east){%
            \mathnote@reviewstampinclude
          };
        \end{tikzpicture}%
        \global\mathnote@reviewstampplacedtrue
      \fi
    }%
  \fi
}
\newcommand{\MathNoteEnableReviewStamp}{%
  \mathnotereviewstamptrue
  \mathnote@enablereviewstamp
}

% --------------------------------------------------
% Metadata defaults (can be overwritten in each file)
% 文档标题/作者等元信息的默认值
% --------------------------------------------------
\providecommand{\notetitle}{数学学习笔记}
\providecommand{\noteauthor}{作者}
\providecommand{\notedate}{\today}
\providecommand{\notesubtitle}{现代数学排版示例}
\providecommand{\noteversion}{v1.0}
\newcommand{\mathnote@maybeversion}{%
  \if\relax\detokenize{\noteversion}\relax
  \else
    \space(\noteversion)%
  \fi
}

% --------------------------------------------------
% Core packages —— 几何版式、字体系统与常用功能包
% --------------------------------------------------
\usepackage{geometry}
\geometry{
  paper=a4paper,
  top=2.35cm,
  bottom=2.4cm,
  left=2.1cm,
  right=2.1cm,
  headheight=16pt,
  headsep=14pt
}

% 字体与编译引擎抽象
\usepackage{fontspec}
\usepackage{metalogo}
\defaultfontfeatures{Ligatures=TeX, Scale=MatchLowercase}

% 本地字体文件探测辅助宏:与 \MathNoteFontDir 拼接使用
\newcommand{\mathnote@fontfile}[1]{\MathNoteFontDir#1}
\newif\ifmathnote@haslocalserif
\newif\ifmathnote@haslocalsans
\newif\ifmathnote@haslocalmono
\newif\ifmathnote@haslocalcjkserif
\newif\ifmathnote@haslocalcjksans
\newif\ifmathnote@haslocalcjkmono
\newif\ifmathnote@haslocalkai
\newif\ifmathnote@haslocalshserif
\newif\ifmathnote@haslocalshsans
\IfFileExists{\mathnote@fontfile{NotoSerif-VF.ttf}}{\mathnote@haslocalseriftrue}{\mathnote@haslocalseriffalse}
\IfFileExists{\mathnote@fontfile{NotoSans-VF.ttf}}{\mathnote@haslocalsanstrue}{\mathnote@haslocalsansfalse}
\IfFileExists{\mathnote@fontfile{NotoSansMono-VF.ttf}}{\mathnote@haslocalmonotrue}{\mathnote@haslocalmonofalse}
\IfFileExists{\mathnote@fontfile{NotoSerifCJK-VF.ttc}}{\mathnote@haslocalcjkseriftrue}{\mathnote@haslocalcjkseriffalse}
\IfFileExists{\mathnote@fontfile{NotoSansCJK-VF.ttc}}{\mathnote@haslocalcjksanstrue}{\mathnote@haslocalcjksansfalse}
\IfFileExists{\mathnote@fontfile{NotoSansMonoCJK-VF.ttc}}{\mathnote@haslocalcjkmonotrue}{\mathnote@haslocalcjkmonofalse}
\IfFileExists{\mathnote@fontfile{LXGWWenKaiSC-Regular.ttf}}{\mathnote@haslocalkaitrue}{\mathnote@haslocalkaifalse}
\IfFileExists{\mathnote@fontfile{SourceHanSerifSC-Regular.otf}}{\mathnote@haslocalshseriftrue}{\mathnote@haslocalshseriffalse}
\IfFileExists{\mathnote@fontfile{SourceHanSansSC-Regular.otf}}{\mathnote@haslocalshsanstrue}{\mathnote@haslocalshsansfalse}

% CJK 斜体(用于中文“强调体”)优先使用霞鹜文楷
\newcommand{\mathnote@cjkitalicfont}{}
\newcommand{\mathnote@cjkitalicfeatures}{Language = Chinese Simplified}
\ifmathnote@haslocalkai
  \def\mathnote@cjkitalicfont{LXGWWenKaiSC-Regular}
  \def\mathnote@cjkitalicfeatures{Path = {\MathNoteFontDir}, Extension = .ttf, Language = Chinese Simplified}
\else
  \IfFontExistsTF{LXGW WenKai SC}{%
    \def\mathnote@cjkitalicfont{LXGW WenKai SC}%
    \def\mathnote@cjkitalicfeatures{Language = Chinese Simplified}
  }{%
    \def\mathnote@cjkitalicfont{}%
    \def\mathnote@cjkitalicfeatures{Language = Chinese Simplified}
  }%
\fi
\ifx\mathnote@cjkitalicfont\@empty
  \def\mathnote@cjkitalicfont{FandolKai}
  \def\mathnote@cjkitalicfeatures{Language = Chinese Simplified}
\fi

% 西文正文字体族:优先可变 Noto Serif,其次系统字体
\newcommand{\mathnote@setlatinfonts}{%
  \ifmathnote@haslocalserif
    \setmainfont{Noto Serif}[
      Path = {\MathNoteFontDir},
      Extension = .ttf,
      UprightFont = NotoSerif-VF,
      ItalicFont = NotoSerif-Italic-VF,
      BoldFont = NotoSerif-VF,
      BoldFeatures = {RawFeature={+wght=760}},
      BoldItalicFont = NotoSerif-Italic-VF,
      BoldItalicFeatures = {RawFeature={+wght=760}}
    ]%
  \else
    \IfFontExistsTF{Noto Serif}{%
      \setmainfont{Noto Serif}[
        ItalicFont = {Noto Serif Italic},
        BoldFont = {Noto Serif Bold},
        BoldItalicFont = {Noto Serif Bold Italic}
      ]%
    }{%
      \setmainfont{TeX Gyre Pagella}%
    }%
  \fi
  \ifmathnote@haslocalsans
    \setsansfont{Noto Sans}[
      Path = {\MathNoteFontDir},
      Extension = .ttf,
      UprightFont = NotoSans-VF,
      ItalicFont = NotoSans-Italic-VF,
      BoldFont = NotoSans-VF,
      BoldFeatures = {RawFeature={+wght=760}},
      BoldItalicFont = NotoSans-Italic-VF,
      BoldItalicFeatures = {RawFeature={+wght=760}}
    ]%
  \else
    \IfFontExistsTF{Noto Sans}{%
      \setsansfont{Noto Sans}[
        ItalicFont = {Noto Sans Italic},
        BoldFont = {Noto Sans Bold},
        BoldItalicFont = {Noto Sans Bold Italic}
      ]%
    }{%
      \setsansfont{TeX Gyre Heros}%
    }%
  \fi
  \ifmathnote@haslocalmono
    \setmonofont{Noto Sans Mono}[
      Path = {\MathNoteFontDir},
      Extension = .ttf,
      UprightFont = NotoSansMono-VF,
      BoldFont = NotoSansMono-VF,
      BoldFeatures = {RawFeature={+wght=740}},
      ItalicFont = NotoSansMono-VF,
      ItalicFeatures = {FakeSlant=0.2}
    ]%
  \else
    \IfFontExistsTF{Noto Sans Mono}{%
      \setmonofont{Noto Sans Mono}[
        BoldFont = {Noto Sans Mono Bold}
      ]%
    }{%
      \setmonofont{TeX Gyre Cursor}%
    }%
  \fi
}

% 中文主字体族:优先思源宋/黑与 Noto CJK,可退回 Fandol 系列
\newcommand{\mathnote@setcjkfonts}{%
  \ifmathnote@haslocalshserif
    \setCJKmainfont{SourceHanSerifSC-Regular}[
      Path = {\MathNoteFontDir},
      Extension = .otf,
      Language = Chinese Simplified,
      BoldFont = SourceHanSerifSC-Bold,
      ItalicFont = {\mathnote@cjkitalicfont},
      ItalicFeatures = {\mathnote@cjkitalicfeatures}
    ]%
    \setCJKfamilyfont{song}{SourceHanSerifSC-Regular}[
      Path = {\MathNoteFontDir},
      Extension = .otf,
      BoldFont = SourceHanSerifSC-Bold
    ]%
  \else
    \IfFontExistsTF{Source Han Serif SC}{%
      \setCJKmainfont{Source Han Serif SC}[Language=Chinese Simplified, ItalicFont={\mathnote@cjkitalicfont}, ItalicFeatures={\mathnote@cjkitalicfeatures}]%
      \setCJKfamilyfont{song}{Source Han Serif SC}[Language=Chinese Simplified]
    }{%
      \ifmathnote@haslocalcjkserif
        \setCJKmainfont{NotoSerifCJK-VF}[
          Path = {\MathNoteFontDir},
          Extension = .ttc,
          Language = Chinese Simplified,
          UprightFont = NotoSerifCJK-VF,
          UprightFeatures = {FontIndex=2},
          BoldFont = NotoSerifCJK-VF,
          BoldFeatures = {FontIndex=2,RawFeature={+wght=780}},
          AutoFakeSlant = 0.18,
          ItalicFont = {\mathnote@cjkitalicfont},
          ItalicFeatures = {\mathnote@cjkitalicfeatures}
        ]%
        \setCJKfamilyfont{song}{NotoSerifCJK-VF}[
          Path = {\MathNoteFontDir},
          Extension = .ttc,
          UprightFont = NotoSerifCJK-VF,
          UprightFeatures = {FontIndex=2},
          BoldFont = NotoSerifCJK-VF,
          BoldFeatures = {FontIndex=2,RawFeature={+wght=780}}
        ]%
      \else
        \IfFontExistsTF{Noto Serif CJK SC}{%
          \setCJKmainfont{Noto Serif CJK SC}[Language=Chinese Simplified, ItalicFont={\mathnote@cjkitalicfont}, ItalicFeatures={\mathnote@cjkitalicfeatures}]
          \setCJKfamilyfont{song}{Noto Serif CJK SC}[Language=Chinese Simplified]
        }{%
          \setCJKmainfont{FandolSong}[BoldFont={FandolSong-Bold}, ItalicFont={\mathnote@cjkitalicfont}, ItalicFeatures={\mathnote@cjkitalicfeatures}]
          \setCJKfamilyfont{song}{FandolSong}[BoldFont={FandolSong-Bold}]
        }%
      \fi
    }%
  \fi
  \ifmathnote@haslocalshsans
    \setCJKsansfont{SourceHanSansSC-Regular}[
      Path = {\MathNoteFontDir},
      Extension = .otf,
      Language = Chinese Simplified,
      BoldFont = SourceHanSansSC-Bold,
      ItalicFont = {\mathnote@cjkitalicfont},
      ItalicFeatures = {\mathnote@cjkitalicfeatures}
    ]%
    \setCJKfamilyfont{hei}{SourceHanSansSC-Regular}[
      Path = {\MathNoteFontDir},
      Extension = .otf,
      BoldFont = SourceHanSansSC-Bold
    ]%
  \else
    \IfFontExistsTF{Source Han Sans SC}{%
      \setCJKsansfont{Source Han Sans SC}[Language=Chinese Simplified, ItalicFont={\mathnote@cjkitalicfont}, ItalicFeatures={\mathnote@cjkitalicfeatures}]
      \setCJKfamilyfont{hei}{Source Han Sans SC}[Language=Chinese Simplified]
    }{%
      \ifmathnote@haslocalcjksans
        \setCJKsansfont{NotoSansCJK-VF}[
          Path = {\MathNoteFontDir},
          Extension = .ttc,
          Language = Chinese Simplified,
          UprightFont = NotoSansCJK-VF,
          UprightFeatures = {FontIndex=2},
          BoldFont = NotoSansCJK-VF,
          BoldFeatures = {FontIndex=2,RawFeature={+wght=780}},
          ItalicFont = {\mathnote@cjkitalicfont},
          ItalicFeatures = {\mathnote@cjkitalicfeatures}
        ]%
        \setCJKfamilyfont{hei}{NotoSansCJK-VF}[
          Path = {\MathNoteFontDir},
          Extension = .ttc,
          UprightFont = NotoSansCJK-VF,
          UprightFeatures = {FontIndex=2},
          BoldFont = NotoSansCJK-VF,
          BoldFeatures = {FontIndex=2,RawFeature={+wght=820}}
        ]%
      \else
        \IfFontExistsTF{Noto Sans CJK SC}{%
          \setCJKsansfont{Noto Sans CJK SC}[Language=Chinese Simplified, ItalicFont={\mathnote@cjkitalicfont}, ItalicFeatures={\mathnote@cjkitalicfeatures}]
          \setCJKfamilyfont{hei}{Noto Sans CJK SC}[Language=Chinese Simplified]
        }{%
          \setCJKsansfont{FandolHei}[ItalicFont={\mathnote@cjkitalicfont}, ItalicFeatures={\mathnote@cjkitalicfeatures}]
          \setCJKfamilyfont{hei}{FandolHei}
        }%
      \fi
    }%
  \fi
  \ifmathnote@haslocalshsans
    \setCJKmonofont{SourceHanSansSC-Regular}[
      Path = {\MathNoteFontDir},
      Extension = .otf,
      Language = Chinese Simplified,
      BoldFont = SourceHanSansSC-Bold,
      ItalicFont = {\mathnote@cjkitalicfont},
      ItalicFeatures = {\mathnote@cjkitalicfeatures}
    ]%
  \else
    \IfFontExistsTF{Source Han Sans SC}{%
      \setCJKmonofont{Source Han Sans SC}[Language=Chinese Simplified, ItalicFont={\mathnote@cjkitalicfont}, ItalicFeatures={\mathnote@cjkitalicfeatures}]
    }{%
      \ifmathnote@haslocalcjkmono
        \setCJKmonofont{NotoSansMonoCJK-VF}[
          Path = {\MathNoteFontDir},
          Extension = .ttc,
          Language = Chinese Simplified,
          UprightFont = NotoSansMonoCJK-VF,
          UprightFeatures = {FontIndex=2},
          BoldFont = NotoSansMonoCJK-VF,
          BoldFeatures = {FontIndex=2,RawFeature={+wght=760}},
          ItalicFont = {\mathnote@cjkitalicfont},
          ItalicFeatures = {\mathnote@cjkitalicfeatures}
        ]%
      \else
        \IfFontExistsTF{Noto Sans Mono CJK SC}{%
          \setCJKmonofont{Noto Sans Mono CJK SC}[Language=Chinese Simplified, ItalicFont={\mathnote@cjkitalicfont}, ItalicFeatures={\mathnote@cjkitalicfeatures}]
        }{%
          \setCJKmonofont{FandolFang}[ItalicFont={\mathnote@cjkitalicfont}, ItalicFeatures={\mathnote@cjkitalicfeatures}]
        }%
      \fi
    }%
  \fi
  \ifmathnote@haslocalkai
    \setCJKfamilyfont{kai}{LXGWWenKaiSC-Regular}[
      Path = {\MathNoteFontDir},
      Extension = .ttf,
      UprightFont = LXGWWenKaiSC-Regular,
      BoldFont = LXGWWenKaiSC-Medium,
      Language = Chinese Simplified
    ]%
    \setCJKfamilyfont{zhkai}{LXGWWenKaiSC-Regular}[
      Path = {\MathNoteFontDir},
      Extension = .ttf,
      UprightFont = LXGWWenKaiSC-Regular,
      BoldFont = LXGWWenKaiSC-Medium,
      Language = Chinese Simplified
    ]%
  \else
    \IfFontExistsTF{LXGW WenKai SC}{%
      \setCJKfamilyfont{kai}{LXGW WenKai SC}[Language=Chinese Simplified]
      \setCJKfamilyfont{zhkai}{LXGW WenKai SC}[Language=Chinese Simplified]
    }{%
      \setCJKfamilyfont{kai}{FandolKai}[Language=Chinese Simplified]
      \setCJKfamilyfont{zhkai}{FandolKai}[Language=Chinese Simplified]
    }%
  \fi
}

% 使用 xeCJK 管理中文字体与标点行为
\usepackage{xeCJK}
\xeCJKsetup{
  CheckSingle = true,
  RubberPunctSkip = true,
  PunctStyle = plain
}
\xeCJKsetwidth{,}{0.5em}
\xeCJKsetwidth{。}{1em}
% 避免段首 / 公式孤行
\clubpenalty=10000
\widowpenalty=10000
\displaywidowpenalty=10000

% 在导言末尾调用一次字体设置,确保主字体就绪
\mathnote@setlatinfonts
\mathnote@setcjkfonts
\renewcommand{\kaishu}{\CJKfamily{kai}}

% 版面微调与行距
\usepackage{microtype}
\usepackage{setspace}
\setstretch{1.15}

% 数学与图形相关常用宏包
\usepackage{amsmath, amssymb, amsthm, mathtools}
\usepackage{bm}
\usepackage{siunitx}
\usepackage{enumitem}
\usepackage{tikz}
\usetikzlibrary{calc, arrows.meta, decorations.pathmorphing, positioning}
\usepackage{xparse}
\usepackage{etoolbox}
\usepackage{graphicx}
\usepackage{svg}
\usepackage{caption}
\usepackage{booktabs}
\usepackage{tabularx}
\usepackage{multicol}
\usepackage{listings}
\usepackage{tcolorbox}
\tcbuselibrary{skins, breakable, hooks, listingsutf8}
\usepackage{zhnumber}
\usepackage{fancyhdr}
\usepackage{lastpage}
\usepackage{hyperref}
\usepackage{bookmark}
