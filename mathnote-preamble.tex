\ProvidesFile{mathnote-preamble.tex}[2024/11/13 Math note modern preamble]
\RequirePackage{iftex}
\ifPDFTeX
  \PackageError{mathnote-preamble}{请使用 \XeLaTeX\ 或 LuaLaTeX 编译该模板}{在 MikTeX / TeX Live 中切换到 \XeLaTeX\ (推荐) 或 LuaLaTeX 后重新编译。}
\fi
\makeatletter
\newif\ifmathnote@fontdirfound
\mathnote@fontdirfoundfalse
\def\mathnote@fontdircandidates{fonts/,./fonts/,../fonts/,../../fonts/,../../../fonts/}
\def\mathnote@fontdir{}
\@for\mathnote@cand:=\mathnote@fontdircandidates\do{%
  \ifmathnote@fontdirfound\else
    \IfFileExists{\mathnote@cand NotoSerif-VF.ttf}{%
      \edef\mathnote@fontdir{\mathnote@cand}%
      \mathnote@fontdirfoundtrue
    }{%
      \IfFileExists{\mathnote@cand SourceHanSerifSC-Regular.otf}{%
        \edef\mathnote@fontdir{\mathnote@cand}%
        \mathnote@fontdirfoundtrue
      }{}%
    }%
  \fi
}
\ifmathnote@fontdirfound\else
  \edef\mathnote@fontdir{fonts/}%
\fi
\@ifundefined{MathNoteFontDir}{%
  \edef\MathNoteFontDir{\mathnote@fontdir}%
}{}

% --------------------------------------------------
% User switches
% --------------------------------------------------
\newif\ifmathnoteprintmode
\mathnoteprintmodefalse
\newcommand{\MathNoteEnablePrint}{%
  \mathnoteprintmodetrue
  \mathnote@applypalette
}
\newif\ifmathnotereviewstamp
\mathnotereviewstampfalse
\newcommand{\mathnote@reviewstamppathbase}{assest/lzlxV-reviewed}
\newcommand{\mathnote@reviewstampsvg}{\mathnote@reviewstamppathbase.svg}
\newcommand{\mathnote@reviewstamppdf}{\mathnote@reviewstamppathbase.pdf}
\newcommand{\mathnote@reviewstampinclude}{}
\newif\ifmathnote@reviewstampplaced
\mathnote@reviewstampplacedfalse
\IfFileExists{\mathnote@reviewstamppdf}{%
  \renewcommand{\mathnote@reviewstampinclude}{%
    \includegraphics[width=20mm,height=20mm,keepaspectratio]{\mathnote@reviewstamppdf}%
  }%
}{%
  \IfFileExists{\mathnote@reviewstampsvg}{%
    \renewcommand{\mathnote@reviewstampinclude}{%
      \includesvg[width=20mm,height=20mm,keepaspectratio]{\mathnote@reviewstamppathbase}%
    }%
  }{}%
}
\newcommand{\mathnote@enablereviewstamp}{%
  \ifx\mathnote@reviewstampinclude\@empty
    \PackageWarning{mathnote-preamble}{Review stamp graphic \mathnote@reviewstampsvg\space (or PDF fallback) not found}%
  \else
    \mathnote@reviewstampplacedfalse
    \AddToHook{shipout/foreground}{%
      \ifmathnote@reviewstampplaced\else
        \begin{tikzpicture}[remember picture, overlay]
          \node[anchor=north east, xshift=-8mm, yshift=-8mm] at (current page.north east){%
            \mathnote@reviewstampinclude
          };
        \end{tikzpicture}%
        \global\mathnote@reviewstampplacedtrue
      \fi
    }%
  \fi
}
\newcommand{\MathNoteEnableReviewStamp}{%
  \mathnotereviewstamptrue
  \mathnote@enablereviewstamp
}

% --------------------------------------------------
% Metadata defaults (can be overwritten in each file)
% --------------------------------------------------
\providecommand{\notetitle}{数学学习笔记}
\providecommand{\noteauthor}{作者}
\providecommand{\notedate}{\today}
\providecommand{\notesubtitle}{现代数学排版示例}
\providecommand{\noteversion}{v1.0}

% --------------------------------------------------
% Core packages
% --------------------------------------------------
\usepackage{geometry}
\geometry{
  paper=a4paper,
  top=2.35cm,
  bottom=2.4cm,
  left=2.1cm,
  right=2.1cm,
  headheight=16pt,
  headsep=14pt
}

\usepackage{fontspec}
\usepackage{metalogo}
\defaultfontfeatures{Ligatures=TeX, Scale=MatchLowercase}

\newcommand{\mathnote@fontfile}[1]{\MathNoteFontDir#1}
\newif\ifmathnote@haslocalserif
\newif\ifmathnote@haslocalsans
\newif\ifmathnote@haslocalmono
\newif\ifmathnote@haslocalcjkserif
\newif\ifmathnote@haslocalcjksans
\newif\ifmathnote@haslocalcjkmono
\newif\ifmathnote@haslocalkai
\newif\ifmathnote@haslocalshserif
\newif\ifmathnote@haslocalshsans
\IfFileExists{\mathnote@fontfile{NotoSerif-VF.ttf}}{\mathnote@haslocalseriftrue}{\mathnote@haslocalseriffalse}
\IfFileExists{\mathnote@fontfile{NotoSans-VF.ttf}}{\mathnote@haslocalsanstrue}{\mathnote@haslocalsansfalse}
\IfFileExists{\mathnote@fontfile{NotoSansMono-VF.ttf}}{\mathnote@haslocalmonotrue}{\mathnote@haslocalmonofalse}
\IfFileExists{\mathnote@fontfile{NotoSerifCJK-VF.ttc}}{\mathnote@haslocalcjkseriftrue}{\mathnote@haslocalcjkseriffalse}
\IfFileExists{\mathnote@fontfile{NotoSansCJK-VF.ttc}}{\mathnote@haslocalcjksanstrue}{\mathnote@haslocalcjksansfalse}
\IfFileExists{\mathnote@fontfile{NotoSansMonoCJK-VF.ttc}}{\mathnote@haslocalcjkmonotrue}{\mathnote@haslocalcjkmonofalse}
\IfFileExists{\mathnote@fontfile{LXGWWenKaiSC-Regular.ttf}}{\mathnote@haslocalkaitrue}{\mathnote@haslocalkaifalse}
\IfFileExists{\mathnote@fontfile{SourceHanSerifSC-Regular.otf}}{\mathnote@haslocalshseriftrue}{\mathnote@haslocalshseriffalse}
\IfFileExists{\mathnote@fontfile{SourceHanSansSC-Regular.otf}}{\mathnote@haslocalshsanstrue}{\mathnote@haslocalshsansfalse}

\newcommand{\mathnote@cjkitalicfont}{}
\newcommand{\mathnote@cjkitalicfeatures}{Language = Chinese Simplified}
\ifmathnote@haslocalkai
  \def\mathnote@cjkitalicfont{LXGWWenKaiSC-Regular}
  \def\mathnote@cjkitalicfeatures{Path = {\MathNoteFontDir}, Extension = .ttf, Language = Chinese Simplified}
\else
  \IfFontExistsTF{LXGW WenKai SC}{%
    \def\mathnote@cjkitalicfont{LXGW WenKai SC}%
    \def\mathnote@cjkitalicfeatures{Language = Chinese Simplified}
  }{%
    \def\mathnote@cjkitalicfont{}%
    \def\mathnote@cjkitalicfeatures{Language = Chinese Simplified}
  }%
\fi
\ifx\mathnote@cjkitalicfont\@empty
  \def\mathnote@cjkitalicfont{FandolKai}
  \def\mathnote@cjkitalicfeatures{Language = Chinese Simplified}
\fi

\newcommand{\mathnote@setlatinfonts}{%
  \ifmathnote@haslocalserif
    \setmainfont{Noto Serif}[
      Path = {\MathNoteFontDir},
      Extension = .ttf,
      UprightFont = NotoSerif-VF,
      ItalicFont = NotoSerif-Italic-VF,
      BoldFont = NotoSerif-VF,
      BoldFeatures = {RawFeature={+wght=760}},
      BoldItalicFont = NotoSerif-Italic-VF,
      BoldItalicFeatures = {RawFeature={+wght=760}}
    ]%
  \else
    \IfFontExistsTF{Noto Serif}{%
      \setmainfont{Noto Serif}[
        ItalicFont = {Noto Serif Italic},
        BoldFont = {Noto Serif Bold},
        BoldItalicFont = {Noto Serif Bold Italic}
      ]%
    }{%
      \setmainfont{TeX Gyre Pagella}%
    }%
  \fi
  \ifmathnote@haslocalsans
    \setsansfont{Noto Sans}[
      Path = {\MathNoteFontDir},
      Extension = .ttf,
      UprightFont = NotoSans-VF,
      ItalicFont = NotoSans-Italic-VF,
      BoldFont = NotoSans-VF,
      BoldFeatures = {RawFeature={+wght=760}},
      BoldItalicFont = NotoSans-Italic-VF,
      BoldItalicFeatures = {RawFeature={+wght=760}}
    ]%
  \else
    \IfFontExistsTF{Noto Sans}{%
      \setsansfont{Noto Sans}[
        ItalicFont = {Noto Sans Italic},
        BoldFont = {Noto Sans Bold},
        BoldItalicFont = {Noto Sans Bold Italic}
      ]%
    }{%
      \setsansfont{TeX Gyre Heros}%
    }%
  \fi
  \ifmathnote@haslocalmono
    \setmonofont{Noto Sans Mono}[
      Path = {\MathNoteFontDir},
      Extension = .ttf,
      UprightFont = NotoSansMono-VF,
      BoldFont = NotoSansMono-VF,
      BoldFeatures = {RawFeature={+wght=740}},
      ItalicFont = NotoSansMono-VF,
      ItalicFeatures = {FakeSlant=0.2}
    ]%
  \else
    \IfFontExistsTF{Noto Sans Mono}{%
      \setmonofont{Noto Sans Mono}[
        BoldFont = {Noto Sans Mono Bold}
      ]%
    }{%
      \setmonofont{TeX Gyre Cursor}%
    }%
  \fi
}

\newcommand{\mathnote@setcjkfonts}{%
  \ifmathnote@haslocalshserif
    \setCJKmainfont{SourceHanSerifSC-Regular}[
      Path = {\MathNoteFontDir},
      Extension = .otf,
      Language = Chinese Simplified,
      BoldFont = SourceHanSerifSC-Bold,
      ItalicFont = {\mathnote@cjkitalicfont},
      ItalicFeatures = {\mathnote@cjkitalicfeatures}
    ]%
  \else
    \IfFontExistsTF{Source Han Serif SC}{%
      \setCJKmainfont{Source Han Serif SC}[Language=Chinese Simplified, ItalicFont={\mathnote@cjkitalicfont}, ItalicFeatures={\mathnote@cjkitalicfeatures}]%
    }{%
      \ifmathnote@haslocalcjkserif
        \setCJKmainfont{NotoSerifCJK-VF}[
          Path = {\MathNoteFontDir},
          Extension = .ttc,
          Language = Chinese Simplified,
          UprightFont = NotoSerifCJK-VF,
          UprightFeatures = {FontIndex=2},
          BoldFont = NotoSerifCJK-VF,
          BoldFeatures = {FontIndex=2,RawFeature={+wght=780}},
          AutoFakeSlant = 0.18,
          ItalicFont = {\mathnote@cjkitalicfont},
          ItalicFeatures = {\mathnote@cjkitalicfeatures}
        ]%
      \else
        \IfFontExistsTF{Noto Serif CJK SC}{%
          \setCJKmainfont{Noto Serif CJK SC}[Language=Chinese Simplified, ItalicFont={\mathnote@cjkitalicfont}, ItalicFeatures={\mathnote@cjkitalicfeatures}]
        }{%
          \setCJKmainfont{FandolSong}[BoldFont={FandolSong-Bold}, ItalicFont={\mathnote@cjkitalicfont}, ItalicFeatures={\mathnote@cjkitalicfeatures}]
        }%
      \fi
    }%
  \fi
  \ifmathnote@haslocalshsans
    \setCJKsansfont{SourceHanSansSC-Regular}[
      Path = {\MathNoteFontDir},
      Extension = .otf,
      Language = Chinese Simplified,
      BoldFont = SourceHanSansSC-Bold,
      ItalicFont = {\mathnote@cjkitalicfont},
      ItalicFeatures = {\mathnote@cjkitalicfeatures}
    ]%
    \setCJKfamilyfont{hei}{SourceHanSansSC-Regular}[
      Path = {\MathNoteFontDir},
      Extension = .otf,
      BoldFont = SourceHanSansSC-Bold
    ]%
  \else
    \IfFontExistsTF{Source Han Sans SC}{%
      \setCJKsansfont{Source Han Sans SC}[Language=Chinese Simplified, ItalicFont={\mathnote@cjkitalicfont}, ItalicFeatures={\mathnote@cjkitalicfeatures}]
      \setCJKfamilyfont{hei}{Source Han Sans SC}[Language=Chinese Simplified]
    }{%
      \ifmathnote@haslocalcjksans
        \setCJKsansfont{NotoSansCJK-VF}[
          Path = {\MathNoteFontDir},
          Extension = .ttc,
          Language = Chinese Simplified,
          UprightFont = NotoSansCJK-VF,
          UprightFeatures = {FontIndex=2},
          BoldFont = NotoSansCJK-VF,
          BoldFeatures = {FontIndex=2,RawFeature={+wght=780}},
          ItalicFont = {\mathnote@cjkitalicfont},
          ItalicFeatures = {\mathnote@cjkitalicfeatures}
        ]%
        \setCJKfamilyfont{hei}{NotoSansCJK-VF}[
          Path = {\MathNoteFontDir},
          Extension = .ttc,
          UprightFont = NotoSansCJK-VF,
          UprightFeatures = {FontIndex=2},
          BoldFont = NotoSansCJK-VF,
          BoldFeatures = {FontIndex=2,RawFeature={+wght=820}}
        ]%
      \else
        \IfFontExistsTF{Noto Sans CJK SC}{%
          \setCJKsansfont{Noto Sans CJK SC}[Language=Chinese Simplified, ItalicFont={\mathnote@cjkitalicfont}, ItalicFeatures={\mathnote@cjkitalicfeatures}]
          \setCJKfamilyfont{hei}{Noto Sans CJK SC}[Language=Chinese Simplified]
        }{%
          \setCJKsansfont{FandolHei}[ItalicFont={\mathnote@cjkitalicfont}, ItalicFeatures={\mathnote@cjkitalicfeatures}]
          \setCJKfamilyfont{hei}{FandolHei}
        }%
      \fi
    }%
  \fi
  \ifmathnote@haslocalshsans
    \setCJKmonofont{SourceHanSansSC-Regular}[
      Path = {\MathNoteFontDir},
      Extension = .otf,
      Language = Chinese Simplified,
      BoldFont = SourceHanSansSC-Bold,
      ItalicFont = {\mathnote@cjkitalicfont},
      ItalicFeatures = {\mathnote@cjkitalicfeatures}
    ]%
  \else
    \IfFontExistsTF{Source Han Sans SC}{%
      \setCJKmonofont{Source Han Sans SC}[Language=Chinese Simplified, ItalicFont={\mathnote@cjkitalicfont}, ItalicFeatures={\mathnote@cjkitalicfeatures}]
    }{%
      \ifmathnote@haslocalcjkmono
        \setCJKmonofont{NotoSansMonoCJK-VF}[
          Path = {\MathNoteFontDir},
          Extension = .ttc,
          Language = Chinese Simplified,
          UprightFont = NotoSansMonoCJK-VF,
          UprightFeatures = {FontIndex=2},
          BoldFont = NotoSansMonoCJK-VF,
          BoldFeatures = {FontIndex=2,RawFeature={+wght=760}},
          ItalicFont = {\mathnote@cjkitalicfont},
          ItalicFeatures = {\mathnote@cjkitalicfeatures}
        ]%
      \else
        \IfFontExistsTF{Noto Sans Mono CJK SC}{%
          \setCJKmonofont{Noto Sans Mono CJK SC}[Language=Chinese Simplified, ItalicFont={\mathnote@cjkitalicfont}, ItalicFeatures={\mathnote@cjkitalicfeatures}]
        }{%
          \setCJKmonofont{FandolFang}[ItalicFont={\mathnote@cjkitalicfont}, ItalicFeatures={\mathnote@cjkitalicfeatures}]
        }%
      \fi
    }%
  \fi
  \ifmathnote@haslocalkai
    \setCJKfamilyfont{kai}{LXGW WenKai SC}[
      Path = {\MathNoteFontDir},
      Extension = .ttf,
      UprightFont = LXGWWenKaiSC-Regular,
      BoldFont = LXGWWenKaiSC-Medium,
      AutoFakeBold = 1.25,
      Language = Chinese Simplified
    ]%
    \setCJKfamilyfont{zhkai}{LXGW WenKai SC}[
      Path = {\MathNoteFontDir},
      Extension = .ttf,
      UprightFont = LXGWWenKaiSC-Regular,
      BoldFont = LXGWWenKaiSC-Medium,
      AutoFakeBold = 1.25,
      Language = Chinese Simplified
    ]%
  \else
    \IfFontExistsTF{LXGW WenKai SC}{%
      \setCJKfamilyfont{kai}{LXGW WenKai SC}[Language=Chinese Simplified]
      \setCJKfamilyfont{zhkai}{LXGW WenKai SC}[Language=Chinese Simplified]
    }{%
      \setCJKfamilyfont{kai}{FandolKai}[Language=Chinese Simplified]
      \setCJKfamilyfont{zhkai}{FandolKai}[Language=Chinese Simplified]
    }%
  \fi
}

\mathnote@setlatinfonts
\mathnote@setcjkfonts
\renewcommand{\kaishu}{\CJKfamily{kai}}
\xeCJKsetup{
  CheckSingle = true,
  RubberPunctSkip = true,
  PunctStyle = plain
}
\xeCJKsetwidth{,}{0.5em}
\xeCJKsetwidth{。}{1em}
\newlength{\mathnote@commaspace}
\setlength{\mathnote@commaspace}{0.5em}
\catcode`,=\active
\protected\def,{,\kern\mathnote@commaspace}
\clubpenalty=10000
\widowpenalty=10000
\displaywidowpenalty=10000

\usepackage{microtype}
\usepackage{setspace}
\setstretch{1.15}

\usepackage{amsmath, amssymb, amsthm, mathtools}
\usepackage{bm}
\usepackage{siunitx}
\usepackage{enumitem}
\usepackage{tikz}
\usetikzlibrary{calc, arrows.meta, decorations.pathmorphing, positioning}
\usepackage{xparse}
\usepackage{etoolbox}
\usepackage{graphicx}
\usepackage{svg}
\usepackage{caption}
\usepackage{booktabs}
\usepackage{tabularx}
\usepackage{multicol}
\usepackage{listings}
\usepackage{tcolorbox}
\tcbuselibrary{skins, breakable, hooks, listingsutf8}
\usepackage{zhnumber}
\usepackage{fancyhdr}
\usepackage{lastpage}
\usepackage{hyperref}
\usepackage{bookmark}

% Block-style paragraph headings to avoid run-in overfull boxes
\renewcommand{\paragraph}{%
  \@startsection{paragraph}{4}{\z@}%
    {1.5ex \@plus 0.5ex \@minus 0.2ex}%
    {0.65em}%
    {\normalfont\normalsize\bfseries}%
}

% --------------------------------------------------
% Colors
% --------------------------------------------------
\definecolor{screenAccent}{HTML}{1565C0}
\definecolor{screenSecondary}{HTML}{00897B}
\definecolor{screenHighlight}{HTML}{F9A826}
\definecolor{screenInfo}{HTML}{546E7A}
\definecolor{screenSurface}{HTML}{FFFFFF}

\definecolor{printAccent}{cmyk}{0.95,0.55,0,0.05}
\definecolor{printSecondary}{cmyk}{0.82,0,0.56,0.08}
\definecolor{printHighlight}{cmyk}{0,0.35,0.80,0}
\definecolor{printInfo}{cmyk}{0.60,0.47,0.43,0.20}
\definecolor{printSurface}{cmyk}{0,0,0,0}
\definecolor{mathnotePureCyan}{cmyk}{1,0,0,0}

% 16-color palettes for screen (sRGB)
\definecolor{screenTone01}{HTML}{0D47A1}
\definecolor{screenTone02}{HTML}{1565C0}
\definecolor{screenTone03}{HTML}{1A73E8}
\definecolor{screenTone04}{HTML}{2196F3}
\definecolor{screenTone05}{HTML}{00ACC1}
\definecolor{screenTone06}{HTML}{00897B}
\definecolor{screenTone07}{HTML}{2E7D32}
\definecolor{screenTone08}{HTML}{558B2F}
\definecolor{screenTone09}{HTML}{9E9D24}
\definecolor{screenTone10}{HTML}{F9A825}
\definecolor{screenTone11}{HTML}{FFB300}
\definecolor{screenTone12}{HTML}{FB8C00}
\definecolor{screenTone13}{HTML}{F4511E}
\definecolor{screenTone14}{HTML}{D84315}
\definecolor{screenTone15}{HTML}{8E24AA}
\definecolor{screenTone16}{HTML}{6A1B9A}

% 16-color palettes for print (CMYK approximations)
\definecolor{printTone01}{cmyk}{1,0.72,0,0.35}
\definecolor{printTone02}{cmyk}{0.9,0.5,0,0.2}
\definecolor{printTone03}{cmyk}{0.85,0.45,0,0.12}
\definecolor{printTone04}{cmyk}{0.65,0.25,0,0.02}
\definecolor{printTone05}{cmyk}{0.75,0.05,0.1,0.05}
\definecolor{printTone06}{cmyk}{0.85,0,0.35,0.2}
\definecolor{printTone07}{cmyk}{0.75,0,0.8,0.38}
\definecolor{printTone08}{cmyk}{0.6,0,1,0.42}
\definecolor{printTone09}{cmyk}{0.35,0,1,0.45}
\definecolor{printTone10}{cmyk}{0,0.2,1,0.02}
\definecolor{printTone11}{cmyk}{0,0.25,1,0}
\definecolor{printTone12}{cmyk}{0,0.45,1,0}
\definecolor{printTone13}{cmyk}{0,0.7,0.8,0}
\definecolor{printTone14}{cmyk}{0,0.85,0.95,0.1}
\definecolor{printTone15}{cmyk}{0.45,0.9,0,0}
\definecolor{printTone16}{cmyk}{0.6,1,0,0.1}

\colorlet{accent}{screenAccent}
\colorlet{secondary}{screenSecondary}
\colorlet{highlight}{screenHighlight}
\colorlet{inkgray}{screenInfo}
\colorlet{surface}{screenSurface}

\newif\ifmathnote@docstarted
\mathnote@docstartedfalse
\AtBeginDocument{\mathnote@docstartedtrue}

\newcommand{\mathnote@applyhypercolors}{%
  \ifmathnoteprintmode
    \hypersetup{
      colorlinks=false,
      hidelinks,
      pdfborderstyle={/S/U/W 0},
      pdfborder={0 0 0}
    }%
  \else
    \hypersetup{
      colorlinks=true,
      linkcolor=accent,
      citecolor=secondary,
      urlcolor=accent,
      pdfborder={0 0 0}
    }%
  \fi
}

\newcommand{\mathnote@applypalette}{%
  \ifmathnoteprintmode
    \colorlet{accent}{printAccent}%
    \colorlet{secondary}{printSecondary}%
    \colorlet{highlight}{printHighlight}%
    \colorlet{inkgray}{printInfo}%
    \colorlet{surface}{printSurface}%
  \else
    \colorlet{accent}{screenAccent}%
    \colorlet{secondary}{screenSecondary}%
    \colorlet{highlight}{screenHighlight}%
    \colorlet{inkgray}{screenInfo}%
    \colorlet{surface}{screenSurface}%
  \fi
  \colorlet{accentline}{accent!65!black}%
  \colorlet{accentbg}{accent!8!white}%
  \colorlet{secondarybg}{secondary!10!white}%
  \colorlet{highlightbg}{highlight!10!white}%
  \colorlet{inkline}{inkgray!60!black}%
  \colorlet{surfacegrid}{inkgray!6!white}%
  \ifmathnote@docstarted
    \mathnote@applyhypercolors
  \else
    \AtBeginDocument{\mathnote@applyhypercolors}
  \fi
}
\mathnote@applypalette
\newcommand{\MathNoteRefreshColors}{\mathnote@applypalette}
\AtBeginDocument{%
  \hypersetup{%
    pdftitle=\notetitle,
    pdfauthor=\noteauthor,
    pdfsubject=\notesubtitle,
    pdfcreator={MathNote dual-medium template}%
  }%
}

% --------------------------------------------------
% Sectioning and spacing
% --------------------------------------------------
\setlength{\parskip}{0.35em}
\setlength{\parindent}{2em}
\newlength{\mathnote@boxindent}
\setlength{\mathnote@boxindent}{\parindent}
\setcounter{secnumdepth}{3}
\setcounter{tocdepth}{2}


\ctexset{
  section={
    name={第,节},
    format+=\Large\sffamily\bfseries\color{accent},
    beforeskip=1.2em,
    afterskip=0.7em
  },
  subsection={
    format+=\large\sffamily\bfseries\color{secondary},
    beforeskip=1em,
    afterskip=0.4em
  },
  subsubsection={
    format+=\normalsize\sffamily\bfseries\color{inkgray},
    beforeskip=0.8em,
    afterskip=0.2em
  }
}

% --------------------------------------------------
% Header / footer
% --------------------------------------------------
\pagestyle{fancy}
\fancyhf{}
\fancyhead[LE]{\small\sffamily\textcolor{accent}{\notetitle\ >\ \nouppercase{\rightmark}}}
\fancyhead[RO]{\small\sffamily\textcolor{accent}{\notetitle\ >\ \nouppercase{\rightmark}}}
\fancyfoot[LE]{\small\sffamily\textcolor{inkgray}{\thepage}\ \textcolor{mathnotePureCyan}{/}\ \textcolor{inkgray}{\pageref{LastPage}}}
\fancyfoot[RO]{\small\sffamily\textcolor{inkgray}{\thepage}\ \textcolor{mathnotePureCyan}{/}\ \textcolor{inkgray}{\pageref{LastPage}}}
\fancyfoot[LO]{}
\fancyfoot[RE]{}
\renewcommand{\headrulewidth}{0.2pt}
\renewcommand{\footrulewidth}{0pt}
\renewcommand{\sectionmark}[1]{\markright{#1}}

% --------------------------------------------------
% Math helpers
% --------------------------------------------------
\DeclarePairedDelimiter\abs{\lvert}{\rvert}
\DeclarePairedDelimiter\norm{\lVert}{\rVert}
\DeclarePairedDelimiter\ceil{\lceil}{\rceil}
\DeclarePairedDelimiter\floor{\lfloor}{\rfloor}

\newcommand{\R}{\mathbb{R}}
\newcommand{\C}{\mathbb{C}}
\newcommand{\Q}{\mathbb{Q}}
\newcommand{\Z}{\mathbb{Z}}
\newcommand{\N}{\mathbb{N}}
\newcommand{\dd}{\mathop{}\!\mathrm{d}}
\newcommand{\ee}{\mathrm{e}}
\newcommand{\dv}[2]{\frac{\dd #1}{\dd #2}}
\newcommand{\pdv}[2]{\frac{\partial #1}{\partial #2}}

\lstset{
  backgroundcolor=\color{surfacegrid},
  basicstyle=\ttfamily\small,
  keywordstyle=\color{screenTone04}\bfseries,
  commentstyle=\color{screenTone06},
  stringstyle=\color{screenTone12},
  frame=none,
  columns=fullflexible,
  showstringspaces=false
}

\NewDocumentCommand{\keyword}{m}{%
  \textcolor{accent}{\textbf{#1}}%
}

\NewDocumentCommand{\inlinehint}{m}{%
  \textcolor{secondary}{\sffamily\footnotesize #1}%
}

\NewDocumentCommand{\MathNotePaletteSwatch}{mm}{%
  \tikz[baseline=(label.base)]{
    \node[rounded corners=2pt, draw=#1!65!black, fill=#1, minimum width=0.85cm, minimum height=0.4cm] (chip) {};
    \node[right=0.28cm of chip, anchor=west, font=\sffamily\scriptsize\color{inkgray}] (label) {#2};
  }%
}

\NewDocumentCommand{\ModeBadge}{O{accent}m}{%
  \tikz[baseline=(label.base)]\node[label/.style={}] (label) [inner xsep=6pt, inner ysep=1.6pt, rounded corners=2pt, fill=#1!12!white, draw=#1!80!black, font=\sffamily\scriptsize\bfseries\color{#1!30!black}] {#2};%
}

\newcommand{\mathnote@ifblank}[3]{%
  \if\relax\detokenize{#1}\relax
    #2%
  \else
    #3%
  \fi
}

\newenvironment{focuspoints}{%
  \begin{itemize}[label=\tikz{\filldraw[accent] (0,0) circle (2pt);}, leftmargin=1.8em, itemsep=0.2em, topsep=0.1em]
}{\end{itemize}}

\newcounter{roadmapstep}
\newlength{\mathnote@roadmapindent}
\setlength{\mathnote@roadmapindent}{1.4em}
\newcommand{\mathnote@roadmaparrow}{%
  \par\smallskip
  \noindent\hspace{1.7em}\tikz{
    \draw[accent, line width=0.85pt, -{Latex[length=3mm]}] (0,0) -- (0,-0.9);
  }%
  \par\smallskip
}
\NewDocumentEnvironment{roadmap}{O{}}{%
  \par\smallskip
  \setcounter{roadmapstep}{0}%
}{%
  \par\smallskip
}
\newcommand{\RoadmapStep}[1]{%
  \stepcounter{roadmapstep}%
  \ifnum\value{roadmapstep}>1
    \mathnote@roadmaparrow
  \fi
  {%
    \noindent\parfillskip=0pt plus 1fil\ModeBadge[accent]{第\zhnumber{\value{roadmapstep}}步}\par
    \vspace{0.2em}%
    \noindent\hspace{\mathnote@roadmapindent}%
    \begin{minipage}[t]{\dimexpr\linewidth-\mathnote@roadmapindent\relax}
      \raggedright\sloppy #1
    \end{minipage}\par
  }%
}

% --------------------------------------------------
% Box styles
% --------------------------------------------------
\tcbset{
  mathnote box/.style={
    enhanced,
    sharp corners,
    boxrule=0.5pt,
    colback=surface,
    coltitle=inkgray,
    fonttitle=\sffamily\bfseries,
    left=1em,
    right=1em,
    top=0.7em,
    bottom=0.7em,
    before skip=10pt,
    after skip=10pt,
    breakable,
    width=\dimexpr\linewidth-\mathnote@boxindent\relax,
    left skip=\mathnote@boxindent,
    borderline west={1pt}{0pt}{accentline}
  }
}
\newtcolorbox{definitionbox}[2][]{%
  mathnote box,
  title=\mathnote@ifblank{#2}{定义}{#2},
  colback=surface,
  colframe=secondary!70!black,
  coltitle=secondary!15!white,
  fonttitle=\sffamily\bfseries\color{secondary!35!white},
  borderline west={2pt}{0pt}{secondary},
  #1
}

\newtcolorbox{theorembox}[2][]{%
  mathnote box,
  title=\mathnote@ifblank{#2}{定理}{#2},
  colback=surface,
  colframe=accent!70!black,
  coltitle=accent!10!white,
  fonttitle=\sffamily\bfseries\color{accent!35!white},
  borderline west={2pt}{0pt}{accent},
  #1
}

\newtcolorbox{examplebox}[2][]{%
  mathnote box,
  title=\mathnote@ifblank{#2}{例题}{#2},
  colback=surface,
  colframe=highlight!80!black,
  coltitle=highlight!15!white,
  fonttitle=\sffamily\bfseries\color{highlight!40!white},
  borderline west={2pt}{0pt}{highlight},
  #1
}

\newtcolorbox{lemmabox}[2][]{%
  mathnote box,
  title=\mathnote@ifblank{#2}{引理}{#2},
  colback=surface,
  colframe=inkline,
  coltitle=inkgray!30!white,
  fonttitle=\sffamily\bfseries\color{inkgray!45!white},
  borderline west={2pt}{0pt}{inkgray},
  #1
}

\newtcolorbox{notebox}[2][]{%
  mathnote box,
  title=\mathnote@ifblank{#2}{提示}{#2},
  colback=surface,
  colframe=highlight!60!black,
  coltitle=highlight!20!white,
  fonttitle=\sffamily\bfseries\color{highlight!45!white},
  borderline west={2pt}{0pt}{highlight},
  #1
}

\newtcolorbox{summarybox}[2][]{%
  mathnote box,
  title=\mathnote@ifblank{#2}{总结}{#2},
  colback=surface,
  colframe=accent!20!black,
  borderline west={2pt}{0pt}{accent},
  coltitle=accent!10!white,
  fonttitle=\sffamily\bfseries\color{accent!45!white},
  #1
}

\newtcolorbox{conceptbox}[2][]{%
  mathnote box,
  title=\mathnote@ifblank{#2}{概念骨架}{#2},
  colback=surface,
  colframe=secondary!40!black,
  coltitle=secondary!15!white,
  fonttitle=\sffamily\bfseries\color{secondary!40!white},
  borderline west={2pt}{0pt}{secondary},
  #1
}

\newtcolorbox{proofbox}[2][]{%
  mathnote box,
  title=\mathnote@ifblank{#2}{证明}{#2},
  colback=surface,
  colframe=inkline,
  coltitle=inkgray!35!white,
  fonttitle=\sffamily\bfseries\color{inkgray!60!white},
  borderline west={2pt}{0pt}{inkline},
  #1
}

\newtcolorbox{warningbox}[2][]{%
  mathnote box,
  title=\mathnote@ifblank{#2}{排版警示}{#2},
  colback=surface,
  colframe=highlight!80!black,
  borderline west={2pt}{0pt}{highlight},
  coltitle=highlight!20!white,
  fonttitle=\sffamily\bfseries\color{highlight!45!white},
  #1
}

% --------------------------------------------------
% TikZ styles
% --------------------------------------------------
\tikzset{
  mathnote lines/.style={
    line width=0.8pt,
    >=Stealth,
    draw=accentline,
    text=inkgray
  },
  mathnote grid/.style={
    color=inkgray!30,
    line width=0.3pt
  }
}

% --------------------------------------------------
% Tables and lists
% --------------------------------------------------
\newcommand{\mathnote@listbarbegin}[2][0.8em]{%
  \par\noindent
  \begin{tcolorbox}[
    blanker,
    enhanced,
    sharp corners,
    boxrule=0pt,
    colback=surface,
    left=#1,
    right=0pt,
    top=0.25em,
    bottom=0.25em,
    borderline west={1.3pt}{0pt}{#2}
  ]%
  \ignorespaces
}
\newcommand{\mathnote@listbarend}{\end{tcolorbox}\ignorespacesafterend}

\setlist[itemize]{leftmargin=1.8em, itemsep=0.25em, before=\mathnote@listbarbegin{accent}, after=\mathnote@listbarend}
\setlist[enumerate]{leftmargin=2.1em, itemsep=0.3em, label=\textbf{\arabic*.}, before=\mathnote@listbarbegin[1em]{secondary}, after=\mathnote@listbarend}
\setlist[description]{font=\sffamily\bfseries, labelsep=0.5em}

\renewcommand{\arraystretch}{1.2}
\captionsetup{font=small, labelfont=bf}

% --------------------------------------------------
% Utility commands
% --------------------------------------------------
\newcommand{\ScreenOnly}[1]{\ifmathnoteprintmode\else #1\fi}
\newcommand{\PrintOnly}[1]{\ifmathnoteprintmode #1\fi}
\newcommand{\DualMode}[2]{\ifmathnoteprintmode #2\else #1\fi}

\NewDocumentCommand{\PageTag}{O{accent}m}{%
  \begin{tikzpicture}[remember picture, overlay]
    \node[anchor=north east, xshift=-6mm, yshift=-10mm, fill=#1, text=white, rounded corners=2pt, inner xsep=6pt, inner ysep=2pt, font=\sffamily\footnotesize] at (current page.north east) {#2};
  \end{tikzpicture}%
}

\newcommand{\SectionTag}[1]{%
  \textcolor{accent}{\Large\bfseries\sffamily #1}%
}

\makeatother
